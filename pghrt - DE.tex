\documentclass{article}
\usepackage{hyperref}
\usepackage{float}
\usepackage{csquotes}
\usepackage[style=iso]{datetime2}
\usepackage[usenames,dvipsnames]{color}
\usepackage{booktabs}
  \setlength\heavyrulewidth{0.20ex}
  \setlength\cmidrulewidth{0.10ex}
  \setlength\lightrulewidth{0.10ex}

\usepackage[font=normalsize,labelfont={bf}]{caption}
  \captionsetup[table]{aboveskip=3pt}

\hypersetup{
    colorlinks=true,
    linkcolor=blue,
    filecolor=magenta,      
    urlcolor=magenta,
 }
 
\usepackage{graphicx}
\graphicspath{ {./img/} }
\renewcommand{\abstractname}{DISCLAIMER}
\title{EINE PRAKTISCHE ANLEITUNG ZUR FEMINISIERENDEN HORMONTHERAPIE}
\author{\href{https://katea.gay/}{Katie Tightpussy}}
\date{\today}
\setcounter{section}{-1}
\urlstyle{same}

\begin{document}


\maketitle
\tableofcontents
\section{VORWORT ZUR ÜBERSETZUNG}

Im deutschsprachigen Raum war bisher keine umfassende, aktuelle, verständlich geschriebene Informationssammlung zur feminisierenden Hormontherapie verfügbar. Diese Übersetzung des hervorragenden Werks von Katie Tightpussy ist ein Versuch, diesen Mangel auszugleichen. Nicht alle sind in der Lage, englischsprachige Anleitungen und Übersichten zu verstehen. Die KI-gestützte Übersetzung, trotz ihrer Vorzüge hinsichtlich der leichten Handhabung und Schnelligkeit, ist für diese Aufgabe ebenfalls ungeeignet, da viele der Nuancen und Feinheiten des Textinhalts dabei verloren gehen. Diese in liebevoller Handarbeit entstandene Übersetzung soll es allen deutschsprachigen Transpersonen (und weiteren interessierten Menschen) ermöglichen, sich selbst zu bilden und informierte Entscheidungen treffen zu können. 


Im Text wird, wenn notwendig, durch die Abkürzung "A.d.Ü." (Anmerkung der Übersetzerinnen) auf besondere Entscheidungen bei der Übersetzung oder auf Unklarheiten verwiesen.


Für etwaige Fehler entschuldigen sich die Übersetzerinnen im Voraus. Kritik und Lobgesänge aller Art werden auf Bluesky bei @camille.gay und @synthiecat.bsky.social dankend angenommen.

Virgo Sine Ruga, a.k.a. Camille, Oktober 2025
synthie aka. Fun_Tell_7441 aka. Lina, Dez. 2025/Jan. 2026

\begin{abstract}
    Ich bin keine Ärztin. Ich arbeite nicht im medizinischen Bereich. Ich bin in keiner Weise medizinisch ausgebildet. Ich bin eine Nichtfachfrau, die nichtfachmännische Meinungen anbietet, auf der Basis und im Umfang meiner eigenen Erfahrungen. Alle folgenden Informationen und Behauptungen sollten dementsprechend als bloße Meinungsäußerungen verstanden werden, nicht als Fakten oder medizinische Ratschläge. Diese Anleitung gibt der in der Community erarbeiteten moralische Wahrheit den Vorzug, wo die Wissenschaft noch nicht soweit ist. Kurzum: Sei nicht böse auf mich. 
\end{abstract}


\section{VORWORT}

Der Zweck dieses Dokuments ist die Katalogisierung meiner Gedanken und Meinungen zur feminisierenden Hormontherapie (A.d.Ü.: zur besseren Übersichtlichkeit ab jetzt mit "HRT" abgekürzt), da die verschiedenen Wikis der Community meiner Meinung nach ungeeignet sind. Diese sind zwar wertvolle Ressourcen, aus meiner Sicht jedoch für Personen ungeeignet die eine klare, handlungsorientierte Anleitung suchen, statt umfassende Information und Diagramme zu allen biologischen Vorgängen. Mein Ziel ist es, eine umfassende und schnell erfassbare Anleitung mit nachvollziehbaren Antworten zu den meisten Fragen zur Durchführung von HRT verfügbar zu machen. Diese lebensrettende Medizin soll sowohl für Menschen, die HRT in Betracht ziehen, als auch für erfahrene Transpersonen entmystifiziert werden. Somit setze ich eine gewisse Vertrautheit mit den Effekten von HRT voraus. Falls Du nicht damit vertraut bist: HRT ist sehr effektiv und macht wahrscheinlich mehr als Du denkst. Es ist eine tolle Erfahrung! \textbf{Dein Geschlecht zu verändern ist richtig cool und macht Spaß. Ich empfehle es.} Du verdienst eine gute medizinische Versorgung deiner Transition und kannst die besten Entscheidungen für Dich treffen. Ich hoffe, dass dieses Dokument eine nützliche Hilfe in deinem Entscheidungsprozess sein kann, und, falls es Dich interessiert, als Startpunkt für weitere Recherche dient.

Und halte Dich fern von den transbezogenen Subreddits. Vertrau mir einfach, okay? Vermeide zumindest /r/mtf, da dieses besonders schlecht ist. Die subs sind keine gesunden, unterstützenden Ressourcen und dort ist nicht viel Weisheit zu erwarten. Du wirst jahrelang verfaulte Gehirnwürmer herausziehen müssen um die Fehlinformationen aus deinem Kopf zu bekommen. Wirklich der beste Rat, den ich dir geben kann! (A.d.Ü. /r/germantrans ist bisher ganz okay, was toxische Einstellungen etc. angeht und für unsere kleinere, zum Teil zersplitterte Community schon wichtig und hilfreich. Trotzdem nur in kleineren Mengen zu genießen!)

Für die Jungs und Männer da draußen: teile dieses Dokuments sind auch für euch sehr relevant, aber es gibt offensichtlich entscheidende Unterschiede in den Zielen und Effekten. \href{https://docs.google.com/document/d/1DXFxzN0XTudPZez\_SO61fpqncRLPH\_Be\_QG\_8Pcz9LU/edit?tab=t.0}{Diese Anleitung für eine maskulinisierende HRT} \textcolor{red}{(Warnung: Google Docs link (A.d.Ü. und auf Englisch))} sieht ziemlich gut aus, ich habe sie aber nicht vollumfänglich durchgearbeitet, setze also auf Deinen Verstand und lass Vorsicht walten. Die sollten sich sowieso ein tboy Katie Tightpussy einfallen lassen. Oliver Longdick oder so. Vielleicht Xavier. 

\textbf{Wenn Du dieses Projekt durch eine Spende unterstützen möchtest,} \href{https://cash.app/Katitties}{CashApp}, \href{https://ko-fi.com/katitties}{Ko-Fi}, und \href{https://account.venmo.com/u/katitties}{Venmo} gehen alle. Vielen Dank!

\subsection*{Wie dieses Dokument benutzt wird}

Das Dokument hat eine lineare Struktur als eine Reihe von Fragen und Antworten, sodass in der Regel jede Frage und Sektion fließend in die nächste übergeht. Ich empfehle, von oben bis unten alles durchzulesen, um hoffentlich wie in einer Unterhaltung alle mögliche Fragen zu beantworten (sogar diejenige von denen Du nicht wusstest, dass Du sie hast), aber das dauert offensichtlich eine Weile. Nimm Dir Zeit und lese dieses Dokument Stück für Stück.

Du kannst das Inhaltsverzeichnis benutzen, um zu einer bestimmten Frage oder Sektion zu navigieren, vor allem beim Wiederlesen oder Nachschlagen. Ich empfehle, diese Seite/dieses Dokument zu speichern, um es immer zur Hand zu haben wenn Du Fragen zu deiner HRT hast. Es ist sehr viel Information auf einmal, es ist okay wenn Du es  Dir nicht alles sofort merkst! Es gibt keine große Eile. 

\noindent\textbf{\href{pghrt.pdf}{Hier kann das Dokument als PDF heruntergeladen werden. Mach das bitte.}}

\noindent\href{pghrtgretchensversion.txt}{Hier kannst Du es alternativ als 90er-Style Textdatei lesen, wenn Dir das Spaß macht.} Diese Version wird nicht aktualisiert. (A.d.Ü. und ist nur auf Englisch verfügbar!)



\section*{WIDMUNG}
\addcontentsline{toc}{section}{WIDMUNG}

Dieses Dokument ist all unseren Schwestern gewidmet, die es nicht geschafft haben. Mögen wir das Licht ihrer Fackel in einen neuen Tag tragen.
 

\section{INTRODUCTION}

\subsection{Ist eine Östrogen HRT sicher?}

Mit modernen, bioidentischen Hormonen ist HRT so sicher wie noch nie. Du ersetzt einfach den Saft, den dein Körper in erster Linie benutzt, durch den anderen, und veränderst die Gewichtung der Hormone, die sich schon in deinem Körper befinden. Obwohl die Details der Optimierung komplex sein können, ist der Prozess ziemlich fehlertolerant. Der Körper ist flexibel und Du wirst es schaffen, es so zu justieren dass es sich gut für Dich anfühlt.

\subsection{Welche Verabreichungsform sollte ich für Östrogen wählen?}

Injektionen. Sie sind insgesamt die effektivste, einfachste, einheitlichste, sicherste und preiswerteste Form von HRT. Für manche werden Injektionen zu einem willkommenen Ritual, oder machen sogar Spaß. (A.d.Ü. Injektionen sind in Deutschland fast immer DIY. Es gibt einige wenige Apotheken die Estradiol Valerat Injektionen anbieten, eine Übersicht findet ihr \href{https://transdb.de/search?type=pharmacy&offers=eInjection}{hier})

\noindent\underline{\textbf{Aber merke: jede Form von Östrogen ist besser als kein Östrogen.}}

\subsection{Warum keine Empfehlung für Pillen, Gels oder Pflaster?}

Weil sie alle gegenüber Injektionen große Nachteile haben. Es ist nicht so, dass sie nichts bringen, aber Du hast es nicht verdient, diese Nachteile aushalten zu müssen. Ich wiederhole: \textbf{alle Formen von HRT können gute Ergebnisse erreichen}, aber das heißt nicht, dass sie alle gleichwertig oder gut sind.

\subsection{Ist die Östrogen-Dosis die gleiche zwischen den verschiedenen Verabreichungsformen?}

Nein. Dies ist wichtig genug, dass ich es nicht in der Sektion \ref{MM} “MYTHEN AND VESCHIEDENES” gepackt habe. \textbf{Die Dosierung von Östrogen kann zwischen verschiedenen Verabreichungsformen nicht direkt verglichen werden.} 1mg von der einen Sorte ist nicht 1mg von der anderen. Unterschiedliche Formen des gleichen Wirkstoffes haben unterschiedliche Eigenschaften, welche die Aufnahme des Östrogens durch den Körper beeinflussen (\textit{“Bioverfügbarkeit”}),sowie die Geschwindigkeit der Aufnahme und die daraus resultierende Halbwertszeit des Wirkstoffes.

\subsection{Was bedeutet "Halbwertszeit"?}

Vereinfacht ausgedrückt, ist die \textit{Halbwertszeit} einer Substanz die Zeit, die vergeht, bis die Hälfte dieser Substanz ausgeschieden wurde. Im Kontext von HRT ist dies die entscheidende Angabe dafür, wie lang eine Dosis in deinem System aktiv bleibt, und somit dafür, wie oft Du eine weitere Dosis verabreichen musst. Das ist dann dein Hormonzyklus, und bildet eine Kurve. Der Östrogenspiegel steigt nach der Einnahme, erreicht einen Höhepunkt, und sinkt dann wieder ab. Die Eigenschaften dieser Kurve (wie sich der Östrogenspiegel über die Zeit verändert) sind wichtig.

\subsection{Was spricht gegen Pillen?}

Das größte Problem mit Pillen sind das erhöhte Risiko für Blutgerinnsel und Leberstörungen bei langfristiger Anwendung. Dieses Risiko kann abgeschwächt werden, indem die Pillen sublingual oder bukkal (also indem die Pille unter der Zunge oder in der Wange aufgelöst wird) statt sie oral einzunehmen (die Pille einfach schlucken), um so den sogenannten First-Pass-Effekt in der Leber zu vermeiden. Es ist jedoch anzunehmen, dass sogar bei sublingualer oder bukkaler Anwendung etwas von der Pille verschluckt wird, sodass die Risiken nicht ganz weg sind. Bitte verstehe, dass das absolute Risiko immer noch sehr niedrig ist (z.B. ist \textit{Paracetamol} um eine Größenordnung gefährlicher für die Leber als Östrogen), jedoch \textbf{wird dieses Risiko weiter in Kombination mit nikotinbedingtem Östrogen-Wechselwirkungen verstärkt.} Siehe Frage \ref{11-2} dazu.

Darüber hinaus sprechen zwei weitere Eigenschaften von Pillen gegen ihre Anwendung: 1) ihre sehr kurze Halbwertszeit und schlechte Bioverfügbarkeit, 2) der in Kombination mit Pillen oft notwendige Einsatz von Antiandrogenen. Die erste Eigenschaft bedeutet, dass Pillen für eine Monotherapie (wird weiter unten genauer erklärt) im Vergleich mit Injektionen meistens ungeeignet sind. Die zweite bedeutet oft eine Menge an unerwünschten Nebeneffekten, je nachdem welches Antiandrogen verwendet wird (siehe Sektion \ref{AA} “ANTIANDROGENE”). Zusammen führen diese Eigenschaften zu mehr Variabilität, die schwierige Abläufe und unerwünschte Nebeneffekte  (wie z. B. niedrige Energie/Libido und langsamere Ergebnisse) wahrscheinlicher machen als andere Verabreichungsformen. Es ist auch schwieriger, mit Pillen einen Vorrat anzulegen, und mancherorts sind sie teurer als Injektionen. Merke auch, dass die Einfuhr von Pillen aus dem Ausland in großen Mengen den Zoll alarmieren kann, was zu finanziellen Verlusten, zum Verlust der Pillen und/oder möglicherweise zu rechtlichen Problemen je nach deinem lokalen Gesetz führen kann. \textbf{Falls irgendjemand fragt, weißt du nicht, wer diese Pillen bestellt hat.}

\textbf{Falls Du aus welchem Grund auch immer Pillen nimmst, nimm bitte 4-8mg sublingual über den Tag verteilt. Unter 4mg ist fast nie eine ausreichende Dosis.

\subsection{Was spricht gegen Pflaster?}

\begin{itemize}
  \item Relativ teuer (meistens sogar teurer als Pillen);
  \item Schwieriger über DIY zu bekommen (nur über den sogenannten "grauen Markt");
  \item Benötigen meistens ein Antiandrogen (siehe Sektion \ref{AA} “ANTIANDROGENE”);
  \item Kann zu Hautirritationen führen;
  \item Müssen 24/7 dran bleiben;
  \item Neigen dazu sich abzulösen;
  \item Die Aufnahme ist nicht immer gleichmäßig (verändert sich z.B. durch Hitze);
  \item Schwierig, einen Vorrat anzulegen (schwer, sie in größeren Mengen zu bekommen);
  \item Schaffen es meistens nicht, Östrogen-Werte über Wechseljahren-Niveau anzuheben, sogar mit mehreren Pflastern auf einmal.
\end{itemize}

\subsection{Was spricht gegen Gel?}

\begin{itemize}
  \item Schwer, es genau zu dosieren, was zu ungleichmäßigen Werten führt;
  \item Es muss regelmäßig aufgetragen werden, da die Halbwertszeit relativ niedrig ist;
  \item Anwendung kann unangenehm sein (schleimig);
  \item Risiko der passiven Exposition von anderen durch Hautkontakt;
  \item Benötigt meistens ein Antiandrogen (siehe Sektion \ref{AA} “ANTIANDROGENE”).
\end{itemize}

Es muss jedoch angemerkt werden, dass Gel mit relativ wenig Aufwand selbst produziert werden kann, was in machen Fällen ein Segen sein kann.

\subsection{Was ist mit Pellets (Implantaten)?}

\begin{itemize}
  \item Meistens viel teurer als andere Optionen;
  \item Sehr selten verfügbar;
  \item Die Dosis kann nur über einen langen Zeitraum justiert werden;
  \item Fehlerhafte Pellets können zu schlechten Werten führen;
  \item Gebrochene/zerquetschte Pellets können zu unerwartet erhöhten Werten führen;
  \item Generell nicht als DIY Option verfügbar.
\end{itemize}

Der letzte Punkt bedeutet, dass es nur über wenige, wahrscheinlich teure Anbieter möglich ist, überhaupt an Pellets zu kommen. Vielleicht hast du hier zum ersten Mal davon gehört. Siehst Du das Problem? (A.d.Ü. Bisher gibt es keine Anbieter*innen für Implantate in der DACH Region.)

\subsection{Was ist mit Sprays?}

Diese sind noch recht experimentell, daher kann nicht so viel darüber gesagt werden, aber sie teilen sich Vor- und Nachteile mit Gel. Sie werden hier erwähnt damit Du weißt, dass es sie gibt.

\subsection{Ist der Unterschied wirklich so relevant?}

\textbf{Ja.}So relevant, dass ich das alles ausgeschrieben habe, damit ich mich weniger wiederholen muss indem ich diesen Punkt einfach verlinke. Ein gut ausgearbeiteter Injektionsrythmus ist zum Erreichen von Östrogen-Monotherapie-Werten der beste Weg.

\subsection{Ist diese Tabelle korrekt?}\label{1-12}

\begin{figure}[H]
	\centering
	\includegraphics[width=1\linewidth]{STUPID_CHART_evil_bad_bad_destroy_evil_bad.png}
	\caption{Diese Tabelle ist scheiße.}
	\label{fig:scebbdeb}
\end{figure}

\textbf{Nein.} Während es nicht die Aufgabe dieses Leitfadens jede Bullshit-Information direkt zu widerlegen die in sozialen Netzwerken als auch ärztlichen Kontexten aufkommt rechtfertigt die weite Verbreitung dieser Tabelle es im speziellen. Die schiere Menge an Schaden, die dieses Bild angerichtet hat rechtfertigt mehr als nur eine Erwähnung in \ref{11-6}.

\textbf{Diese Tabelle ist kategorisch falsch.} Nahezu jeder Aspekt ist in irgendeiner Weise irreführend, mit der einzigen Ausnahme, dass es völlig richtig ist, dass Östrogen keine Stimmveränderungen verursacht. Die Zeitspannen, die für das „erwartete Einsetzen“ angegeben werden, sind schlichtweg, und die Idee einer zeitlichen Begrenzung für die „maximale Wirkung“ ist so irreführend, dass sie fast als Kriminell zu bezeichnen ist. Eine Vielzahl von Veränderungen wird nicht aufgeführt (z. B. psychische Wirkungen) oder ist offensichtlicher Unfug (die Tabelle widerspricht sich selbst in Bezug auf erektile Dysfunktion). Auch auf die Art der Einnahme wird nicht eingegangen.

\textbf{Die wichtigsten Punkte der Grafik, werden in diesem Leitfaden genauer anschauen. Das eine wichtige Punkt den man sich merken sollte ist der Fakt, dass Veränderungen unterschiedlich schnell passieren.} Diese Tabelle weckt falsche Erwartungen, vermittelt ein ungenaues Bild von der Wirkung der HRT und führt Transmenschen in die Irre, indem sie ein falsches Verständnis von Hormonen erzeugt. Bitte ignoriere sie einfach komplett. Nach diesem kleinen Umweg machen wir jetzt weiter im Programm.

\section{WARUM INJEKTIONEN}

\subsection{Was macht Injektionen so gut?}

Beständigkeit. Bei HRT kommt es auf Kontinuität an. Gleichbleibende Hormonenwerte bedeuten Stabilität, und Stabilität ist gut. Sogar die am wenigsten gute Art von Injektionen (kommen wir gleich zu) kann einen einheitlicheren Hormonzyklus als andere Verabreichungsformen ermöglichen.

\subsection{Sind Antiandrogene bei Injektionen notwendig?}

Meistens nicht. Ein richtig dosierter und zeitlich abgestimmter Injektionszyklus, der kontinuierliche, ausreichend hohe Östrogen-Werte erreicht, kann die Testosteron-Produktion auf natürlicher Weise unterdrücken. Somit wird kein Antiandrogen benötigt, was fast immer der richtige Weg ist. Dieses Vorgehen wird \textit{“Monotherapie”} genannt.

\subsection{Wie funktioniert Monotherapie?}\label{2-3}

Ganz vereinfacht gesagt interessiert es das Gehirn nicht, welches Hormon es hat, solange es genug davon hat. Wenn ständig genug Hormone in deinem Körper vorhanden sind, werden keine weitere produziert. Diese Beständigkeit in den Hormonwerten ist das, was Injektionen möglich machen und andere Verabreichungsformen oft nicht schaffen. Zum Beispiel ist Monotherapie mit Pillen in den meisten Fällen so gut wie unmöglich. Genauer gesagt und im Bezug zur Hypothalamus-Hypophysen-Nebennierenrinden-Achse (HHN-Achse) werden durch einen ausreichend hohen Estradiol-Serumspiegel das \textit{luteinisierende Hormon} (LH) und \textit{follikelstimulierende Hormon} (FSH) unterdrückt, was wiederum die Produktion vom Gonadotropin-Releasing-Hormon (GnRH) unterdrückt und damit die Produktion von Testosteron in den Testikeln verhindert.

\subsection{Inwiefern sind Injektionen sicherer?}

Da Antiandrogene meistens nicht gebraucht werden, werden die damit verbundenen Langzeitrisiken vermieden(siehe Sektion \ref{AA} “ANTIANDROGENE”). Bioidentisches Östrogen, das gar nicht durch die Leber verarbeitet wird (siehe Frage \ref{11-1}), kommt einer natürlichen Östrogen-Produktion am nächsten und verringert somit weiter die Risiken.

\subsection{Aber ist das Injizieren nicht an sich schon gefährlich?}

Ja, aber durch eine einfache Anleitung (siehe Sektion \ref{ts} “METHODE UND ZUBEHÖR”) lässt sich die Gefahr bis auf den einen oder anderen blauen Fleck minimieren. Es ist ein Bisschen wie das Fahrrad fahren: sobald Du weißt, wie es geht, müsstest Du dich schon SEHR anstrengen, es richtig falsch zu machen.

\subsection{Inwieweit sind Injektionen leichter?}

Wenn Du erstmal eingepegelt bist, ist alles gut. Injektionen müssen nicht so oft verabreicht werden (so reicht z.B eine wöchtentliche Injektionen im Vergleich zu mehreren Pillen am Tag), haben eine geringe Gefahr, falsch dosiert zu sein (anders als Gele), können nicht mitten im Zyklus von der Haut abfallen (wie bei Pflaster), und man muss dafür nirgendwo hinfahren (wie es bei Pellets der Fall ist).

\subsection{Inwieweit sind Injektionen preiswert?}

In einfachen Worten, es wird weniger Östrogen gebraucht. Ein 10ml Fläschchen (A.d.Ü. oftmals "Vial" genannt), das dich über ein Jahr lang mit Östrogen versorgen kann, beinhaltet nur 400-500mg Estradiol, während das äquivalente Minimum an Pillen (4mg * 365 Tage = 1460 mg) wesentlich mehr benötigt. Dies ist kein rigoroser Vergleich, macht den Unterschied im Maßstab jedoch deutlich. Als weiterer lustiger Vergleich kann man 1 bis 2 Jahre Östrogen-Fläschchen in einer typischen Packung Pillen für drei Monate verstauen. (A.d.Ü. Die Werte hier sind beim Übersetzen auf gängige Größen in Europa angepasst worden.)

\subsection{Aber ich habe keine Krankenversichrung / meine Versicherung zahlt es nicht / durch meine Versicherung sind Pillen billiger als Injektionen / Injektionen gibt es in meinem Land nicht / mein Arzt verschreibt mir keine Injektionen?}

Siehe bitte Sektion \ref{sv} “ÖSTROGEN-BEZUGSQUELLEN”. Du wirst erstaunt sein und, sehr wahrscheinlich, radikalisiert werden.

\subsection{Ist es gut, auch nach Jahren der HRT zu Injektionen zu wechseln?}

\textbf{Ja.} Nichts ist sicher, aber viele Leute erleben wesentliche und deutliche Unterschiede nachdem sie zu Injektionen gewechselt sind, selbst nach Jahren der Hormontherapie. Diese reichen von verbessertem Brustwachstum, verbessertem psychischen Wohlbefinden, weniger Nebeneffekte von Antiandrogene oder anderen Östrogen-Formen, bis hin zu allgemein besserer Stimmung etc. Der Wechsel lohnt sich.

\subsection{Aber Injektionen machen mir Angst.}

Ja, das geht allen am Anfang so. Niemand mag Nadeln, und man hat eine natürliche Angst davor sich selbst zu stechen, aber mit der richtigen Methode und dem richtigen Zubehör merkt man es kaum. Es gibt zahlreiche Menschen, die Anfangs eine intensive Nadelphobie hatten und sich so sehr dran gewöhnt haben, dass sie es jetzt nur noch langweilig finden. Deine Angst ist normal und weit verbreitet, aber sie kann überwunden werden, und es lohnt sich. “Oh, das war ja gar nicht so schlimm” ist ein Satz, den man oft hört. Wie Seneca einst schrieb: „Trau dich, sei mutig! Kein Übel ist so schlimm wie die Angst davor“. Du wirst es schaffen.

\subsection{Sind Injektionen so schlimm wie eine Blutabnahme oder eine Impfung?}

Nein. Zur Blutabnahme werden viel größere Nadeln (auch oft Kanülen genannt) verwendet, die an einem empfindlicheren Ort reingesteckt werden, und dazu wird auch noch Blut abgenommen, was unangenehm ist. Impfungen beinhalten Impfstoffe, die schmerzhafte Immunreaktionen auslösen, weil das die Aufgabe von Impfstoffen ist. HRT-Injektionen bringen ein bisschen Hormone in deine Haut ein, wodurch Du dich wohlfühlst, weil Du Hormone in dir hast. Ich hoffe Du siehst den Unterschied. Es kann auch einfacher sein, sich selbst eine Injektion zu verabreichen, als wenn eine fremde Person dich sticht, je nachdem wie du drauf bist.

\subsection{Gibt es barrierefreies Zubehör für Injektionen?}

Ja. Auto-Injektoren existieren und können zum Beispiel bei Problemen mit Feinmotorik sinnvoll sein. Siehe bitte Frage \ref{5-21}, oder lese einfach weiter.

\subsection{Aber ich \textit{bin} anders und kann nicht injizieren weil ich Glasknochen habe und meine Haut ist aus Papier \textemdash{}!?}

Ich verstehe die Angst, aber wenn Du wirklich unter keiner Bedingung Injektionen machen willst und keine legitime Kontraindikation wie Hämophilie hast, dann mach es halt nicht. Das kannst Du einfach sagen. Es ist okay. Wenn Du deine Meinung änderst wird diese Anleitung immer noch da sein. Und wenn nicht, dann nicht. 

 

\section{ARTEN UND DOSIERUNGEN}\label{td}

\subsection*{Wichtige Begriffe}
\addcontentsline{toc}{subsection}{\textemdash{} Wichtige Begriffe}

\subsection{Was sind die verschiedenen "Arten" von injizierbarem Östrogen?}

Die vier wichtigsten Arten, die für HRT benutzt werden, sind \textit{Estradiol Valerate} (EV), \textit{Estradiol Cipionate} (EC), \textit{Estradiol Enantate} (EEn), und \textit{Estradiol Undecylate} (EUn). Diese "Arten" bezeichnet man als \textit{Estradiol} Ester und sie werden im Körper zu \textit{Estradiol (E2)} umgewandelt. 

Es ist wichtig anzumerken, dass in manchen Regionen Pillen mit dem Namen \textit{Estradiol Valerate} verkauft werden, was verwirrend ist, aber hier geht es nur um die Form für Injektionen.

(A.d.Ü. um das Finden von weiterführenden Informationen zu erleichtern verwenden wir die englische Schreibweise, also Estradiol Enantate anstatt Estradiolenantat. Wir sind uns dieser orthographischen Ungenauigkeit bewusst und bitten darum, diese zu verzeihen.)

\subsection{Was sind die Unterschiede zwischen den verschiedenen injizierbaren Estern?}

Der einzige wichtige Unterschied zwischen Estern ist, dass jeder eine andere Halbwertszeit und eine andere resultierende Hormonspiegelkurve hat, was wiederum die Dosierung und Häufigkeit beeinflusst.

\subsection{Ist ein bestimmter Ester besser als die anderen?}

\textbf{Nein.} Die Unterschiede beziehen sich auf die Dosis und notwendige Häufigkeit der Injektion. Das ist jedoch nur ein qualitativer Unterschied in der Anwendung, und man kann sich da den eigenen Wünschen entsprechend etwas aussuchen. Alle vier Ester haben eine gleichwertige Wirkung und die vorher erwähnten Vorteile von Injektionen gegenüber andere Methoden. 

\subsection{Welche Art von injizierbarem Östrogen sollte ich wählen, wenn ich es mir aussuchen kann?}

Wenn Du die Wahl hast, wird \textit{Estradiol Enantate} von den meisten Menschen bevorzugt, da es sehr stabile Werte ermöglicht. Es muss jedoch angemerkt werden, dass es in den meisten Ländern nur bei DIY zur Verfügung steht (siehe Sektion \ref{sv} “ÖSTROGEN-QUELLEN”). Über die ärztliche Versorgung kann es sein, dass dir\textit{Estradiol Cipionate} angeboten wird (A.d.Ü. in Deutschland soweit mir bekannt nicht), aber meistens in niedrigkonzentrierter Form, was zu hohen Injektionsmengen führen und somit die Vorteile im Vergleich verringert. Die am häufigsten verschriebene Art (insbesondere in den USA (A.d.Ü. und einzige in Deutschland)), \textit{Estradiol Valerate}, kann richtig gute Ergebnisse ermöglichen, hat aber ihre Macken (insbesondere bei DIY), weshalb sie nicht die erste Wahl sein sollte. Lies weiter.

\subsection{Was bedeutet “Konzentration”?}

Östrogen-Fläschchen beinhalten Östrogen, das in Öl aufgelöst ist. Die \textit{Konzentration} eines Fläschchen ist wie viel Östrogen in dieser Öllösung ist. Dies wird als ein Verhältnis von Gewicht und Volumen für das Fläschchen ausgewiesen. Anders gesagt: Für jedes Milliliter Öl (Volumen), gibt es so und so viel Milligramm Östrogen (Gewicht). Oft wird die Konzentration für das Gesamtvolumen des Fläschchens angegeben (z.B. 200mg / 5ml), es ist aber fast immer besser, dieses Verhältnis vereinfacht auszudrücken (also in diesem Fall 40 mg/ml). \textbf{Typische Konzentrationen sind 5 mg/ml, 10 mg/ml, 20 mg/ml, 40 mg/ml, und manchmal 50 mg/ml.}

\subsection{Was ist mit "Dosis und Häufigkeit der Anwendung" gemeint?}

\textit{Dosierung} und \textit{Häufigkeit} sind die zwei Faktoren, die deinen Hormonzyklus bestimmen. Die \textit{Dosierung} ist die Menge an Östrogen, die Du zu dir nimmst (gemessen in mg), und die \textit{Häufigkeit} ist wie oft du es zu dir nimmst (gemessen in Tagen oder Wochen). Manchmal wird für die Kombination aus beiden Faktoren das Wort "Regimen" benutzt, das dann beschreibt, wie oft und wieviel HRT-bezogenen Sachen Du zu dir nimmst. (A.d.Ü. 
 Das ist am ehesten als "Therapieplan" zu übersetzen - was aber selten innerhalb der Community als Vokabel verwendet wird.)

\subsection{Wie finde ich meine Dosierung raus?}

Deine Dosierung ist die Konzentration deines Fläschchen mal das Volumen, das du injizierst. \[Konzentration (mg/ml) * Volumen (ml) = Dosierung (mg)\] \textbf{Bitte verstehe, dass das Volumen allein keine Aussage zur Dosierung macht.} Es ist wie beim Backen: Du kannst nicht einfach sagen "45 Minuten im Ofen backen", ohne auch die Temperatur zu nennen.

\subsection{Gibt es Beispiele für Dosierungsberechnungen?}

Die Berechnung ist ganz einfach, versprochen! Weiter unten ist eine kleine Referenztabelle die Konzentration und Volumen für einige gängige Dosierungen vergleicht. Es wird nur auf zwei Kommastellen aufgerundet. Du wirst keine Spritzen verwenden, die genau 0.153ml messen könnten. Das liegt innerhalb der Rundungsfehler-Toleranz und macht in unserem Fall keinen relevanten Unterschied.

\begin{table}[]
\centering
\caption{Beispiele für Dosierungen von üblichen Konzentrationen, bei Volumen}
\label{tab:Konzentrationen}
\begin{tabular}{@{}lllll@{}}
    \toprule
    \multicolumn{1}{c}{} & \multicolumn{4}{c}{Konzentration (mg/ml)} \\
    \cmidrule(rl){2-5}
            & 5    & 10  & 20 & 40    \\
            \cmidrule(rl){2-5}
Dosierung (mg) & \multicolumn{4}{c}{Volumen (mL)}  \\
    \cmidrule(r){1-1} \cmidrule(lr){2-5} 
4        & 0.8  & 0.4 & 0.2  & 0.1      \\
5        & 1    & 0.5 & 0.25 & 0.13   \\
6        & 1.2  & 0.6   & 0.3  & 0.15     \\
7        & 1.4  & 0.7 & 0.35  & 0.18  \\
8        & 1.6  & 0.8   & 0.4  & 0.2    \\
9        & 1.8  & 0.9 & 0.45  & 0.23 \\
10       & 2    & 1   & 0.5  & 0.25   \\
    \bottomrule
\end{tabular}
\end{table}

\textbf{Wie diese Tabelle gelesen wird:} Links findest du deine gewünschte Dosierung, rechts das entsprechende Volumen und in der oberen Zeile die jeweilige Konzentration. Du wirst merken, dass das benötigte Volumen für eine vernünftige Dosierung bei einer Konzentration von 5 mg/ml schwierig ist. Das liegt daran, dass Fläschchen mit einer Konzentration von 5 mg/ml nicht sonderlich praktisch sind.

\subsection{Wie kann ich Dosierungen zwischen den verschiedenen Estern konvertieren?}

\textbf{Kannst du nicht.} Da die Ester sich jeweils anders verhalten, gibt es keine "Konvertierung" im direkten Sinne. Falls Du zu einem anderen Ester wechselst, solltest Du einfach eine typische Dosierung für dieses Ester nutzen und schauen, wie es sich bei dir verhält. Es ist möglich, Ester zu vergleichen, aber es gibt keine Methode für eine direkte Konvertierung.

\subsection{Wie kann ich Hormonwertkurven und Dosierungen zwischen Estern vergleichen?}

Falls Du etwas nerdy bist, kann ich \href{http://estrannai.se}{estrannai.se} sehr empfehlen. Sowas zu benutzen ist nicht verpflichtend, aber es ist ein gutes Werkzeug für grobe Vergleiche. \href{https://estrannai.se/\#i0__cu,7,7,1-cu,5,7,3-cu,5,7,2}{Hier ist ein beispielhefter Vergleich zwischen typischen wöchentlichen Dosierungen} die wir uns jetzt einzeln anschauen werden.

\textbf{Beachte, dass die unten aufgeführten Dosierungen im unteren Bereich meist ausreichend sind.} Fang mit einer niedrigen Menge an und nimm nur mehr, wenn du es brauchst (A.d.Ü. um auf die Zielwerte zu kommen). Mehr ist nicht zwangsläufig besser, aber darüber reden wir später. Die Herkunft deines Fläschchens hat wahrscheinlich keinen Einfluss auf diese Dosierungsempfehlungen.

\subsection*{Lerne die Ester kennen!}
\addcontentsline{toc}{subsection}{\textemdash{} Lerne die Ester kennen!}

\subsection{Wie dosiere ich \textit{Estradiol Valerate}?}

Für gute Werte mit \textit{Estradiol Valerate} empfehle ich entweder eine niedrige Dosis zwei mal die Woche, oder eine höhere Dosis einmal die Woche. Es ist eine Frage der Bequemlichkeit und Verträglichkeit. Die typische Faustregel ist etwa 1 mg pro Tag in einem Zyklus von 3 bis 7 Tagen. \textbf{Ich empfehle eine wöchentliche Dosis von 6-8mg}, aber 4-5mg in  5 Tagen ist auch üblich. \textbf{Es sollte immer mindestens wöchentlich injiziert werden (also nie länger als 7 Tagen zwischen Injektionen).} Ein wöchentlicher Rythmus ist schon an der Grenze dessen, was das Ester leisten kann. Mehr Zeit zwischen Injektionen wird ausdrücklich nicht empfohlen, aufgrund der Varianz-bedingten Nebenwirkungen (Siehe Frage \ref{7-3}).

Bemerkenswerterweise werden in manchen Regionen Pillen mit dem Namen \textit{Estradiol Valerate} verkauft, was verwirrend ist, aber hier geht es nur um die Form für Injektionen.

\subsection{Wie sieht die Hormonwertkurven für \textit{Estradiol Valerate} aus?}

\textit{Estradiol Valerate} ist ziemlich vertrackt. Es geht schnell hoch, mit einem Gipfelwert wenige Tage nach der Injektion, geht aber genauso schnell wieder runter. Diese relative Instabilität kann anstrengend sein, je nach deiner Empfindung, lässt sich aber durch die richtige Einstellung von Dosierung und Frequenz teilweise mildern.

 \begin{figure}[H]
     \centering
     \includegraphics[width=1\linewidth]{ev.png}
     \caption{Estradiol-Serumspiegel (pg / ml) von Estradiolvalerat vs Zeit (Tagen) }
     \label{fig:ev}
 \end{figure}

\subsection{Wie dosiere ich \textit{Estradiol Cypionate}?}

\textit{Estradiolcypionat} kann problemlos wöchentlich angewendet werden. \textbf{Eine wöchentliche Dosierung von 5-7mg ist typisch.} Eine niedrigere Frequenz (z.B. alle 10 Tage) wird nicht empfohlen, da sie weniger Effizient ist und für gute Werte höhere Dosierungen verlangt. Jede Verlängerung über 7 Tage hinaus erhöht die Gefahr von Varianz-bedingten Nebenwirkungen (Siehe Frage \ref{7-3}).

\subsection{Wie sieht die Hormonwertkurve für \textit{Estradiol Cypionate} aus?}

\textit{Estradiol Cypionate} ist weniger extrem als \textit{Estradiolvalerat}. Die Kurve geht nicht so schnell hoch und runter, es kommt aber über eine Woche hinweg immer noch einer gewissen Schwankung.

 \begin{figure}[H]
     \centering
     \includegraphics[width=1\linewidth]{ec.png}
     \caption{Estradiol-Serumspiegel (pg / ml) von Estradiolcipionat vs Zeit (Tage) }
     \label{fig:ec}
 \end{figure}

\subsection{Wie dosiere ich \textit{Estradiol Enanthate}?}

\textit{Estradiol Enanthate} kann leicht wöchentlich eingesetzt werden und kann (notfalls) auch eine Frequenz von 10 Tagen ermöglichen, wenn Bedarf besteht. Ein noch längerer Abstand zwischen den Injektionen ist theoretisch möglich, jedoch aufgrund der entstehenden Varianz der Werte nicht zu empfehlen. \textbf{Eine wöchentliche Dosiserung von 4-6mg ist typisch}, bei 10 Tagen wird 5-7mg empfohlen. Eine Verlängerung auf 10 Tage scheint jedoch keine wesentliche Vorteile zu bieten, sodass die wochentliche Anwendung empfohlen wird.

\subsection{Wie sieht die Hormonwertkurve für \textit{Estradiol Enanthate} aus?}

\textit{Estradiol Enanthate} ist bei injizierbarem Östrogen der Goldstandard. Die Kurve ist über die Dauer der typischen wöchentlichen Anwendung sehr flach (wenig Varianz). Dadurch werden sehr stabile Werte ermöglicht, was die Gefahr von Varianz-bedingten Nebeneffekten verringert (Siehe Frage \ref{7-3}).

 \begin{figure}[H]
     \centering
     \includegraphics[width=1\linewidth]{een.png}
     \caption{Estradiol-Serumspiegel (pg / ml) von Estradiolenanthat vs Zeit (Tage) }
     \label{fig:een}
 \end{figure}

\subsection{Wie dosiere ich \textit{Estradiol Undecylate}?}

\textit{Estradiol Undecylate} kann die Frequenz weit über das wöchentliche bis ins monatliche strecken. Die dafür empfohlene Dosierung ist jedoch nicht standardisiert oder erforscht. Anders als bei den anderen Estern sind die Faktoren, die die Aufnahme des Östrogens bedingen (\textit{“Pharmakokinetik”}) bei \textit{Estradiol Undecylate} sehr wichtig. Dementsprechend ist die Anwendung noch als experimentell zu verstehen und wird in dieser Anleitung nicht besprochen. Frag am besten mal im Esoterikladen nach dem Mondkalender, um dich bei jedem Vollmond zu injizieren.

\subsection{Wie sieht die Hormonwertkurve für \textit{Estradiol Undecylate} aus?}

Wissen wir nicht wirklich. Es gibt zu wenige Daten und zu viele Varianten, um ein genaues Bild davon zu haben. Wenn es dich interessiert kannst du gern selbst (und an dir selbst) experimentieren, sei dir jedoch der Risiken bewusst. Ich empfehle es nicht, wenn Du nicht ohnehin weißt, was du tust.

 \begin{figure}[H]
     \centering
     \includegraphics[width=1\linewidth]{moon.png}
     \caption{Der Mond}
     \label{fig:moon}
 \end{figure}

 

\section{BLUTTESTS UND WERTE}

\subsection*{Werte bekommen}
\addcontentsline{toc}{subsection}{\textemdash{} Werte bekommen}

\subsection{Wie oft sollte ich meine Werte testen?}

In der Einstellphase willst Du relativ oft testen. Nach jeder Veränderung in der Anwendung solltest Du ein bis zwei Monate warten, damit deine Werte sich stabilisieren, und dann testen.

\subsection{Muss ich meine Werte vor Beginn der HRT testen?}

Eigentlich nicht, da das Testosteron zu hoch und das Östrogen zu niedrig sein werden, was erwartbar ist, aber ein allgemeiner Blutwerte-Check-Up (Leber, Lipide etc.) kann deiner Gesundheit nicht schaden. Bei Verdacht einer Intergeschlechtigkeit, welche die HRT beeinflussen könnte, kann ein vorbereitender Bluttest empfehlenswert sein, um dies auszuschließen.

\subsection{Muss ich meine Werte testen, wenn ich über längerer Zeit nicht verändert habe?}

Eingentlich nicht, da sich von selbst eben nichts verändert haben sollte. Es kann aber Sicherheit geben, wenn andere Aspekte deiner Routine sich verändert haben. Falls Du mit \textit{Estradiol Undecylate} experimentierst solltest Du mindestens vierteljährlich testen.

\subsection{Ich bin nicht versichert oder habe keinen Arzt, wie kriege ich ein Bluttest?}

Suche nach privaten Labors in deiner Region, je nachdem, ob es legal ist. In vielen Regionen können private Bluttests gekauft werden, diese sind aber oft nicht billig. Online-Angebote sind manchmal billiger, das ist aber regional unterschiedlich (A.d.Ü.: Im deutschsprachigen Raum gibt es viele verschiedene Anbieter für private Bluttests. Google einfach, was es in deiner Region/dein Preissegment gibt.)

\subsection{Ich bekomme keinen Bluttest/kann mir keinen leisten. Kann ich trotzdem HRT machen?}

Es ist natürlich besser, Information über die Werte bekommen zu können, aber an sich ist HRT sehr sicher und bei typischen Dosierungen unproblematisch. Du musst dich halt mehr auf das verlassen, was du fühlst und an dich beobachtest.

\subsection{Welche Werte soll ich testen lassen?}

Mindestens \textit{Estradiol} (E2) und \textit{Testosteron gesamt} (T) da diese Werte uns am meisten interessieren. \textit{Sexualhormon-bindendes-Globulin} (SHBG), \textit{Dihydrotestosteron} (DHT), \textit{Estrone} (E1), und \textit{Prolaktin }(PRL) zu testen kann auch sinnvoll sein, wenn Du Schwierigkeiten hast, um bei der Aufklärung zu helfen. \textit{Follikel stimulierendes Hormon} (FSH) und \textit{luteinisierendes Hormon} (LH) können dir sagen, ob deine Hypothalamus-Hypophysen-Nebennierenrinden-Achse deaktiviert ist, was die Basis für Monotherapie bildet (Siehe Frage \ref{2-3}). Aber ich wiederhole: \textbf{\textit{Estradiol} (E2) und \textit{Testosteron gesamt} sind das Wichtigste. }

\subsection{Zu welchem Zeitpunkt in meinem Hormonzyklus sollte ich testen?}

Am Ende des Zyklus (\textit{ dem “Talwert”}). Du willst so nah wie möglich am tiefsten Wert messen, da dieser die beste Information bietet. Man könnte sogar meinen, dass sie die einzig relevante Information ist, da gleichmäßige Talwerte hier das Wichtigste sind. Zum Beipiel: Wenn Du immer Donnerstag Nachmittag injizierst, solltest dein Bluttest am Vormittag oder frühen Nachmittag des nächsten Donnerstags machen.

\subsection{Mein Artzt sagt, ich soll den Höchstwert / den Mittelwert testen, soll ich?}

\textbf{Nein.} Der Höchstwert sagt gar nichts aus und zeigt nur, welchen Ester Du benutzt. Wohlwollend interpretiert deutet solch eine Vorgehensweise auf Inkompetenz, die auf veraltete, konservative Versorgungsstandards beruht. Etwas weniger wohlwollend interpretiert ist es böswillig, für Östrogenwerte zu sorgen, die zu schlechten Ergebnissen oder gar gesundheitlichen Schäden führen können. \textbf{Ich empfehle, trotzdem den Talwert zu messen.}

\subsection*{Bluttest auswerten}
\addcontentsline{toc}{subsection}{\textemdash{} Bluttest auswerten}

\subsection{Welche Östrogenwerte will ich erreichen?}

In der Transition ist dies vielleicht die umstrittenste Frage. Die einfache Antwort ist: Hoch genug, dass Du dich wohl fühlst und dein Testosteron unterdrückt wird, soweit Du es brauchst. Höhere Werte sind darüber hinaus im besten Fall verschwenderisch, im schlimmsten kontraproduktiv. Das ist jedoch ein breites Spektrum und durch die vielen Variablen gibt es viele individuelle Schwankungen. Anders gesagt: Du willst genug Östrogen, so dass Du dich gut fühlst.

\subsection{Führen höhere Östrogen-Werte zu schnelleren oder besseren Ergebnissen?}

\textbf{Nein.} Östrogen-Werte, die über das Notwendige hinausgehen, werden von manchen aus subjektiven Gründen bevorzugt, aber sie verbessern die Feminisierung nicht. tatsächlich können zu hohe Werte zu unerwünschten Nebeneffekten führen, wie zum Beispiel Stimmungsschwankungen. \textbf{Für die Feminisierung ist es viel wichtiger, das Testosteron zu einem ausreichend niedrigen Wert zu drücken.}

\subsection{Okay, aber welche Zahl will ich bei Östrogen auf meinem Befund sehen?}

Mit dem Verständnis, dass die genaue Zahl keine Rolle spielt, dass die Zahl aufgrund der Latenz immer etwas höher sein wird als das, was beim Talwert gemessen wird, und dass die Zahl in einer Wolke von Möglichkeiten liegt, die auf einer Reihe von Faktoren basieren, \textbf{empfehle ich einen Talwert von mindestens ca. 200 pg/ml (730 pmol/L).} Diese Empfehlung ist etwas konservativ, da die Unterdrückung der HHN-Achse schon früher erfolgt, aber sie bietet etwas Spielraum. Für die meisten funktioniert es in diesem Bereich ganz gut, aber manche mögen es etwas tiefer oder höher. Ich denke nicht, dass diese Zahl voll im Fokus stehen sollte, da sie immer variabel ist und das wichtigste dein Gefühl ist, \textbf{aber} Werte von \textbf{über 300pg/ml (1100 pmol/L) im Talwert ist sehr wahrscheinlich höher als es sein muss oder sollte.} Es gibt Ausnahmen, aber Du bist wahrscheinlich nicht die Ausnahme. Mach aber einfach, was sich für Dich gut anfühlt. Darüber hinaus siehe Frage \ref{11-1}.

\subsection{Was für einen Testosteron-Wert will ich?}

Die Unterdrückung des Testosterons ist die Bedingung für die Feminisierung, also reicht meistens weniger als 50 ng/dL (1.7 nmol/L). \textbf{Wohlbemerkt ist ein Testosteronwert nahe Null nicht erwünscht.} Siehe Sektion \ref{T} “TESTOSTERON”.

\subsection{Ich habe von Natur aus hohes/niedriges T. Muss ich meine Dosierung anpassen?}

Wahrscheinlich nicht. Die Testosteronwerte, die typischerweise vor der HRT bestehen, sind meistens für die Feminisierung höher als erwünscht (Siehe Frage \ref{2-3}). Die Ausnahme wäre irgendeine Form von  Intergeschlechtigkeit, welche ein Grund für eine genauere Justierung der Dosierung sein könnte. Dies führt jedoch über die Grenzen dieser Anleitung hinaus. Du musst wahrscheinlich die Empfehlungen nicht anpassen, aber vielleicht fühlst Du dich besser, wenn Du es tust. Letztendlich sollst Du das machen, was sich gut anfühlt. Siehe Frage \ref{9-2}.

\subsection{Ich hatte eine Geschlechtsangleichende Bottom Operation (GaOP). Sollen meine Östrogen-Werte anders sein?}

Da die Unterdrückung von Testosteron für dich kein Problem mehr ist, könntest Du dich wahrscheinlich mit niedrigeren Östrogen-Werten gut fühlen, \textbf{aber Du brauchst immer noch Östrogen.} Da Du jetzt keine eigene Hormone mehr produzierst ist es extrem wichtig, dass Du deinen Körper mit ausreichend Hormonen versorgst. Keine oder zu wenige Hormone führen zu Wechseljahresbeschwerden, wie sie ältere cis Frauen erleben können. Passe es so an, wie es dir passt.

Genauer gesagt: \textbf{Ein Minimum von ca 100 pg/ml (350 pmol/L) ist notwendig, um Probleme mit der Knochenmineraldichte zu vermeiden.} Wenn deine Feminisierung großenteils schon erfolgt ist, ist dein Hormonprofil in vieler Hinsicht mit dem einer menopausalen cis Frau vergleichbar und relevante Studien können hilfreich sein. (Siehe Frage \ref{11-29}). Ein Supplementierung von Testosteron kann in manchen Fällen von Antriebslosigkeit oder Müdigkeit sinnvoll sein (Siehe Frage \ref{9-2}).

\subsection{Kann ein Bluttest irgendwie ungenau sein?}

Je nachdem, welche Methode beim Bluttest angewandt wird, können Nahrungsergänzungsmittel mit Biotin den \textit{Estradiol} (E2) Wert (und andere, aber uns geht es \textit{Estradiol }) unerwartet hoch erscheinen lassen. Es ist nicht immer möglich zu wissen, welche Methode angewandt wird, so ist es einfacher, für einige Tage vor dem Test kein Biotin zu nehmen. Es ist auch immer möglich, dass bei der Blutprobe oder bei der Ausrüstung für den Test ein Fehler vorkommt, das ist aber sehr unwahrscheinlich. (A.d.Ü. Da viele in Deutschland ihr Gel einsetzen könnte das häufiger vorkommen - stell sicher, dass deine Blutabnahmestelle nicht aus versehen mit Gel verunreinigt wird.)

\subsection{Sind verschiedene Ester oder Verabreichungsformen in Bluttests erkennbar?}\label{4-16}

Nein. Durch ein Bluttest kann nicht erkannt werden, welche Östrogen-Form eine Person einnimmt. Die verschiedenen injizierbaren Ester werden alle im Körper zu \textit{Estradiol } umgewandelt, genau wie wir es wollen, und so ist es auch mit Pillen, Pflastern, Gels, Sprays oder was auch immer. Am Ende des Tages ist es alles Östrogen.

 

\section{METHODE UND ZUBEHÖR} \label{ts}

\subsection*{Injektionsstellen \& Sicherheit}
\addcontentsline{toc}{subsection}{\textemdash{} Injektionsstellen \& Sicherheit}

\subsection{Wie führe ich eine Injektion sicher durch?}

Ich empfehle diese beiden Videos:

\begin{enumerate}
  \item \href{https://www.youtube.com/watch?v=cBabaGC2Dok}{\textit{“How to perform an intramuscular (IM) self-injection”}}
  \item \href{https://www.youtube.com/watch?v=YfNlAZLxLyw}{\textit{“Painless (for me so far) IM Injection Technique”}}
\end{enumerate}
(A.d.Ü.: Die deutschsprachigen Videos, die ich zum Thema finden konnte, sind alle recht lang und für Pflegepersonal gedacht und somit für unsere Zwecke mit unnützen Infos versehen. Im wesentlichen können die hier verlinkten Videos auch ohne Englischkenntnisse verfolgt werden und die wichtigen Punkte werden weiter unten aufgeführt. Du kannst dir die deutschen Videos auf YouTube natürlich gern anschauen) 

Mithilfe dieser zwei Videos solltest Du ausreichend vorbereitet sein, um eine schmerzlose Injektion bei dir vorzunehmen. Ich empfehle dir, sie dir aufmerksam anzuschauen und bei Bedarf zu wiederholen. \textbf{Um Schmerzen zu verhindern ist ein Punkt besonders wichtig: Den abgeschrägten Teil der Nadelspitze (die Fase) beim reinstechen nach oben zu halten).} Anders gesagt: Die Nadel hat eine klar definierte Spitze, und diese soll deine Haut als erste berühren. Du willst schön gerade einstechen. Du kannst dir gern vorstellen, wie deine Hand bzw. dein Handgelenk sich dabei bewegen sollen, wenn es dir hilft dir die Bewegung vorzustellen, aber am Ende ist es Übungssache, bis es intuitiv wird.

\textbf{Merke: Injizieren ist eine besondere Fertigkeit, die gelernt werden will!} Du wirst mit der Zeit besser, und es braucht wirklich nicht lange. Kriegste schon hin.

\subsection{Muss ich genau so injizieren?}

Nun, Variationen sind erlaubt. Da es am Ende bloß darum geht, dich zu pieksen, gibt es viele Wege, wie man es machen kann. Finde den Weg, der für dich am besten ist. Ein schnelles, scharfes Einstechen funktioniert meist am besten, aber wenn Du langsamer gehen willst ist das auch okay, solange Du es immer so machst und dich so mit der Zeit verbesserst.

\subsection{Wie kann ich die Angst vor dem Injizieren überwinden?}

Ich empfehle daraus ein Ritual zu machen. Wenn Du dir eine Routine aufbaust, wird es irgendwann ganz natürlich. Wenn Du dich mit Musik, ein Gespräch, Fernsehen oder was immer für dich gut ist ablenken kannst, damit dein Muskelgedächtnis übernimmt, toll! Finde heraus, was zu dir passt. Es kann helfen, wenn eine befreundete Person oder ein Partner die ersten Injektionen übernimmt. Für die meisten ist die erste Injektion die aufregendste. Die häufigste Reaktion ist ein "Oh, das war's?", weil es nie so schlimm ist, wie man es erwartet.

\subsection{Ist es wichtig, wo ich am Körper injiziere?}

Ja und nein. Es ist wichtig, in sicheren Körperbereichen zu injizieren, aber sonst hängt es davon ab, wie gelenkig du bist, was für ein Volumen du injizierst, was für eine Spritze du benutzst, und dein eigenes Gefühl. Es ist aber wichtig, \textbf{Injektionsstellen abzuwechseln.} Zum Beispiel kannst du jede Woche auf eine andere Körperseite injizieren, mal ins linke Bein, mal ins rechte Bein. So wird das Risiko einer langfristigen Narbenbildung verringert.

\subsection{Welche Injektionsstellen sind sicher?}

Darüber streiten sich die Gelehrten, aber es kommt vor allem auf deinen Körperbau an. Ich empfehle die Beine, wie in den Video(s) zu sehen, da es für die meisten gut erreichbar ist und gut einheitlich sein kann, wenn Du eingeübt bist, aber machen bevorzugen die Pobacke oder den Bauch. \href{https://vertisis.com/articles/how-to-self-administer-a-subcutaneous-injection}{Dieses Video auf dieser Webseite} zeigt andere Injektionsstellen, die je nach deinem Ausstattung in Frage kommen können. (A.d.Ü.: Eine entsprechende Deutsche Version mit einer detaillierten Anleitung als PDF findest Du \href{https://www.bk-trier.de/media-bkt/docs/PIZ_HZ_Subkutane_Injektion_2021.pdf}{hier}.) Finde raus, was für dich am besten ist.

\subsection{Was bedeuten "intramuskulär" (IM) und “subkutan” (SubQ/SC)?}

Im Kontext von Injektionen werden diese Begriffe oft verwendet. \textit{Intramuskulär} bedeutet, dass in das Muskelgewebe injiziert wird, \textit{subkutan} dass in das Unterhautfettgewebe injiziert wird.

\subsection{Was ist der Unterschied zwischen intramuskulären (IM) und subkutanen (SubQ/SC) Injektionen?}

\textbf{Im Kontext von HRT gibt es keine wesentliche Unterschiede zwischen subkutanen und intramuskulären Injektionen.} Subkutane Injektionen werden zwar etwas langsamer aufgenommen als intramuskuläre, der Unterschied ist aber generell nicht groß genug um für die Dosierung relevant zu sein. Darüber hinaus wird bei jeder Injektion nicht immer eindeutig nur das Unterhautfettgewebe oder nur das Muskelgewebe getroffen, sodass der Unterschied in der Praxis weiter verschwimmt.

Kleine Randbemerkung: Pharmazeutische Hersteller von Estradiol-Fläschchen geben in der Regel an, dass diese nur für intramuskuläre Injektionen bestimmt sind. Das liegt daran, dass diese nur für diesen Zweck offiziell zugelassen sind. Das ist aber egal.

\subsection{Sollte ich intramuskuläre (IM) oder subkutane Injektionen (SubQ/SC) durchführen?}

\textbf{Das ist die falsche Frage.} \textbf{Eine Injektion ist eine Injektion.} Subkutane Injektionen werden oft empfohlen, da viele davon aus ausgehen, dass sie weniger schmerzhaft sind, es gibt aber keinen wesentlichen Unterschied in der Durchführung. \textbf{Die Vorteile, von denen ausgegangen wird, haben weniger mit der Art der Injektion zu tun als mit Faktoren, die den möglichen Injektionsschmerz beeinflussen.} Die bessere Frage wäre "wie kann ich Injektionsschmerzen minimieren?", aber zuerst zwei weitere Fragen.

\subsection{Spielt mein Injektionswinkel und/oder meine bevorzugte Injektionsmethode eine Rolle?}

Nein. Ich wiederhole: Der wichtigste Aspekt einer Injektion ist, dass Du eine Nadel in deinem Körper reinsteckst und eine Flüssigkeit reinspritzt. Solange die Flüssigkeit nicht wieder rauskommt (oder zumindest nicht viel davon) und es nicht weh tut (oder zumindest nicht doll) hast Du einen fantastischen Job gemacht. \textbf{Ich kann nicht genug betonen, dass die "Wahl" zwischen subkutaner und intramuskulärer nicht relevant ist und für die Wirksamkeit von injizierbaren Östrogen keine Rolle spielt.} Einzig \textit{Estradiol Undecylate} ist der Fall, in dem die Art der Injektion möglicherweise einen Unterschied macht, aber die Details sind noch unklar. Der Punkt ist: Bitte mach dir Gedanken über die Sachen, die wichtig sind, und nicht über die Sachen, die unwichtig sind.

\subsection{Muss ich beim Injizieren auf die Aspiration achten?}

Nein. “Aspiration” meint das kurzzeitige Zurückziehen des Spritzenstempels beim Injizieren um festzustellen, ob versehentlich ein Gefäß getroffen wurde. Über die Notwendigkeit wird gestritten, aber im Kontext von Hormoninjektionen ist die Gefahr ein Blutgefäß innerhalb der empfohlenen Injektionsareale mit kurzen Nadelspitzen zu treffen ohnehin verschwindend gering, und gibt es keine wesentliche Vorteile. Bei der Injektion in ein Gewebe wird die Aspiration von den meisten medizinischen Fachstellen auch nicht mehr empfohlen.

\subsection{Wie kann ich den Schmerz beim Injizieren minimieren?}

Du kannst üben und deine Technik verbessern, aber darüber hinaus ist deine Spritze- und Nadelkombination der wichtigste Faktor. \textbf{Um Beschwerden zu minimieren, sollte die höchstmögliche Nadelstärke ("Gauge", siehe nächste Frage) verwendet werden, die zum Trägeröl in deinem Fläschchen passt, zusammen mit einer passenden Spritze und Nadellänge. }Die richtige Frage ist "Welche Nadelstärke und -länge brauche ich?". Um das herauszufinden, lass uns darüber reden wie Nadeln bzw. Kanülen funktionieren.

\subsection*{Nadelkunde}
\addcontentsline{toc}{subsection}{\textemdash{} Nadelkunde}

\subsection{Was bedeutet "Gauge" bei Nadelgrößen?}

\textit{Gauge }ist die Einheit für die Dicke der Nadel/Kanüle (A.d.Ü.: in diesem Abschnitt wird öfters der Begriff "Kanüle" verwendet, da dieser genauer ist. Nadel interessieren uns hier eigentlich nicht, da sie streng gesehen zum Nähen, Stricken oder zur Akupunktur benutzt werden. "Nadel" wird jedoch im Sinne einer einfachen Sprache auch weiter verwendet). Je höher die Zahl, desto dünner die Nadel. Eine 25G-Kanüle ist zum Beispiel dünner als eine 20G-Kanüle. Dünnere Kanülen sind meist kürzer, da sie sich leichter verbiegen können. Es ist nicht überraschend, dass dünnere Nadeln weniger weh tun. Die Stärke der Kanüle hat wohl bemerkt keinen Einfluss auf die HRT selbst, es geht nur darum, wie angenehm (oder unangenehm) die Injektion wird.

\subsection{Was sind “Luer lock” und “Insulin” Spritzen/Nadel?}\label{5-13}

\textit{Luer lock-Spritzen} ermöglichen es, die Kanüle/Nadel zu wechseln, damit jeweils eine passende fürs Aufziehen der Lösung und für die Injektion verwendet werden kann. \textit{Insulinspritzen} habe eine feste Nadel, sodass sie sowohl beim Aufziehen als auch beim Injizieren verwendet wird. Wenn möglich werden Insulinspritzen bevorzugt, da sie bequemer sind und einen sehr geringen Hohlraum/Deadspace haben (Siehe Frage \ref{5-26}) "Luer slip" Nadel werden nicht empfohlen, da sie fehleranfällig sind.

\textbf{Sicherheitshinweis: Das Wiederaufsetzen der Schutzkappe auf Nadeln wird generell nicht empfohlen, da die Gefahr besteht, sich selbst zu stechen. Solltest Du es dennoch tun (z. B. beim Auswechseln einer Ziehnadel), drücke NIE mit deiner Hand in Richtung der Nadel.}

Die Kappe kann brechen und Du könntest dich verletzen, wenn sie nicht richtig aufgesetzt wird. Es wird empfohlen, die Kappe sanft auf einer ebenen Fläche mit der Nadel "aufzufangen", um sie dann gegen eine Wand oder mit den Fingern an den Seiten zuzudrücken. Bei Injektionen am eigenen Körper gibt es keine Gefahr einer Krankheitsübertragung, sodass diese Warnung in diesem Fall nicht so streng genommen werden muss, aber bei Injektionen an anderen Personen ist dies SEHR wichtig. Zur Entsorgung der Spritzen siehe Frage \ref{5-27}.

\subsection{Welche Nadel-/Kanülenstärke sollte ich beim Aufziehen verwenden?}

Bei Luer-lock-Spritzen empfiehlt sich, eine niedrigere Gauge (= eine stärkere Nadel) beim Aufziehen zu verwenden als beim Injizieren. Allerdings kann eine zu niedrige Gauge zum Ausstanzen/Beschädigung deines Stopfens (A.d.Ü. Im Deutschen haben wir dafür auch "coring" übernommen - entsprechend verwenden wir das Lehnwort hier weiterhin) führen (Siehe Frage \ref{5-23}), sodass mindestens 21-23G empfohlen wird. Wenn Du geduldig bist und keine großes Volumen injizieren muss wird hohe Gaugen empfohlen, um die Gefahr eines corings zu reduzieren. Die Nadel wird durch das Einstechen in den Gummistopper nicht stumpf. Diese Frage ist bei Insulinspritzen nicht relevant, da die Nadel in diesem Fall nicht austauschbar ist.

\subsection{Welche Nadel-/Kanülenlänge sollte ich beim Aufziehen verwenden?}

Bei Luer-lock-Spritzen ist die Nadellänge beim Aufziehen nicht so wichtig, zu lange Nadeln können jedoch unpraktisch sein. Anders gesagt gibt es keinen Grund wählerisch zu sein. Diese Frage ist bei Insulinspritzen nicht relevant, da die Nadel in diesem Fall ohnehin nicht austauschbar ist.

\subsection{Mit welcher Nadel-/Kanülenstärke sollte ich injizeren?}\label{5-16}

Diese Frage ist nicht ganz einfach und eher subjektiv, die Antwort hängt im Wesentlichen von 4 Faktoren: 1) das Trägeröl, was Du injizierst; 2) ob das Fläschchen ein zusätzliches Lösemittel beinhaltet; 3) ob Du die Geduld hast, die Nadel länger in dir zu haben; and 4) deine Bereitschaft/Fähigkeit, den Spritzenstempel stärker herunterzudrücken. Es ist eine Frage der Gemütlichkeit. Dickflüssigere Öle können mit hohen Gauge länger brauchen und mehr Druck benötigen, aber dafür tun sie weniger weh beim Reinstechen. \textbf{In der Regel ist 25G das Minimum, um keine Schmerzen zu verursachen. }Die meisten Öle gehen bis 27G gemütlich rein, während MCT-Öl bemerkenswerterweise bis 30G gut geht (Siehe Frage \ref{6-16}).

\subsection{Mit welcher Nadel-/Kanülenlänge sollte ich injizieren?}

\textbf{Ich empfehle zwischen 12.5mm und 25mm, je nach Gauge.} Unter 12.5mm erhöht wird Ausfluss wahrscheinlicher. 6.5mm kann funktionieren, je nach deiner Technik und dem Öl, den Du injizierst, but aber 12.5mm ist die sichere Wahl. Alles über 25mm ist unnötig beängstigend und schmerzhaft, ohne irgendeinen Mehrwert zu bieten. (A.d.Ü. Auch 8mm funktionieren sehr gut und sind relativ gebräuchlich in Deutschland.)

\subsection{Ist die Spritzengröße wichtig?}

\textbf{Ja, die Größe ist wichtig.} Dafür gibt es zwei Gründen. 1) Größere Spritzen mit mehr Volumen sind meist weniger genau und können zu ungenauen Dosierungen führen, und 2) größere Spritzen mit mehr Volumen sind rein physikalisch schwieriger zu benutzen. Um genau zu dosieren willst Du eine Spritze benutzen, die nicht viel größer ist als das Volumen was Du injizierst (z.B. sollten für Injektionen von weniger als 0.1ml Spritzen benutzt werden, die kleiner sind als 1ml). \textbf{Vermeide 3ml-Spritzen gänzlich, soweit Du kannst.} Du kannst sie natürlich benutzen, wenn es nichts anderes gibt, aber warum diese am häufigsten in Apotheken ausgegeben werden, erschließt sich mir nicht. Vielleicht ein schlechter Scherz. (A.d.Ü.: Im deutschsprachigen Raum sollte dies kein Problem sein. Nadeln und Spritzen sind frei verkäuflich und in einer breiten Auswahl verfügbar.)

\subsection{Wo kaufe ich Spritzen und Nadeln/Kanülen?}

Es hängt von deinem Standort ab, da der Verkauf von Nadeln und Spritzen mancherorts als Strafe gegen drogenabhängige Personen eingeschränkt ist. Sonst sind medizinische und Veterinär-Versorgungsgeschäfte gute Quellen, oder direkt bei den Herstellern. \textbf{Amazon wird nicht empfohlen} da die Qualität dort oft unsicher ist. (A.d.Ü.: Siehe oben. Im deutschsprachigen Raum Apotheke deiner Wahl, auch online.)

\subsection{Ist es okay, Nadeln oder Spritzen wiederzuverwenden?}

\textbf{Nein. Benutze Nadeln und Spritzen nur einmal. }Und teile sie auch nicht mit anderen. Das weißt Du wahrscheinlich schon, aber ich wiederhole es, weil es wirklich nicht gut oder sicher ist, es zu tun!

\subsection{Was wenn ich Injektionen machen will, es aber schwierig finde, mich selbst zu injizieren?}\label{5-21}

Du könntest ein Autoinjektor probieren. Wie es im Name steht führt der Autoinjektor die Injizierung für dich aus. Ein Autoinjektor wie der \href{https://unionmedico.com/90-super-grip/}{\textit{UnionMedico 45/90 Super Grip}} (A.d.Ü.: auf Deutsch hier \href{https://www.b12-injektion.de/}) kann 1ml-Spritzen aufnehmen und das Injizieren vereinfachen (aber Du musst immer noch selbst draufdrücken), während Autonijektoren wie der \href{https://www.owenmumford.com/us/medical-devices/autoject-2}{\textit{Owen Mumford Autoject 2}} (A.d.Ü.: auf Deutsch hier \href{https://shop.owen-mumford.de/Sonstiges/Autoinjektor-Autoject-2-mit-austauschbarer-Nadel.html}) die Nadel einer Insulinspritze ganz verstecken und von selbst herunterdrücken. Es gibt auch verschiedene 3D-gedruckte Modelle,d ie online verfügbar sind. Ich habe keine dieser Produkte getestet und dies ist keine Empfehlung.

\subsection*{Vials: Eine kleine Fläschchenkunde}
\addcontentsline{toc}{subsection}{\textemdash{} Vials}

\subsection{Worauf sollte ich bei einem Vial achten?}

(A.d.Ü. Ja, "Vial" ist kein deutsches Wort - aber eben gängig. Du liest das hier ja auch nicht auf einem Klugtelefon.) Abgesehen von einem klaren Coring (siehe unten) solltest Du auf Anzeichen wie Verfärbung, Separation der Lösung, Kontaminierung, Kristallisierung, Brüche im Glas, Staub bzw. Haare innerhalb des Gefäßes, etc. achten. Ein gut hergestelltes Vial sollte sich visuell von anderen aus der gleichen Herstellung nicht unterscheiden lassen. \textbf{Untersuche dein Vial immer, bevor Du es benutzst. Injiziere nicht aus einem Vial, das nicht gut aussieht.}

\subsection{Was ist mit dem Ausstanzen des Stopfens bzw. Coring gemeint?}\label{5-23}

Jedes Fläschchen hat ein Gummistopfen, der die Lösung schützt. Das \textit{Coring} (deutsch: Ausstanzen) passiert, wenn ein Stück vom Gummi herausgeschnitten wird/herausbricht und in die Lösung gelangt. Das kann passieren, wenn eine zu große Nadel beim aufziehen verwendet wird, wenn immer wieder an der gleichen Stelle eingestochen wird, oder wenn zu häufig eingestochen wird (z.B. wenn immer sehr geringen Mengen aus einem recht großen Vial entnommen werden). \textbf{Ein Vial mit ausgestanzten Core sollte weggeworfen werden. }Mit der \href{https://www.youtube.com/watch?v=w5F0SLoMjC8}{\textit{45-90° Technik}} kann diese Gefahr verringert werden. (A.d.Ü.: Dies betrifft nur Leur-Lock Spritzen. Insulinspritzen sind meist zu kurz dafür.)

Du willst einfach keine Gummistückchen in deinen Körper injizieren. Wenn Du größere Gummistücke siehst, könnte es auch kleinere, unsichtbare Stücke geben. Darüber hinaus soll der Gummistopfen die Lösung vor der Luft und vor Bakterien schützen und wenn es einen Loch gibt erhöht sich die Gefahr einer Kontaminierung oder Oxidierung. \textbf{Nebenbei: Bitte entferne den mittleren Teil der Metallabdeckung oben am Vial, bevor Du es benutzt. }Das mag selbstverständlich erscheinen, aber bei manchen Vials kann das verwirrend sein. 

\subsection{Wie lang ist ein Fläschchen haltbar?}

Ein geschlossenes Vial könnte sich jahrelang halten, wenn es bei stabilen Temperaturen und in Dunkelheit gelagert wird. Bei der Haltbarkeit geht es vor allem um die Gefahr der Oxidierung oder Verlust der Sterilität. Ein angefangenes Vial, das einen Konservierungsstoff beinhaltet (siehe Frage \ref{6-17}), sollte mindestens ein Jahr halten, oder wie immer lang es braucht bis es aufgebraucht ist. Oft steht auf Fläschchen "28 Tage haltbar", das ist aber nur das Minimum, was von Herstellern verlangt wird, nicht die tatsächliche maximale Haltbarkeit. (A.d.Ü. Dies ist eigentlich nur der Fall auf professionell hergestellten Vials.)

\subsection{Wie lagere ich das Fläschchen?}

Bei stabiler Raumtemperatur und dunkel. Hitze und Sonnenlicht können dem Trägeröl schaden, und Kälte kann zur Kristallisierung führen. Kristalle lösen sich wieder auf, wenn die Lösung erwärmt wird, aber wenn dies nicht vollständig passiert kann es zu Irritation beim Injizieren führen. Das trifft auf angefangene und neue Vials zu.

\subsection{Was genau ist der Hohlraum/Deadspace?}\label{5-26}

Mit \textit{Hohlraum/Deadspace} ist die Menge an Flüssigkeit gemeint, die bei einer Injektion verschwendet wird. Diese Flüssigkeit bleibt in der Kanüle/in der Spritze gefangen. Bei einer Luer-lock-Nadel/Spritze kann dies bis zu 1mL betragen, bei einer Insulinspritze kann ist es oft viel weniger bis zu nur 0.003mL. Es lohnt sich, Hohlraum zu vermeiden, da am Ende ganz schön viel Östrogen dadurch verloren geht. \href{https://hrtcafe.net/Calc/}{Dieser Rechner} kann helfen, die verschwendete Menge je nach Spritzenart einzuschätzen.

Wenn Du die Nadel/Kanüle zwischen dem Aufziehen und dem Injizieren wechselst, dann solltest Du den Spritzenstempel leicht zurück ziehen bevor Du die Aufziehnadel abnimmst, damit die darn enthaltene Flüssigkeit nicht verlorengeht. Es ist nicht viel, aber es kann einen Unterschied machen. Siehe Frage \ref{7-7} für eine andere mögliche Herangehensweise wenn Deadspace ein Problem ist.

\subsection{Was mache ich mit meinen benutzten Nadeln und Spritzen?}\label{5-27}

Werfe sie alle (vorsichtig, mit der Spitze nach unten) in eine Entsorgungsbox weg (entweder eine richtige medizinische Entsorgungsbox für medizinischen Sondermüll oder einen stich- und bruchfesten Abfallbehälter, z.B. eine Blechdose vom Kaffee). Wenn es dreiviertel-voll wird, mache es fest zu, sodass es sich nicht von selbst öffnen kann. Schreibe "BENUTZTE SPRITZEN" drauf und entsorge es nach den bei dir geltenden Richtlinien. \textbf{Du darfst es NICHT in den Restmüll tun.} (A.d.Ü: In Deutschland ist dies je nach regionalem Entsorgungsunternehmen unterschiedlich. Am besten ist es, den Behälter in der Apotheke oder bei irgendeinem Arztbesuch abzugeben, damit es fachgerecht und sicher entsorgt werden kann.)



\section{Vials besorgen}\label{sv}

\subsection{Wo bekomme ich Östrogen-Vials zum Injizieren?}

Generell gibt es zwei Möglichkeiten: \textit{pharmazeutische Quellen} und \textit{DIY-Quellen}. Für \textit{pharmazeutische Quellen} brauchst Du meistens ein Rezept vom Arzt, da HRT in den meisten Ländern nicht rezeptfrei ist (oder zumindest nicht in injizierbarer Form). \textit{DIY-Quellen} umfassen alles andere. 

A.d.Ü. Während es ein paar Apotheken in Deutschland gibt, die Injektionen herstellen gibt es dafür leider eine Unmenge an Voraussetzungen: Du benötigst eine 1) Indikation 2) einen Arzt der gewillt ist dir Injektionen zu verschreiben und 3) Geld. Während es möglich sein sollte, die Injektionen auf Kassenrezept zu bekommen stellen sich erfahrungsgemäß die meisten Ärztys quer. Bezugsquellen brauchen spezifische Rezepte - diese regulierten Quellen finden sich auf der trans*DB.

\subsection{Sollte ich pharmazeutische Quellen benutzen, oder DIY-Quellen?}

Theoretisch ist das deine Entscheidung, aber manchmal gibt es gar keine Entscheidungsmöglichkeiten. Es gibt bei jeder Variante Vor- und Nachteile. Es hält dich natürlich nichts davon ab, Östrogen aus verschiedenen Quellen zu beziehen, um die Vorteile beider Varianten zu genießen. In vielen Situationen kann das empfehlenswert sein.

\subsection*{Pharmazeutische Quellen}
\addcontentsline{toc}{subsection}{\textemdash{} Pharmazeutische Quellen}

\subsection{Was sind die Vorteile von pharmazeutischen Quellen?}

\begin{itemize}
  \item Vertrauen in der Qualitätskontrolle und Zertifizierung;
  \item Die Krankenversicherung kann es gänzlich oder teilweise übernehmen;
  \item Kann praktischer sein, je nachdem wie viel Glück Du mit deinen Ärzten hast;
  \item Das Produkt wird sehr wahrscheinlich konsistent sein;
  \item \textbf{Für die Übernahme von Operationen durch die Krankenkasse kann zumindest der Anschein von einer Benutzung von pharmazeutischen Quellen notwendig sein.} (A.d.Ü. Wird in Deutschland nicht benötigt)
\end{itemize}

\subsection{Was sind die Nachteile von pharmazeutischen Quellen?}

\begin{itemize}
  \item Kleiner (oder gar keine) Auswahl zwischen Estern; (A.d.Ü. NUR Estradiol Valerate)
  \item Möglicherweise lange Wartezeit (Monate oder Jahren);
  \item Möglicherweise nur auf Rezept (ja nach Land) (A.d.Ü.: So ist es in Deutschland);
  \item Die Krankenkasse übernimmt es vielleicht nicht ganz oder gar nicht;
  \item Wird möglicherweise in deinem Land gar nicht verschrieben;
  \item Dein Arzt kann sich einfach weigern, es dir zu verschreiben;
  \item Dein Arzt kann sich einfach weigern, dir eine neues Rezept zu geben;
  \item Lieferengpässe können dein Rezept nutzlos machen;
  \item Du wirst wahrscheinlich unter den stringenten WPATH-Richtlinien behandelt, oder was schlimmeres;
  \item Schwierig, ein Vorrat anzulegen;
  \item Dein Zugang zu HRT hängt stark von der politischen Stimmung deines Landes ab und dein Trans sein wird sehr wahrscheinlich in der medizinischen Akte vermerkt. 
\end{itemize}

\subsection*{DIY-Quellen}
\addcontentsline{toc}{subsection}{\textemdash{} DIY-Quellen}

\subsection{Was sind die Vorteile von DIY-Quellen?}

\begin{itemize}
\item In den meisten Regionen viel billiger;
\item Überall auf der Welt verfügbar;
\item Der Zugang kann viel schneller sein (Wartezeit nur für Produktion und Versand);
\item Leicht, ein Vorrat anzulegen;
\item Volle Auswahl an Estern;
\item Keine Notwendigkeit, mit einem Gesundheitssystem umzugehen;
\item Es ist wahrscheinlich mit Liebe gemacht.
\end{itemize}

\subsection{Was sind die Nachteile von DIY-Quellen?}

\begin{itemize}
  \item Sehr wahrscheinlich nicht in einem zertifizierten Reinraum hergestellt;
  \item Produktionsqualität kann je nach Quelle variieren;
  \item Kann je nach Quelle umständlich sein;
  \item Setzt Vertrauen in die Quelle voraus;
  \item Es muss eine Quelle gefunden werden;
  \item Quellen verschwinden eher als deine lokale Apotheke;
  \item Versandzeiten können variieren;
  \item Arbeiten sehr wahrscheinlich mit Kryptowährungen, was nervig ist;
  \item Wir nicht von der Krankenkasse übernommen.
\end{itemize}

Darüber hinaus wird, wie bereits erwähnt, für die meisten OPs eine gewisse Zeit mit einer "offiziellen" Hormonbehandlung vorausgesetzt. Je nach deinen Plänen kann das relevant sein. (A.d.Ü. Irrelevant für Deutschland)

\subsection{Welche Formen von injizierbaren Östrogen gibt es nur bei DIY-Quellen?}

Vor allem \textit{Estradiol Enanthate}. Bei pharmazeutischen Quellen gibt es fast immer \textit{Estradiol Valerate}, aber nicht immer mit einer Konzentration von 40 mg/ml. \textit{Estradiol Cypionate} wird manchmal verschrieben, aber selten mit Konzentrationen über 5 mg/ml oder 10 mg/ml, was die Dosierung schwierig macht. Allein die Vorteile von \textit{Estradiol Enanthate} sind ein guter Grund, DIY in Betracht zu ziehen, aber es ist möglich jeden Ester mit einer Konzentration von 40mg/ml oder mehr aus DIY-Quellen zu beziehen. \textit{Estradiol Undecylate} ist ebenfalls nur als DIY möglich, aber wie bereits erwähnt würde ich es nur experimentierfreudigen Menschen empfehlen.

\subsection{Was für Menschen stehen hinter DIY-Quellen \textit{wirklich}?}

Kommerzielle Hersteller, solidarische Projekte, deine Freunde, und vielleicht sogar du, wenn du einen unternehmerischen Geist hast!

\subsection{Wo kann ich DIY-Vials bekommen?}

Bist Du bei der Polizei oder was? Das sage ich dir nicht. Darum geht es in dieser Anleitung nicht. Dafür gibt es andere Quellen. Bleib fokussiert.

\subsection{Wie können DIY-Quellen billiger sein als pharmazeutische Quellen?}

Der Herstellungspreis für eine Vial ist ca. \$10, inklusive der Personalkosten und den amortisierten Investitionskosten. Das ist wahrscheinlich eine Überschätzung. Für die kommerziellen DIY-Quellen entstehen die meisten Kosten durch die benötigte Anonymität und der Versand. Nicht-kommerzielle DIY-Quellen habe keine solchen Kosten. Pharmazeutische Quellen haben meist keine Motivation, ihre Preise zu senken. 

\subsection{Ist DIY legal?}\label{6-11}

In den meisten Ländern, inklusive der USA, wird Östrogen nicht streng reguliert, während Testosteron mehr oder weniger streng reguliert ist. Die USA sind in der Hinsicht eine Ausnahme, da die meisten Länder den Besitz von Testosteron nicht kriminalisieren, aber eine strafrechtliche Verfolgung ist selten. \textbf{Dieser Leitfaden stellt keine Rechtsberatung dar.} 
(A.d.Ü.: Nach dem Verständnis der Übersetzerin bringt man sich in Deutschland rechtlich nicht in Gefahr, wenn man Östrogen zu DIY-Zwecken bestellt. Im schlimmsten Fall kann z.B. Östrogen aus dem Nicht-EU Ausland vom Zoll eingezogen werden. Endverbraucherinnen werden (bisher) nicht verantwortlich gemacht. Der kommerzielle Vertrieb ist wahrscheinlich eine andere Frage, aber darum geht es hier nicht).

\subsection{Ist DIY sicher?}

“DIY” als Ganzes ist weder sicher noch unsicher, aber nicht alle DIY-Quellen sind gleich. Wenn es darum geht, eine Substanz in deinen Körper zu injizieren, ist die wichtige Frage: Hast Du genug Vertrauen in der Person, die das Vial produziert hat, darin dass sie richtig und sauber gearbeitet hat, dass das Fläschchen steril ist und nur das beinhaltet, was Du brauchst? Bei den pharmazeutischen Quellen wird dieses Vertrauen aufgrund der geltenden Gesetzen und Regeln angenommen. Bei DIY-Quellen muss dieses Vertrauen erst aufgebaut werden, durch Information zum Herstellungsprozess, unabhängige Prüfungen und Feedback innerhalb der Community.

\subsection{Worauf sollte ich aufpassen um rauszufinden, ob eine DIY-Quelle vertrauenswürdig ist?}

Verlasse dich auf dein Bauchgefühl und deinen Kopf. 

\begin{itemize}
  \item Reden sie offen über ihren Produktionsprozess oder machen Angaben dazu? (Wird zum Beispiel Staub gefiltert? Die Antwort sollte ja sein!!!)
  \item Machen sie einen kompetenten Eindruck?
  \item Haben sie ihr Produkt testen lassen? 
  \item Werden sie in der Community als vertrauenswürdig eingeschätzt?
  \item Werden sie von anderen, dir vertrauten Menschen aus der Community empfohlen? Gibt es Rezensionen oder Erfahrungsberichte? 
  \item Fehler können passieren, aber wie wird damit umgegangen? Wird offen kommuniziert oder werden negative Berichte unterdrückt?
  \item Werden bei kommerziellen Quellen Probleme mit Bestellungen aufgeklärt?
  \item Werden bei kommerziellen Quellen Bestellungen angenommen, obwohl das Produkt noch gar nicht vorliegt? (Du solltest niemals etwas bestellen, was noch nicht produziert wurde!)
  \item Beinhaltet die Vial ein Konservierungsmittel? (Sollten sie!)
  \item Wie lange sind produzieren sie schon? (Diese Frage wird aus guten Gründen nicht immer beantwortet!)
  \item Wie viel produzieren sie? (Diese Frage wird aus guten Gründen nicht immer beantwortet!)
  \item Fühlt sich irgendwas einfach \textit{falsch} an?
\end{itemize}

Diese Beispielfragen können dir helfen rauszufinden, ob die Quelle vertrauenswürdig ist, und ob ihr die Qualität des Produkts so wichtig ist wie dir.

\subsection{Soll ich bei verschiedenen DIY-Quellen verschiedene Standards anwenden?}

Wahrscheinlich ja. Kommerzielle Anbieter, die dein Geld nehmen, sollten hohe Standards erfüllen können, da sie es sich leisten können. Bei Produkten die solidarisch und kostenlos verteilt verteilt werden, kann man nicht ganz so pingelig sein, was aber nicht heißt, dass sie unbedingt besser oder schlechter sind. Bei Freunden oder bei deiner eigenen Produktion kannst nur Du es einschätzen!

\subsection*{Aufbau eines Vials}
\addcontentsline{toc}{subsection}{\textemdash{} Aufbau eines Vials

\subsection{Was sollte in einem Vial enthalten sein?}

Der Inhalt setzt sich aus einem \textit{“aktiven”} Stoff und einem \textit{“Träger”} oder Hilfsstoff zusammen. Der \textit{ aktive} Stoff ist in unserem Fall das Östrogen-Ester, während der \textit{Träger} und weitere Hilfsstoffe den Rest ausmachen. Es gibt generell insgesamt drei oder vier Inhaltsstoffe im Vial: 1) Das Östrogen-Ester; 2) das Trägeröl; 3) ein Konservierungsmittel; und manchmal 4) weitere Lösemittel. Die Ester haben wir schon in der Sektion \ref{td} “ARTEN UND DOSIERUNGEN” besprochen. Vial aus pharmazeutischen Quellen beinhalten fast immer alle vier Inhaltsstoffe.

\subsection{Welches Trägeröl will ich im Vial haben?}\label{6-16}

Die Antwort zu dieser Frage ist subjektiv und hängt von der persönlichen Verträglichkeit und von möglichen Allergien ab. \textbf{Der wichtigste relevante Aspekt für die Injektion ist die Viskosität des Öls, da die Injektion dadurch bequemer oder praktischer sein kann.} Dünnere, flüssigere Öle sind, wie bereits besprochen (Siehe Frage \ref{5-16}), für das Aufziehen und Injizieren mit dünneren Kanülen von Vorteil. \textbf{MCT/MKT-Öl und Rizinusöl werden bei HRT am häufigsten benutzt. }Rizinusöl ist dickflüssiger, führt aber dafür am wenigsten zu Irritationen und wird typischerweise von pharmazeutischen Herstellern benutzt. MCT-Öl ist dünnflüssiger, führt aber bei manchen Menschen zu Irritationen und wird nur von DIY-Quellen benutzt. Baumwollsamenöl und Traubenkernöl werden manchmal eingesetzt, aber meistens nicht für HRT. Andere Öle wie z.B. Sonnenblumenkernöl oder Sesamöl werden manchmal eingesetzt, sind aber nicht empfohlen. Je nach deinen Umständen kann diese Frage irrelevant oder zwingend sein, oder Du kannst gar keine Auswahl haben. 

(A.d.Ü. Einige pharmazeutische Hersteller in Deutschland nutzen Sesamöl, eine bekannte DIY Quelle bietet Traubenkernöl an.)

\subsection{Welches Konservierungsmittel will ich im Vial haben?}\label{6-17}

Für Injektionen wird niedrig konzentriertes \textit{Benzylalkohol} (BA) am häufigsten verwendet. Das muss sein und darf nicht fehlen. \textbf{Benutze nie ein Fläschchen ohne Konservierungsmittel. }Für Menschen mit der seltenen Allergie dagegen wird meist \textit{Chlorobutanol }als Alternative verwendet, was aber eher selten bei DIY-Quellen der Fall ist, da sie den Stoff dafür extra auftreiben müssten.

\subsection{Welche zusätzliche Lösemittel will ich im Vial haben?}

Am häufigsten wird \textit{Benzylbenzoat} (BB) verwendet,  das die Lösung flüssiger macht. Das ist an sich optional, wird aber manchmal empfohlen und ist bei manchen Trägerölen und Konzentrationen notwendig. Bei manchen Menschen führt es zu Irritationen, bei anderen nicht. 
 

\section{HÄUFIGE FEHLER (UND LÖSUNGEN)}

\subsection*{Unsicherheit bei der Dosierung}
\addcontentsline{toc}{subsection}{\textemdash{} Unsicherheit bei der Dosierung}

\subsection{Meine Hormon-Werte sind nicht so, wie ich es erwartet habe. Warum?}

Dafür kann es verschiedene Ursachen geben. Modelle und Simulationen sind nicht in der Lage alle mögliche Faktoren einzubeziehen und deine Werte können dadurch abweichen. Bedenke auch, dass es mehrere Injektionen braucht, bis deine Werte stabil sind. Wenn Du vor kurzem die Dosierung verändert hast, könnte es also daran liegen. Frag am besten eine befreundete Person dir beim Aufziehen zuzuschauen, um sicher zu sein, dass Du tatsächlich so viel injizierst, wie Du denkst. Dieses Problem kommt öfters vor als man denken würde. Bei DIY-Quellen kann auch aufgrund von Unerfahrenheit oder ungenauem Equipment die Konzentration niedriger sein als erwartet. In diesem Fall solltest Du bei diesem Vial etwas mehr injizieren als sonst. \textbf{Aber am wichtigsten bleibt immer wie Du dich fühlst, und nicht die Werte an sich!. }Beachte auch, dass sogar in einer Apotheke professionell angemischte Vials manchmal eine niedrigere oder abweichende Konzentration bieten können, auch wenn das (hoffentlich) sehr selten vorkommt!

\subsection{Kann ich Hormonwerte von verschiedenen Tests vergleichen wenn ich nicht den Talwert gemessen habe?}

\textbf{Nein.} Zumindest nicht genau. Deshalb sollte immer der Talwert gemessen werden. Mehrere Stunden vor dem normalen Zeitpunkt deiner Nächsten Injektion; dann willst Du messen. Die Daten werden viel nützlicher, wenn Du so viele Variablen wie möglich ausschließt. Wenn Du dir sonst nichts aus dieser Anleitung merkst, dann bitte das: Messe den Talwert.

\subsection{An den Tagen um den Talwert herum fühle ich mich richtig schlecht. Was soll ich tun?}\label{7-3}

In den meisten Fällen ist entweder die Dosierung zu niedrig oder der Abstand zwischen den Injektionen zu groß. Das ist bei \textit{Estradiol Valerate} und \textit{Estradiol Cypionate} besonders relevant. Passe die Dosierung (innerhalb der empfohlenen Werten) oder deinen Rhythmus an. Lass dir Zeit und finde heraus, was sich für dich gut anfühlt. Insbesondere bei \textit{Estradiol Valerate} kann es sein, dass deine Dosierung tatsächlich zu *hoch* ist und nicht zu tief, da die großen Schwankungen innerhalb des Zyklus die Verstimmung auslösen können. Kurz gesagt: wechsle zu \textit{Estradiol Enanthate}, wenn Du kannst.

\subsection*{Injektionsprobleme}
\addcontentsline{toc}{subsection}{\textemdash{} Injektionsprobleme}

\subsection{Die Injektion ist schwieriger, wenn es kalt is. Was soll ich tun?}

Erwärme das Fläschchen vor dem Aufziehen, und erwärme die Spritze vor dem Injizieren. Beides kannst Du machen, indem Du das Fläschchen oder die Spritze zwischen den Handflächen rollst oder hältst. Mach dir das zur Gewohnheit, um deine Injektionsroutine zu erleichtern.

\subsection{Die Injektion schmerzt mehr, wenn es kalt ist. Was soll ich tun?}

Wärme deine Injektionsstelle vorher auf. Du kannst die Muskeln mit einer Massage oder einer warmen Dusche vor dem Injizieren entspannen (zum Beispiel indem Du, wenn Du am Bein injizierst, mit dem heißen Wasserstrahl gezielt die Stelle aufwärmst).

\subsection{Ich habe nach der Injektion geblutet. Werde ich sterben?}

Nein. Das bedeutet nur, dass Du ein kleines Blutgefäß getroffen hast, das kann passieren. Vielleicht kriegst Du einen blauen Fleck oder es tut später etwas weh. Wenn Du ein süßes Dino-Pflaster drauf klebst, wird es schneller heilen. (A.d.Ü. Einhornpflaster helfen auch!)

\subsection{Es gab etwas Luft in meiner Spritze. Werde ich sterben?}\label{7-7}

Nein. Du willst natürlich nicht nur Luft injizieren, und zu viel Luft kann die Dosierung beeinflussen, aber alles unter 0.1ml ist sehr wahrscheinlich egal. Bei manchen Substanzen kann es sogar empfohlen sein. Zum Beispiel wird bei der \textit{"Air-Lock-Technik"} (eine Standardtechnik zum Spritzen von Flüssigkeiten, die reizen oder Flecken hinterlassen können; kein wichtiges Wissen für die HRT) normalerweise 0,1–0,3 ml Luft gespritzt) also musst du dir keine Sorgen machen. Du machst ja keine intravenösen Injektionen, ist also für deine HRT nicht relevant. 

\subsection{Ein Teil der Flüssigkeit ist aus der Injektionsstelle ausgelaufen. Habe ich meine Injektion verschwendet und/oder werde ich sterben?}

Nein. Es kann aus vielen Gründen dazu kommen, dass ein Bisschen ausläuft, und es ist nur selten genug, um einen Unterschied zu machen. Die Injektion musst Du nicht wiederholen. Versuche in der Zukunft, die Nadel 5-10 Sekunden drin zu lassen und danach auf die Stelle zu drücken. Wenn Du dir besonders viele Sorgen um das Auslaufen machst, kannst Du versuchen, die vorher erwähnte Air-lock-technik anzuwenden. 

\subsection{Manchmal tut es nach einer Injektion richtig weh. Werde ich sterben?}

Nein. Auch wenn Du alle Anweisungen hier sonst gefolgt hast, kann es trotzdem vorkommen, dass die Flüssigkeit sich an einer besonders unangenehmen Stelle sammelt. Das wird beim nächsten mal besser! \textbf{Pass auf, deine Injektionsstellen zu durchzuwechseln / zu rotieren!} Du willst nicht, dass sich durch das wiederholte injizieren an der gleichen Stelle eine Narbe bildet. Wenn eine Injektionsstelle bereits weh tut, willst Du sie nicht noch schmerzhafter machen.

\subsection{Ich hab nach der Injektion echt starken Juckreiz und Reizungen an der Stelle. Werde ich sterben?}

Wahrscheinlich nicht. Dafür kann es verschiedene Gründe geben. Am schlimmsten wäre eine Infektion, aber das ist in den meisten Fällen sehr unwahrscheinlich. \textbf{Geh sofort zu einem Arzt wenn Du Fieber oder starke Schmerzen hast oder Muskelschmerzen, Eiter, sich ausbreitende Rötungen oder andere Zeichen einer Infektion bemerkst. }Jucken, leichte Rötungen, leichtes Anschwellen oder Wärme an der Injektionsstelle gehen jedoch meistens darauf zurück, dass das Östrogen und das Trägeröl sich in der Lösung getrennt haben. Siehe unten. Es ist auch möglich, dass Du eine allergische Reaktion auf das Trägeröl (oder einen anderen Inhaltsstoff) erlebst, aber wenn dies plötzlich auftaucht und die vorigen Injektionen ohne problemlos waren, wird es eher an einer Separation der Lösung liegen.

\subsection{In meinem Vial sind Kristalle. Kann ich es trotzdem benutzen?}

Sehr wahrscheinlich liegt es daran, dass es zu kalt geworden ist. Wärme es leicht auf und schüttel es, um die Lösung wieder zu mischen. Wenn die Kristalle nicht verschwinden kann es sein, dass die Lösung sich gänzlich getrennt hat. Mit viel mehr Wärme und Durchmischung könnten die Kristalle vielleicht wieder verschwinden, aber, wenn es möglich ist, solltest du einfach auf ein neues Vial ausweichen.

 

\section{PROGESTERON}

\subsection{Will ich Progesteron nehmen?}

\textbf{Wahrscheinlich.} Aus was für einem Grund auch immer ist diese Frage kontrovers. (A.d.Ü.: Die Gründe sind hauptsächlich historischer Natur und auf eine alte WPATH Studie zurück zu führen, die jedoch kein bioidentisches Progesteron untersucht hat; siehe unten) Gegner (vor allem Ärzte) argumentieren, dass eine feminisierende Wirkung durch keine Studien belegt werden und es dementsprechend nicht genommen werden sollte. Abgesehen davon, dass transfeminine Themen chronisch nicht erforscht sind, ist Progesteron heuristisch gesehen ein zentrales weibliches sexuelles Hormon, das viele wichtige Funktionen im Körper und im Gehirn erfüllt. Ungeachtet der äußerlichen Feminisierung ist ein wichtiges Hormon eine gute körperliche und mentale Gesundheit sicherstellt und sollte nicht leichtfertig übergangen werden.

\subsection{Was ist der Unterschied zwischen “Progesteron” und “Progestine” / ”Progestagene”?}

Die Hormone, welche auf die Progesteron-Rezeptoren wirken, heißen “Gestagene” und können sowohl bioidentisch/natürlich als auch synthetisch sein. Das wichtigste, natürliche und biodentische Hormon darunter ist “Proges\textbf{teron}”. Synthetische Gestagene werden manchmal als “Proges\textbf{tine}” bezeichnet. Die Bezeichnung sind sich alle recht ähnlich und werden oft untereinander verwechselt, obwohl sie \textbf{nicht }gleichwertig sind. (A.d.Ü.: Die deutsche Nomenklatur weicht in diesem Kontext relativ stark von der englischen ab, das Problem bleibt aber ein Stück weit bestehen.)

\subsection{Will ich Progesteron oder ein Progestin/Gestagen?}

Progesteron. Du willst NUR bioidentisches Progesteron.

\subsection{Was ist das Problem mit Progestine/Gestagene?}

Progestine, darunter typischerweise \textit{Medroxyprogesteron}, \textit{Medroxyprogesteronacetat }, oder \textit{Levonorgestrel}, sind oft mit den schädlichen Nebeneffekten (Brustkrebs, Blutgerinnsel, Depression, etc.) verbunden, die fälschlicherweise Progesteron zugeschrieben werden. Sie sind nicht bioidentisch und verhalten sich deshalb nicht wie Progesteron, und können nicht direkt damit verglichen werden.

\subsection{Wie trägt Progesteron zur Feminisierung bei?}

Es wird angenommen, dass Progesteron vor allem eine wichtige Rolle beim Brustwachstum und bei der Libido spielt, aber es ist wie gesagt auch davon abgesehen ein wichtiges Hormon. Es ist auch ein Antigonadotropin (das heißt, es trägt zur Unterdrückung von Testosteron bei) was manchmal relevant sein kann.

\subsection{Ist es relevant, wann ich mit Progesteron anfange?}

Das wissen wir nicht. Es wird manchmal behauptet, dass eine frühe Einnahme das langfristige Brustwachstum stören kann, aber das ist rein theoretisch und es gibt anekdotische, gegenteilige Berichte. Somit ist die Antwort unklar. Eine konservative Empfehlung wäre, ca. bis ein Jahr nach Anfang der HRT (also bis Tanner Stadium 3 oder 4) zu warten, für den Fall, dass es einen Unterschied macht.

\subsection{Wie wird Progesteron normalerweise eingenommen?}

Abgesehen von örtlichen Anwendungen wird es meist als Pille verabreicht. Es wird als Pille verschrieben, ist aber als Zäpfchen effektiver. Sprays und Salben zur örtlichen Anwendung funktionieren auch gut.

\subsection{Meinst Du es ersnt, dass das Progesteron als Zäpfchen, also rektal, eingenommen werden soll?}

Progesteron wird rektal ganz anders metabolisiert als oral, da es oral erstmal durch die Leber geht. Oral eingenommenes Progesteron wird primär zu \textit{Allopregnanolon} verarbeitet, das zu schwerer Müdigkeit führen kann, während rektales Progesteron primär zu Progesteron selbst verarbeitet wird, was wir wollen (ein wenig davon wird trotzdem zu anderen Stoffen verstoffwechselt). Manche Menschen nehmen extra orales Progesteron ein, um besser schlafen zu können, aber zu viel \textit{Allopregnanolon }kann manchmal auch zu negativen psychischen Nebenwirkungen führen.

\subsection{Wie nehme ich Progesteron als Zäpchen ein?}

Etwas Wasser auf der Pille sollte reichen, dann trocknen und Hände Waschen. Offensichtlich solltest Du innerhalb der nächsten Stunde oder so nicht auf Klo gehen, daher empfiehlt es sich vor dem Schlafen. Falls Du Probleme damit hast, dass es sich nicht auflöst, kannst du es probieren, die Kapsel vorher anzupieksen, aber das sollte meistens keine Problem sein. Bei großen, hausgemachten Zäpchen mit Kokosöl solltest Du dir bewusst sein, dass das Öl nicht in dir bleiben will. (A.d.Ü. Gleitgel hilft ebenfalls sehr gut.)

\subsection{Wie viel Progesteron sollte ich nehmen?}

Bei Pillen ist eine tägliche 100-200mg vor dem Schlafen gehen Standard. Das ist eine etwas arbiträre Dosis; 200mg ist das Maximum, das von den meisten Ärzten verschrieben wird. Manche Menschen nehmen mehr als 200mg, aber eine kurzfristige Erhöhung der Werte kann zu einem unangenehmen Crash führen, siehe die nächste Frage.

Bei örtlicher Anwendung auf der Haut weiß es niemand, da diese Verabreicherungsform sehr variabel ist und es keine Leitlinien zu den gewünschten Werten oder zur Frequenz (wahrscheinlich täglich) gibt, da Progesteron in der trans HRT nicht genug untersucht ist. Deshalb würde ich empfehlen, deine Dosis langsam anzupassen um zu verstehen, wie Progesteron bei dir wirkt.

\subsection{Bringt es Vorteile, Progesteron "zyklisch" einzunehmen?}\label{8-11}

Nein. Manche Menschen tun es, um den Zyklus einer cis Frau zu imitieren, aber es gibt keine Gründe, positive Effekte zu erwarten. Es kann sogar eher zu PMS-ähnlichen negativen Symptomen führen. Die einzige Ausnahme wäre beim Verdacht einer Intergeschlechtigkeit. Sonst empfehle ich es nicht. Siehe Frage \ref{11-10}.

\subsection{Wie lange soll ich Progesteron einnehmen?}

So lange wie Du Östrogen einnimmst und so lange wie Du willst. Also wahrscheilich für immer.

Manchmal sagen Leute (oder Ärzte), man sollte Progesteron nur für X Jahren einnehmen. Es gibt null theoretische oder empirische Gründe für diese arbiträre Empfehlung. Es macht genauso viel Sinn wie wenn eine cis Person (oder, spezifischer: ein Ärtzy) eine trans Person fragen würde, wie lange sie vorhat, HRT zu nehmen\textemdash{}oh nee warte das fragen die echt!

\subsection{Kann Progesteron in \textit{Dihydrotestosteron} (DHT) umgewandelt werden?}

Nein. Naja, ganz genau gesehen schon, aber auch wieder nicht. Es handelt sich dabei größtenteils um einen Mythos \href{https://whsah.co/posts/rethinking-progesterone-and-androgens/}{wie von alix in diesem Artikel ausführlich dargelegt}, jedoch kann es bei Menschen mit \textit{ nicht-klassischem adrenogenitalen Syndrom zu negativen Nebeneffekten durch erhöhte Androgenaktivität aufgrund der Einnahme von Progesteron kommen. In solchen Fällen sollte Progesteron nicht eingenommen und eine passende Diagnose und Therapie für Nebennierenerkrankungen gesucht werden.

\subsection{Gibt es neben der Einnahme von Tabletten noch weitere Vorteile bei der topischen Anwendung von Progesteron?}

Vielleicht. Es ist eine Alternative zu den Pillen, insbesondere im Falle eine Erdnuss-Allergie (die meisten Pillen beinhalten Erdnussöl), aber auch hier ist die Dosierung unklar. Manche Menschen finden mehr Progesteron besser. Pass auf dich auf und hab Spaß.

Zur Klarheit: Salben können auf den Schenkelinnenseiten (oder anderswo falls so angewiesen) aufgetragen werden, oder optional skrotal (dort ist die Haut dünn und besonders gut durchblutet), insbesondere bei Sprays. Und nein, das Progesteron direkt auf den Brüsten aufzutragen wird sie nicht größer oder schneller wachsen lassen. 

\subsection{Kann ich Progesteronpulver schnupfen?}

Bitte nicht. Es ist ziemlich schlecht für die Nebenhöhlen. Es ist nicht schwer, ein Spray selbst herzustellen, dazu gibt es Anleitungen. Mach das lieber. Es ist viel effektiver, konsistenter und sicherer.

\subsection{Wo bekomme ich Progesteron?}

Progesteron ist bei DIY-Quellen oft teurer, da größere Mengen an Hormonen benötigt werden, daher solltest Du es idealerweise von pharmazeutischen Quellen über die Krankenversicherung beziehen. Es gibt auch sogenannte "graue" Apotheken aus dem Ausland, aber dort zu bestellen ist oft schwieriger. Salben zur örtlichen Anwendung sind in manchen Ländern auch rezeptfrei erhältlich, aber je nach Konzentration machen sie vielleicht wirtschaftlich keinen Sinn.  (A.d.Ü. Ärztliche Verordnung braucht eine Indikation; gängige, rektal verwendbare Markennamen sind Utrogest, Famentia und Progestan.)

\subsection{Ich würde gern mehr zu Progesteron im Kontext von HRT lesen. Welche Quellen bieten sich an?}\label{8-17}

Ursprünglich war hier ein Dokument verlinkt, das ich aber entfernt habe, da es teilweise irreführend war. Das Problem mit Progesteron ist, dass sich niemand über einen einzigen Aspekt davon einig ist. Ich kenne keine einzige Quelle, die von allen als gut akzeptiert wird. Ich sag's dir, es können sich nicht mal alle darüber einigen, dass es mit "P" beginnt. \textbf{Das wichtigste ist, dass Progesteron für die volle Feminisierung oder für Brustwachstum nicht zwingend notwendig ist. Wenn es keine Kontraindikationen gibt, lohnt es sich wahrscheinlich jedoch, es einzunehmen.}

Es ist anzumerken, dass es im Zusammenhang mit Gestagenen unzählige Mythen und Lügen gibt, die von Befürwortern wie Kritikern gleichermaßen erfunden wurden. Dies erschwert es zusätzlich, die Wahrheit aus der ohnehin lückenhaften Forschung herauszufiltern. Fantastische Behauptungen über magische Vorteile und die Panikmache über angebliche, haltlose Risiken sind gleichermaßen kontraproduktiv, wobei Letzteres meiner Meinung nach noch schlimmer ist, wenn es von einer medizinischen Autorität stammt, sei es aus Nachlässigkeit oder böswilliger Absicht.

\subsection{Interagiert Progesteron mit anderen HRT-bezogenen Medikamenten?}\label{8-18}

Wenn Du 5-Alpha-Reduktasehemmer wie \textit{Finasterid} und \textit{Dutasterid} einnimmst (Siehe Sektion \ref{AA} “ANTIANDROGENE”, oder les weiter) können diese die Verarbeitung von Progesteron in \textit{Allopregnanolon} beeinflussen, was wiederum in manchen Menschen zu depressiven Verstimmungen führen kann, egal wie sie das Progesteron einnehmen. Es ist nicht ganz klar, inwieweit die Art der Einnahme der 5-Alpha-Reduktasehemmer (örtlich oder oral) eine Rolle spielt, aber eine niedrigere systemische Absorption durch örtliche Anwendung könnte diese negative Nebeneffekte reduzieren. Es wird empfohlen, die Hemmer nicht einzunehmen, wenn diese Nebeneffekte bei dir auftreten, aber sie treten nicht in allen Fällen auf. Merke, dass diese depressive Verstimmungen bis zu einem Monat nach der Unterbrechung der Einnahme anhalten können. 
 

\section{TESTOSTERON}\label{T}

\subsection{Warum wollen nicht \textbf{gar kein} Testosteron?}

Testosteron ist ein lebenswichtiges Hormon, das eine Schlüsselrolle in deiner Gesundheit und deinem Wohlbefinden spielt. Wir wollen es für die Feminisierung unterdrücken, aber extrem niedriges Testosteron (weniger als 10 ng/dl, oder 0.35 nmol/L) kann sich negativ auswirken: Verlust der Libido, Antriebslosigkeit, körperliches Schwäche (über den Muskelverlust aufgrund der HRT hinaus), Konzentrationsschwierigkeiten, Schlaflosigkeit, etc. Diese Symptome ähneln wohlgemerkt denen von cis Frauen in den Wechseljahren. Cis Frauen haben auch Testosteron in ihrem Körper, das braucht also nicht deine Sorge sein. \textbf{Gute Hormonwerte sind wichtig!}

\subsection{Gibt es Situationen, in denen ich Testosteron zusätzlich einnehmen möchte?}\label{9-2}

Ja. Wenn Du die eben beschriebenen Symptome spürst und deine Östrogenwerte sonst gut sind, könntest Du überlegen, eine Mikrodosis Testosteron zusätzlich einzunehmen. Vielleicht willst Du deine Fähigkeit, Erektionen zu bekommen, verbessern, oder die Atrophie deiner Geschlechtsteile im Vorfeld einer OP entgegenwirken, oder einfach experimentieren, welche Hormon-Kombination sich für dich am besten anfühlt. Alles gute Gründe, Testosteron in einem anderen Kontext als vor der HRT zu erkunden.

\subsection{Falls ich Testosteron zusätzlich einnehmen will, wie könnte ich es tun?}

Es gibt einige Möglichkeiten. Testosteron gibt es entweder als Injektionslösungen oder als Sprays/Gels, wie beim Östrogen. Ein örtliche Verabreichungsform wie Gel wird am ehsten verschrieben. Örtliche Verabreichungsformen haben die gleichen Nachteile, die wir schon bei Östrogen besprochen haben, aber in diesem Fall ist eine genaue Dosierung weniger wichtig.

\subsection{Was sind die örtlichen Verabreichungsformen für Testosteron?}

Es gibt Gel und Salben. Meistens wird Gel verschrieben, aber manche Apotheken können eine Salbe mit geringer Penetrationsrate herstellen, falls es nur um eine örtliche Anwendung auf den Genitalien geht. Das ist aber schwierig zu bekommen und oft teurer.

\subsection{Ist bei Testosteron der Ort der Anwendung wichtig?}

Es hängt davon ab, ob es ein Gel oder eien Salbe ist. Bei einer örtlichen Salbe wie oben beschrieben sollte sie direkt auf den Genitalien aufgetragen werden. Gel wird auf den Oberarmen oder Schultern aufgetragen. Pass auf, nichts anzufassen, bis es wirtlich ganz trocken ist!

\subsection{Wie viel und wie oft sollte ich Testosteron anwenden?}

Je nach Geschmack. Es hängt vor allem davon ab, wie Du dich fühlst. Falls Du zu viel nimmst, können Testosteron-Nebeneffekte auftreten (z.B. ölige Haut oder Körperbehaarung), aber nur Du weißt, was sich gut für dich anfühlt. Eine wöchentliche Injektion von 5 bis 10mg \textit{Testosteroncypionat} könnte für dich funktionieren, aber die 1-prozentige Gels, die oft in 25/50mg Packungen kommen, können mehr Variation herbeiführen. Ein halbe Packung ist fast immer zu viel, vor allem nicht täglich. Ich würde dir raten, mit viel weniger anzufangen, als Du denkst zu brauchen, und dich so heran zu tasten um zu sehen wie es sich anfühlt.

\subsection{Wo kann ich Testosteron bekommen?}

Du könntest in deinem lokalen Fitnessstudio nach den muskulösesten Bodybuilder Ausschau halten und dann höflich fragen. Achtung: Das war ein Witz. Siehe Frage \ref{6-11} “Ist DIY legal?”

\subsection{Können andere Steroide Testosteron ersetzen, im HRT-Kontext?}

Anabol-androgene Steroide, also Stoffe, die in ihrer Struktur testosteron ähneln, sind nicht alle gleichwertig. Oft benutzte Schwarzmarkt-Steroide wie \textit{Trenboloneazetat} haben viele negative Nebeneffekte, aber Steroide wie \textit{Nandrolondecanoat }werden manchmal bei postmenopausalen cis Frauen eingesetzt, da sie relativ niedrige androgene Eigenschaften haben. Das macht sie für transfeminine Menschen auch interessant. Nichtsdestotrotz ist es unwahrscheinlich, dass dir etwas anderes als Testosteron verschrieben wird. (A.d.Ü.: im Original wird hier explizit auf die US-Lage hingewiesen. Wie es im deutschsprachigen Raum läuft und inwiefern hierzulande Alternativen verschrieben werden, wissen wir nicht. Schreib uns, wenn Du es weißt!)

\subsection{Was ist die Beziehung zwischen Testosteron und \textit{Dihydrotestosteron} (DHT)?}

\textit{Dihydrotestosteron} wird im Körper auf der Basis von Testosteron durch das Enzym 5α-Reduktase gebildet, dabei wird ca. 5\% des Testosterons im Körper umgewandelt. Grob gesagt, wenn der Testosteronwert richtig unterdrückt ist (oder wenn Du eine geschlechtsangleichende OP hattest), dann sollte es nicht viel Testosteron zum Umwandeln geben, der Wert wird aber nicht null sein, da einiges immer noch lokal produziert wird. Je nachdem, wie es bei deinem Körper läuft, könnte dies ein Grund sein, ein 5$\alpha$-Reduktase-Hemmer zu supplementieren, wie in der nächsten Sektion besprochen wird. Zur Erinnerung, \textit{Dihydrotestosteron }sorgt für Körperbehaarung und androgenen Haarverlust.

\textbf{Für die Transmascs, die hier mitlesen} Ich will hier kurz besprechen, dass es bisher nicht bekannt ist, inwieweit dieses Hormon beim bottom growth eine Rolle spielt, sei es bei der Geschwindigkeit oder Größe, im Kontext der Unterdrückung von 5$\alpha$-Reduktase. Das heißt: bekannt ist, dass \textit{Dihydrotestosteron }eine primäre Rolle bei der Penis-Entwicklung spielt, aber es ist unklar, inwieweit die Abwesenheit davon eine transmaskuline Person treffen könnte. Wenn wir das Wissen zur Behandlung von Mikropenissen anwenden, wissen wir, dass eine lokal angewandte Salbe effektiver ist als Injektionen, insbesondere wie \textit{Dihydrotestosteron }-Salbe bei Patienten sinnvoll ist, die auf Testosteron nicht reagieren (wie bei 5$\alpha$-Reduktase-Defizienz). Etwas zum Nachdenken. Oliver Longdick soll sich darum kümmern!

 

\section{ANTIANDROGENE}\label{AA}

\subsection{Was sind "Antiandrogene"?}

\textit{Antiandrogene, }oft "Testoblocker" oder nur "Blocker" genannt, verhindern die Wirkung von Androgenen (also Testosteron) im Körper, und deshalb heißen sie auch so. Es gibt viele verschiedene Antiandrogene und sie werden oft als Teil der HRT verschrieben. Sie werden gebraucht, wenn die Person noch Testosteron produziert und eine Form der HRT macht, die sich für Monotherapie nicht eignet, aber an sich sind sie nicht wünschenswert. Es muss auch bemerkt werden, dass (die meisten) Antiandrogene die tatsächlichen Testosteronwerte nicht bedeutend unterdrücken, sondern die Effekte vom Testosteron im Körper reduzieren/vermeiden. Das ist bei der Auswertung von Blutwerten und so weiter wichtig.

\subsection{Warum würde ich keine Antiandrogene wollen?}

Das größte Problem mit den meisten Antiandrogenen ist, dass sie oft unerwünschte Nebenwirkungen haben und gar nicht notwendig wären, wenn das Testosteron durch genug Östrogen unterdrückt wird. Diese Nebenwirkungen könnten also (in den meisten Fällen) durch eine wohldosierte Monotherapie prinzipiell vermieden werden. Eine geschlechtsangleichende OP macht sie ebenfalls (in den meisten Fällen) überflüssig.

\subsection{Wann könnte ich Antiandrogene wollen?}

Wenn Du nicht "die meisten Fälle" bist, wenn es dir einen inneren Frieden bringen würde, oder wenn deine Krankenversicherung das zur Voraussetzung für andere Prozeduren machst, dann könntest Du Antiandrogene gebrauchen. Die Stoffe, die als Antiandrogene verwendet werden, können auch andere Effekte haben, die für deine Gesundheit hilfreich sein können. Und falls Du Androgene supplementierst, könntest Du dir einen \textit{Dihydrotestosteron }-Blocker wünschen, um die Nebeneffekten (Körperbehaarung und Haarverlust) zu minimieren. Das hängt aber auch davon ab, ob Du bioidentisches Testosteron einnimmst(z.B. \textit{Nandrolondecanoat}), da nicht alle Androgene sich gleich verhalten.

\textbf{Es muss angemerkt werden, dass die Anwendung von Antiandrogenen am Beginn einer geplanten Monotherapie weder notwendig noch empfohlen ist.} So oder so kommt es zu einer Anpassungszeit, wenn der Körper sich auf die neuen Hormone einstellt. Es gibt also keinen Grund, die Sache komplizierter zu machen. Mach dir keine Sorgen.

\subsection{Was für Antiandrogene gibt es?}

Im Kontext von HRT werden \textit{Spironolacton}, \textit{Bicalutamid} und \textit{Cyproteronacetat} zur Unterdrückung von Testosteron verwendet. Die Medikamente zur Unterdrückung der Umwandlung von Testosteron in \textit{Dihydrotestosteron} (DHT), die “5$\alpha$-Reduktase-Hemmer”,  sind \textit{Finasterid} und \textit{Dutasterid}. GnRH-Analoge wie \textit{Leuprorelin} und \textit{Triptorelin} werden als Pubertätsblocker verwendet, aber in manchen europäischen Ländern werden sie auch bei Erwachsenen angewendet.

\subsection{Wann sollte ich \textit{Spironolacton} in Betracht ziehen?}

Da für die antiandrogene Wirkung sehr hohe Dosen mit signifikanten Nebeneffekten erforderlich sind, würde ich \textit{Spironolacton} nur dann empfehlen, wenn Du von den anderen Effekten etwas hättest, z.B. der Wirkung als  Aldosteron-Antagonist im Kontext von Blutdruckproblemen oder Schwellungen. \textbf{Wenn Du darauf bestehst, \textit{Spironolacton} einzunehmen, dann bitte nicht mehr als 100mg täglich.} Der schlechte Ruf ist begründet. Es ist sozusagen der Teufel.

Falls Du es nicht weißt, einige der Nebeneffekte sind: Falls du nicht weißt, was die Nebenwirkungen sind: Gehirnnebel, Lethargie, schlechtes Gedächtnis, häufiger Harndrang, niedriger Blutdruck, niedriger Natriumspiegel/Elektrolytungleichgewicht usw. Mit anderen Worten: \textit{Spironolacton} ist ein blutdrucksenkendes Diuretikum, das ein mittelmäßiges Antiandrogen ist und normalerweise in hohen Dosen bei ansonsten gesunden Leuten für eine fragwürdige Off-Label-Anwendung verschrieben wird. In jedem anderen medizinischen Kontext wäre das Verschreiben (oder SOLLTE es zumindest!) angesichts der unerwünschten Nebenwirkungen und der bereits verfügbaren, vorzuziehenden Alternativen höchst unratsam sein, aber so sieht es nun mal im Bereich der Trans-Gesundheitsversorgung aus.

\subsection{Wann sollte ich \textit{Bicalutamid} in Betracht ziehen?}

Wenn du ein Antiandrogen nehmen willst, ist \textit{Bicalutamid} wahrscheinlich das Mittel der Wahl für dich. Es wird im Allgemeinen gut vertragen, abgesehen von 1\% der Fälle, in denen abnormale Leberwerte und Symptome einer Leberfunktionsstörung auftreten, aber ansonsten wirkt es mit relativ geringen Nebenwirkungen. Wenn du Bicalutamid nimmst, solltest du regelmäßig deine Leberwerte überprüfen lassen, um sicherzustellen, dass sie im Normbereich liegen. Die Risiken für die Leber hängen eher von deinem Körper ab als von der kumulativen Wirkung, sodass sich eventuelle Probleme wahrscheinlich innerhalb des ersten Jahres zeigen würden. Ansonsten sollte es keine Probleme geben.

\subsection{Wann sollte ich \textit{Cyproteronacetat} in Betracht ziehen?}

Wahrscheinlich nie. Nimm lieber \textit{Bicalutamid}.

Das langfristige Risikoprofil ist schlecht, und ich kann mir keine Situation vorstellen, in der ich das einer anderen Lösung vorziehen würde. Du kannst alles erreichen, was \textit{Cyproteronacetat} kann, indem du einfach mehr Östrogen nimmst und Progesteron zu deiner Behandlung hinzufügst. (A.d.Ü. Ein besonderes "yey!" für die deutsche Trans-Gesundheitsversorgung in der CPA regulär verschrieben wird.)

\subsection{Wann sollte ich \textit{Dutasterid} in Betracht ziehen?}

Wenn du dir große Sorgen wegen Haarausfall machst und/oder deine Chancen auf Haarwachstum maximieren willst, solltest du vielleicht \textit{Dutasterid} nehmen. Wenn dein Testosteronspiegel ansonsten unterdrückt ist, sollte es theoretisch nicht viel bringen, da dein \textit{Dihydrotestosteronspiegel} (DHT) relativ niedrig sein sollte, aber der Körper kann kompliziert sein, sodass es für dich vielleicht trotzdem interessant sein könnte. Siehe auch Frage \ref{11-14}.

st zu beachten, dass \textit{Dutasterid} bei manchen Menschen unerwünschte Auswirkungen auf die Stimmung haben kann. (A.d.Ü. Bis hin zu starken Depressionen.) In diesem Fall wird dringend empfohlen, die Einnahme abzubrechen. Beachte auch, dass diese depressiven Wirkungen bis zu einem Monat nach Absetzen des Medikaments spürbar sein können.

\subsection{Wann sollte ich \textit{Finasterid} in Betracht ziehen?}

Wenn dir \textit{Dutasterid} nicht verschrieben wird oder deine Versicherung speziell \textit{Finasterid} für die Haarbehandlung vorschreibt, ist \textit{Dutasterid} die bessere Wahl, weil es besser wirkt und verträglicher ist.

Beachte, dass Finasterid bei manchen Leuten depressive Verstimmungen auslösen kann. In diesem Fall solltest du es besser absetzen. Denk auch daran, dass diese depressiven Effekte bis zu einem Monat nach dem Absetzen noch spürbar sein können.

(A.d.Ü. Das deutsche Gesundheitsministerium hat zu Finstarid und Dutasterid einen \href={https://www.bfarm.de/SharedDocs/Risikoinformationen/Pharmakovigilanz/DE/RHB/2025/rhb-finasterid.pdf?__blob=publicationFile}{Rote-Hand-Brief} im September 2025 herausgegeben und rät die Verwendung der Medizin abzubrechen wenn es zu Suizidgedanken kommt.)

\subsection{Wo kann ich Antiandrogene bekommen?}

Abgesehen davon, dass sie dir von deinem Arzt verschrieben werden oder vielleicht rezeptfrei erhältlich sind, gibt es auch die Möglichkeit, sie über ausländische Apotheken auf dem Graumarkt zu kaufen. Das sind einfach Apotheken in einem anderen Land, wobei der Kauf dort oft mit einigen Hürden verbunden ist. \textit{Dutasterid} und \textit{Finasterid} sind aufgrund ihrer Verbreitung als Medikamente gegen Haarausfall in der Regel am einfachsten rezeptfrei zu bekommen.

 

\section{MYTHEN UND IRRTÜMER}\label{MM}

\subsection*{Häufig gestellte Fragen}
\addcontentsline{toc}{subsection}{\textemdash{} Häufig gestellte Fragen}

\subsection{Sollte ich mir wegen Blutgerinnseln Sorgen machen?}\label{11-1}

Ja und nein. Es stimmt, dass es einen Zusammenhang zwischen der Östrogendosis/-konzentration und dem Risiko für Blutgerinnsel gibt, aber das hängt vor allem davon ab, wie das Östrogen verabreicht wird und um welche Art von Östrogen es geht. Synthetische Östrogene sind echt ein Grund zur Sorge und erhöhen das Risiko für Blutgerinnsel deutlich, aber bioidentische Östrogene sind nicht so bedenklich. Vor allem die Art der Verabreichung macht einen großen Unterschied. Oral eingenommenes bioidentisches Östrogen passiert die Leber, was das erhöhte Risiko für Blutgerinnsel verursacht. Injektionen umgehen die Leber, und es gibt keine Hinweise oder Gründe zu der Annahme, dass Injektionen von bioidentischem Östrogen ein signifikant erhöhtes Risiko darstellen, das über die natürlichen Unterschiede zwischen Testosteron und Östrogen hinausgeht. Trotz dieser Unterschiede hält sich die weit verbreitete Panikmache gegenüber allen Östrogenen seit Jahrzehnten hartnäckig.

\textbf{Wenn du dich einer Operation unterziehst, solltest du wissen, dass WPATH es nicht mehr empfiehlt, die HRT wegen der Gefahr von Blutgerinnseln zu unterbrechen.} Viele Chirurgen schreiben es immer noch in ihre Vorbereitungsrichtlinien, weil sie sich Sorgen wegen Blutgerinnseln machen, aber das ist Quälerei, die widerlegt wurde, und sogar die WPATH empfiehlt es nicht mehr. Echt krass, ich weiß. Per \href{https://www.tandfonline.com/doi/pdf/10.1080/26895269.2022.2100644}{WPATH SOC 8 Statement 12.19}: \blockquote{Nach genauer Untersuchung haben die Forscher keinen perioperativen Anstieg der VTE-Rate [KT: \textit{venöse Thromboembolie}, also ein Blutgerinnsel] bei Transgender-Personen festgestellt, die sich einer Operation unterzogen haben und währenddessen die Sexualhormonbehandlung weitergemacht haben, im Vergleich zu Patienten, bei denen die Sexualhormonbehandlung vor der Operation abgebrochen wurde.(Gaither et al., 2018; Hembree et al., 2009; Kozato et al., 2021; Prince \& Safer, 2020).} Ich sollte das eigentlich in einer separaten Frage beantworten, aber um die Links nicht zu zerstören, müsste das am Ende eines Abschnitts stehen, und ich finde, das ist zu wichtig dafür, also schreibe ich es hier rein. Eine echt wichtige Klarstellung, die ich schon früher hätte machen sollen.

\subsection{Ist es okay, während der HRT Nikotin zu konsumieren?}\label{11-2}

Das hängt mit der Frage oben zusammen. \textbf{NNikotinkonsum während der HRT, vor allem wenn du Pillen nimmst, erhöht zusätzlich zu all den anderen Gründen, warum Nikotin nicht gut ist, das Risiko für Blutgerinnsel.} Das gilt für alle Arten von Nikotinkonsum, aber Rauchen ist natürlich mit Abstand das Schlimmste. Du willst echt kein Blutgerinnsel. Selbst wenn du keine Pille nimmst, stört Nikotin den Östrogenstoffwechsel und kann zu einer deutlichen Verringerung der Feminisierungseffekte führen. Dieser Aspekt ist noch nicht ausreichend erforscht, aber es gibt viele Erfahrungsberichte aus der Community. Es ist nicht einfach, aufzuhören, aber ich glaube an dich. Es gibt gute Hilfsmittel und Strategien wie die schrittweise Reduzierung mit Nikotinpflastern, die echt funktionieren. Du schaffst das.

Nur um das klar zu sagen: \textbf{Das heißt nicht, dass du kein Östrogen nehmen kannst oder solltest. Die Nachteile, wenn du gar kein Östrogen nimmst, sind viel schlimmer als keine wegen Nikotin zu nehmen.} Dieser Abschnitt soll dich nur auf die erhöhten Risiken und die möglicherweise langsamere Transition aufmerksam machen, um klar zu Empfehlen (und dich zu ermutigen), aufzuhören. Aber: Ein Schritt nach dem anderen.

\subsection{Ist es besser, mit einer niedrigen Dosis anzufangen oder mit einer hohen?}

Soweit ich weiß, nein. Sexhormone sind nicht wie andere Medikamente, bei denen man die Dosis anpassen muss, um Nebenwirkungen zu vermeiden, weil wir die Dosierungen kennen, die bei den meisten Leuten funktionieren. Deshalb finde ich persönlich “Anfangsdosen” und “Antiandrogen zuerst”-Behandlungen als medizinische Folter. Manche Leute denken, dass es am besten ist, den langsamen Verlauf der Pubertät nachzuahmen (auch wenn bei der viel mehr passiert als nur der Anstieg des Östrogenspiegels), aber dafür gibt's keine Beweise. Eine Orchiektomie am ersten Tag wäre vermutlich das Beste, aber wer macht das schon, sobald man versteht trans zu sein bzw. sich entscheidet, mit der Hormonersatztherapie anzufangen?

Anders gesagt: \textbf{Es gibt keinen Grund zu glauben, dass es für die Feminisierung besser ist, mit einer Dosis unter dem normalen Bereich “langsam anzufangen”.} Man muss sich keine Sorgen machen, dass man “zu schnell" vorgeht oder so. Sowohl Ärzte als auch andere Transfrauen erfinden anscheinend jeden Tag neue Mythen.

\subsection{Hat das Körpergewicht einen Einfluss auf die Dosierung?}

Nein. Weil es keinen “optimalen“ Blutspiegel für Östrogen gibt und weil der therapeutische Bereich akzeptabler Werte so breit ist, hat das Körpergewicht keinen nennenswerten Einfluss auf die Dosierung der Hormonersatztherapie. Aus dem gleichen Grund ist es unwahrscheinlich, dass geringfügige Abweichungen in der Dosierung Ihr Befinden beeinflussen. Es gibt kein „zu leicht“ oder „zu schwer“ für die HRT.

\textbf{Die Anpassung deiner Dosis in Schritten von 0,1 mg wäre ein Unterschied, den du wahrscheinlich nicht spüren könntest, weil unser Körper einfach nicht empfindlich genug für so differenzierte Wahrnehmung ist}, ganz zu schweigen von der hohen Wahrscheinlichkeit von Ungenauigkeiten bei der Injektion, die eine genaue Messung nahezu unmöglich macht. Mit anderen Worten: Eine solche Genauigkeit der Injektionen ist zu viel geforderte Präzision.

\subsection{Kann man mit Östrogen zu spät anfangen?}

\textbf{Nein.} Egal, wann du anfängst, Östrogen kann echt viel bewirken und mit der richtigen Behandlung kannst du super Ergebnisse erzielen. Sexualhormone gehören zu den stärksten Hormonen in unserem Körper, wenn es um unser Aussehen geht. Wir wünschen uns alle, das wir früher angefangen hätten, aber das ist kein Grund, nicht jetzt anzufangen. Selbst wenn du schon seit Jahren Östrogen nimmst, lohnt es sich trotzdem, die Qualität deiner Behandlung zu verbessern.

\subsection{Hört die Feminisierung/Brustentwicklung nach X Jahren auf?}

\textbf{Nein.} Es gibt keinen bestimmten Zeitpunkt, an dem Östrogen plötzlich nicht mehr wirkt. Es werden verschiedene Zahlen genannt, die meistens entweder 1) komplett erfunden sind oder 2) auf eine Studie verweisen, die nur X Jahre lang lief. Vor allem Ärzte sagen Transfrauen gerne, dass sie nicht mehr als Körbchengröße B erwarten sollen (was nicht mal so ist, wie Brustgrößen funktionieren, aber ich schweife ab) oder dass nach zwei Jahren kein Wachstum mehr zu erwarten ist, aber das stimmt einfach nicht. Es gibt Fälle, in denen Leute nach einer mehrjährigen Pause wieder mit Östrogen angefangen haben und trotzdem noch neues Wachstum hatten.

\subsection{Ich hab seit Jahren keine Veränderungen durch Injektionen bemerkt. Würde es was bringen, wieder auf Tabletten umzusteigen?}

Vielleicht, aber vielleicht auch nicht. Es gibt ein paar Anekdoten von Leuten, die von Injektionen wieder zu Pillen gewechselt haben (oder zusätzlich zu den Injektionen Pillen genommen haben) und nach einer „Stagnationsphase” ein stärkeres Brustwachstum festgestellt haben, aber der Mechanismus dahinter ist nicht klar. Es gibt Spekulationen, dass das E1:E2-Verhältnis (\textit{Estron} : \textit{Estradiol}) bei oralen Pillen im Vergleich zu E2 bei Injektionen stark in Richtung E1 verschoben ist, was bei manchen Menschen einen Unterschied machen könnte, obwohl \Textit{Estron} normalerweise nicht mit Feminisierung in Verbindung gebracht wird. Wahrscheinlich spielen noch andere Faktoren eine Rolle, aber du kannst gerne experimentieren, wenn du möchtest. Die Datenlage ist begrenzt.

\subsection{Ist es normal, dass man sich bei einer HRT schlapp fühlt und wenig Lust auf Sex hat?}

Im Allgemeinen: Nein. Wie sich die Libido äußert, ändert sich am Anfang, aber meistens, wenn jemand eine ungewöhnlich geringe Libido hat, liegt das daran, dass die Hormone nicht im Gleichgewicht sind. Das Gleiche gilt für Energielosigkeit. Bring deine Hormone in Ordnung und schau dir als Nächstes deine Ernährung/Vitamine an. Stell sicher, dass du nicht zufällig einen kritisch niedrigen Vitamin-D-Spiegel oder ähnliches hast. Das kommt öfter vor, als du denkst.

\subsection{Ich hab von [Mystery Medikament/einer Strategie] gehört, von dem eine befreundete Person sagt, dass es bei der Feminisierung hilft. Stimmt das wirklich?}

Vielleicht, aber wahrscheinlich nicht. Es gibt viele wilde Spekulationen darüber, wie man Feminisierungsziele erreichen kann, aber viele davon sind eher Schlangenöl oder haben sogar potenziell ernsthafte Risiken, die weit über die HRT selbst hinausgehen. Du hast das Recht auf körperliche Selbstbestimmung, und ich kann dich nicht davon abhalten, aber ich kann dich ermutigen, klug zu handeln. Je mehr du dich mit den biologischen Details der Geschlechtsangleichung beschäftigst, desto unsicherer wird der Boden, da immer weniger verlässliche Daten verfügbar sind. Verzweiflung kann zu vielen unklugen und gefährlichen Entscheidungen führen. Sei also klug und gehe auf Nummer sicher.

\subsection{Sollten wir den Östrogenzyklus von cis Frauen nachahmen?}\label{11-10}

Wahrscheinlich nicht. Das ist zwar umstritten, aber ich denke, weil wir (naja, die meisten von uns) keine Gebärmutter haben und keinen Menstruationszyklus, der mit unseren Hormonen zusammenhängt, also gibt's keinen Grund, warum wir versuchen sollten, dieses Verhalten nachzuahmen. Meiner Meinung nach ist das ein Problem \textit{zwischen dem, was ist, und dem, was sein sollte}. Das größte hormonelle Problem für die meisten Transfrauen ist die Testosteronunterdrückung, die einen konstant hohen Spiegel erfordert (außer nach einer GaOP, nach der kein Testosteron mehr zu unterdrücken ist), sodass starke Schwankungen und/oder relativ niedrige Spiegel wahrscheinlich unnötiges Leid verursachen. YDu kannst natürlich gerne experimentieren. Vor allem, wenn die Unterdrückung von Testosteron für dich kein Problem mehr ist. Siehe Frage \ref{8-11} und folgend.

\subsection{Haben Transfrauen ihre Periode?}

Ähnlich wie bei der letzten Frage ist es wichtig zu verstehen, was da eigentlich passiert. Die einzigartige Hormonschwankung, die durch deinen speziellen Ester, deine Dosierung und deine Häufigkeit entsteht, kann zu Stimmungsschwankungen führen, wenn dein Östrogenspiegel zwischen den Injektionen schwankt. Manche Transfrauen vergleichen dieses Phänomen mit einer Periode, aber die Ursache für diese körperlichen Veränderungen ist eine andere und meistens ein Zeichen dafür, dass deine Behandlung angepasst werden muss, damit du dich so gut wie möglich fühlst, denn Leiden ist nicht unbedingt weiblich. Schmerzen und Unwohlsein sind keine Voraussetzungen für Weiblichkeit, und wir sollten uns nicht auf bioessentialistische Argumente stützen. Eine Ausnahme bilden hier intersexuelle Transfrauen, die eine Gebärmutter haben und tatsächlich ihre Periode bekommen. In diesem Fall: Ja, klar. Siehe Frage \ref{11-35}.

\subsection{Kann zu viel Östrogen in Testosteron umgewandelt werden?}

\textbf{Nein.} Aromatase ist das Enzym, das Testosteron in Östrogen umwandelt, aber es gibt keinen Mechanismus, der Östrogen in Testosteron umwandelt. Das kann nicht passieren. Das ist ein total falscher Mythos, und du solltest sofort misstrauisch werden, wenn jemand so was behauptet. Leider sind es gerade Ärzte, die diesen Mythos am häufigsten verbreiten.

\subsection{Führt eine GaOP zu einem Anstieg des Testosteronspiegels?}

Nein. Das stimmt nicht. Es gibt keinen magischen Mechanismus, der plötzlich dafür sorgt, dass der Testosteronspiegel steigt, sobald die Hoden entfernt werden. Selbst wenn die Eier magische Kräfte hätten, funktioniert die Hormonproduktion einfach nicht so. “Also, deine Nebennieren…” So funktioniert das auch nicht. Die einzige mögliche Ausnahme wären nicht diagnostizierte Fälle von Nebennierenhyperandrogenismus, (A.d.Ü. sag das drei mal schnell!) die vor der Operation mit einem Antiandrogen wie \textit{Spironolacton} behandelt wurden und nach Absetzen des Antiandrogens wieder auftreten könnten. Bitte hör auf, diesen Mythos zu verbreiten.

\subsection{Wie kann ich Haarausfall verhindern/rückgängig machen?}\label{11-14}

Mechanisch gesehen ist das ziemlich einfach. Eine Standard-HRT-Therapie allein ist in dieser Hinsicht schon fast wie Magie (frag nicht, wo die Magie steckt), aber die Einbeziehung von 5$\alpha$-Reduktasehemmer (5-ARI) wie in Abschnitt \ref{AA} “ANTIANDROGENE” besprochen wird in krassen Fällen empfohlen, um den Verlust komplett zu stoppen. Topisches Minoxidil 5\% ist das Einzige, was neben Hormonen allein hilft, deinen Haaransatz zu festigen. Aber denk dran, dass du außer in Ausnahmefällen nur sterbende/ruhende Haarfollikel retten kannst. Tote Follikel kommen nicht zurück.

Wenn dir das allein nicht reicht, solltest du wissen, dass sich die Haartransplantationstechnologie stark verbessert hat. Das FUE-Verfahren (Follicular Unit Extraction) ist genau das, was du dir anschauen solltest. Hier werde ich in Zukunft einen Leitfaden verlinken, den eine Expertin zum Thema Versicherungsschutz für diese Behandlung geschrieben hat, sobald er fertig ist. Das ist Gruppenzwang. Schau hier regelmäßig vorbei.

\subsection{Beeinflusst Sport die Feminisierung?}

Wahrscheinlich. Die Hormonersatztherapie verändert langsam deinen Körper, also kannst du mit Sport dazu beitragen, dass sich dein Körper verändert. Denk dran, dass dieser Prozess SEHR LANGSAM ist, also ist es wichtig, dass du genug isst, um die nötige Energie zu haben und geduldig zu bleiben. Die Wachstumshormone, die durch die Muskelstimulation beim Krafttraining freigesetzt werden, spielen auch eine Rolle bei der Brustentwicklung, also ist das wahrscheinlich eine gute Sache, abgesehen von den anderen offensichtlichen gesundheitlichen Vorteilen von Sport.

Das ist NICHT nur der kaum verhüllte Fetisch der Autorin; Krafttraining ist wichtig für deine Gesundheit! Ich erwähne das, weil viele Transfrauen denken, dass sie wie der Hulk aussehen werden, wenn sie eine Hantel auch nur anfassen. Ich verstehe das, aber wenn du kein Testosteron hast und keine Steroide nimmst, wirst du nicht so aussehen. Ganz zu schweigen von der Zeit, Mühe und Disziplin, die nötig sind, um auch nur annähernd so auszusehen.

\subsection{Was soll ich denn trainieren?}\label{11-16}

Cardio ist wichtig fürs Leben, das ist echt wichtig. Übungen für den Unterkörper machen deine Hüften und Gesäßmuskeln fester und betonen deine Figur. Übungen für den Oberkörper verbessern deine Haltung und stützen deine Brüste, sodass sie größer aussehen. Mit anderen Worten: alles. Du nimmst Östrogen. Hast du schon mal cis Sportlerinnen gesehen? Sport macht dich weiblicher.

\href{https://docs.google.com/document/d/1-NyE5EY5TTaRRMhk7HlTbKJ7HifjEsA4jlDO1qKQVl0/edit?tab=t.0}{Dieser Leitfaden wurde mir gelinkt} \textcolor{red}{(Warnung: Google Docs)} (A.d.Ü. auf Englisch!) und scheint ein guter Ausgangspunkt zu sein. Ich möchte darauf hinweisen, dass es keine speziellen Übungen gibt, die den Körper feminisieren oder maskulinisieren, da der Körper so nicht funktioniert. Du solltest dich jedoch vielleicht mehr auf Übungen für den Unterkörper und auf Flexibilität konzentrieren als typische Kraftsportler.

\subsection{Kann Östrogen wirklich dazu führen, dass man kleiner wird?}

Ja. Es könnte sein, dass es mit Veränderungen des Wassergehalts in Sehnen und Bändern zusammenhängt, aber das wurde noch nicht untersucht, also ist die Ursache reine Spekulation. Wissenschaftler: eine coole Idee für eine Studie!

\subsection{Kann Östrogen wirklich dazu führen, dass die Füße schrumpfen?}

Ja. Siehe oben.

\subsection{Kann Östrogen wirklich andere Arten von Schrumpfung verursachen?}

Naja, wie man so schön sagt: „Use it or lose it“ – nutze es oder verliere es.

\subsection*{Sexuelle Gesundheit}
\addcontentsline{toc}{subsection}{\textemdash{} Sexuelle Gesundheit}

\subsection{Wie kann ich bei einer Hormonersatztherapie meine Erektionsfähigkeit verbessern?}\label{11-20}

Neben der regelmäßigen Nutzung gibt's noch andere Möglichkeiten, die Erektionsfähigkeit zu verbessern: 1) Verbessere deine Fitness und körperliche Gesundheit, vor allem deine Herz-Kreislauf-Fähigkeiten; 2) denk über Medikamente wie \textit{Tadalafil} oder \textit{Sildenafil} nach; und 3) überleg dir, ob du Testosteron-Ergänzungsmittel nehmen willst (siehe Abschnitt \ref{T} „TESTOSTERON”).

Wenn du mehr darüber wissen willst, wie die Erektion funktioniert, findest du in \href{https://stainedglasswoman.substack.com/p/how-to-maintain-your-penis-function}{diesem (A.d.Ü. englischen) Substack-Artikel} einen guten Überblick über das Thema.

\subsection{Wie kann während meiner HRT die Menge an Sperma/Vorflüssigkeit erhöhen?}

Mach dir keine Sorgen, das ist eine ganz normale Frage. Sonnenblumenlecithin und Pygeum können da helfen. Es scheint auch bei der vaginalen Feuchtigkeit und Erregung bei Frauen, die eine Geschlechtsangleichung hatten, einen Unterschied zu machen, aber es gibt noch nicht genug Daten und Erfahrungsberichte, um das sicher zu sagen. Ansonsten solltest du einfach genug Wasser trinken und auf eine ausgewogene Ernährung achten.

\subsection{Kann ich durch die HRT laktieren?}

Ja. Domperidon, Bockshornklee, Sonnenblumenlecithin, reichlich Östrogen und reichlich Progesteron. Hol dir eine Pumpe. Mach dich fertig.

Man sollte wissen, dass Domperidon Nebenwirkungen und Risiken hat und dass die Fähigkeit zu laktieren keinen Einfluss auf die Entwicklung der Brust hat. Die Newman-Goldfarb-Protokolle sind das, was du dir mal anschauen solltest.

\subsection{Kann die HRT deine Sinne und deine Wahrnehmung, z. B. den Geruchssinn, verändern?}

Du hast vermutlich die Jahre vor Beginn der Hormonersatztherapie unter Dissoziation und Depressionen gelitten. Die Welt ist jetzt lebendiger, weil du nicht mehr rund um die Uhr dissoziierst. Die Wunder der modernen Medizin!

Es kann aber deine Sehstärke direkt verändern. Das kann auf jeden Fall passieren.

\subsection{Kann die HRT meine Sexualität verändern?}

Ähnlich wie bei der oben beschriebenen Dissoziation führt die Hormonersatztherapie oft zu mehr Offenheit und Akzeptanz gegenüber sich selbst, was eine Veränderung in der Ausprägung der eigenen Sexualität bewirken kann. Ob es sich dabei um einen chemischen oder einen verhaltensbedingten Effekt handelt, ist weitgehend eine Frage der Semantik. Eine Frage der Perspektive.

\subsection{Sollte ich eine PrEP machen?}

\textbf{Ja.}

\subsection*{Behandlungsfehler}
\addcontentsline{toc}{subsection}{\textemdash{} Behandlungsfehler}

\subsection{Ich hab gehört, dass Injektionen eigentlich weniger stabil sind, weil man sie seltener macht. Stimmt das?}

Nur wenn du dich an die bescheuerten WPATH SOC 8-Richtlinien hältst, die eine empfohlene Dosierung von \textit{Estradiol Valerate} oder \textit{Estradiol Cypionate} im Bereich von 5-30 mg alle zwei Wochen vorschreiben, was du, um es ganz klar zu sagen, auf keinen Fall tun solltest. „Keinen Schaden anrichten“, my ass.

\subsection{Aber mein Arzt meinte doch...?}

Der durchschnittliche Arzt hat im Grunde keine Ausbildung in Sachen Trans-Gesundheitsversorgung, und \href{https://www.endocrine.org/news-and-advocacy/news-room/2017/endocrinologists-want-training-in-transgender-care }{4 von 5 Endokrinologen haben auch nie eine richtige Ausbildung in Trans-Gesundheitsversorgung gemacht}. (A.d.Ü. Die Statistiken lassen sich natürlich nicht direkt auf Deutschland übertragen aber viel besser sieht es hier auch nicht aus.) Es ist ziemlich wahrscheinlich, dass du ihr erster Transpatient bist und dass es keine vorherigen Erfahrungen mit den praktischen Aspekten der Behandlung von Transpatienten haben. Selbst Ärzte, die sich sehr engagieren, sind oft durch konservative Behandlungsstandards eingeschränkt, die sie befolgen müssen und die nicht immer mit der für dich besten Behandlung übereinstimmen. Siehe oben.

Pass auch auf das "Trans-Armbruch-Syndrom“ auf, also die Tendenz von Ärzten, alles auf die Hormonersatztherapie zu schieben. Wenn dein Arm gebrochen ist, liegt das wahrscheinlich nicht "an diesen Hormonen"

Ich sollte das eigentlich als separate Frage stellen, aber ich will das Layout nicht durcheinanderbringen: Es gibt's keine Situation, in der es okay ist, dass ein Arzt deine Brüste anschauen oder anfassen will, um „das Wachstum zu beobachten“ oder aus irgendeinem anderen Grund. Zum Glück kommt das heutzutage viel seltener vor, aber es ist sexuelle Belästigung und total inakzeptabel.

\subsection{Mein Arzt will mir keine Injektionen verschreiben. Was soll ich machen?}

Attempt to convince them, replace them, or seek DIY sources. Do not let a gatekeeping medical establishment prevent you from receiving the appropriate care that you deserve. \textbf{The most crucial aspect of interfacing with the medical system while trans is that you have to advocate for yourself. }This is compounded with disability, ethnicity, and other afflictions that scare doctors like womanhood.

Versuch, die Ärtzy zu überzeugen, zu einer anderen Ärzty-Person zu wechseln oder such nach DIY-Quellen. Lass dich nicht von einer medizinischen Einrichtung, die als Torwächter fungiert, davon abhalten, die angemessene Versorgung zu bekommen, die du verdienst. \textbf{Der wichtigste Aspekt bei der Interaktion mit dem medizinischen System als Transperson ist, dass du dich für dich selbst einsetzen musst. } Wenn dazu noch Behinderungen, ethnische Zugehörigkeit und andere Probleme kommen, die auf Ärzte ebenso abschreckend wie Weiblichkeit wirken. (A.d.Ü. Der Sarkasmus ist in der englischen Version viel deutlicher - aber seid versichert: Dies ist sarkastisch gemeint.)

\subsection{Wie verhält sich die HRT für cis Frauen in den Wechseljahren zur HRT für Transfrauen?}\label{11-29}

Auch wenn wir meistens unterschiedliche Ziele haben und vor allem ganz unterschiedliche Dosierungsanforderungen, gibt es doch viele Gemeinsamkeiten in den Erfahrungen von Transfrauen und cis-Frauen in den Wechseljahren. Medizinische Frauenfeindlichkeit in Form von Inkompetenz, Abweisungen, Feindseligkeit und/oder Fehlinformationen ist leider etwas, das wir beide erleben. Deshalb ist es super wichtig, Solidarität in dieser Frage aufzubauen. Um zu zeigen, was ich meine: \href{https://www.youtube.com/watch?v=W0XW6av2wLQ}{die ersten 30–40 Minuten dieses Interviews} werden dir wahrscheinlich sehr bekannt vorkommen, und sind nützlich wenn du deinen Blutdruck in die Höhe treiben möchtest. Die Interviewte selbst weist auch auf diesen Zusammenhang hin! Die WHI hat das Leben unzähliger Frauen ruiniert.

\subsection*{Intersexualität und Begleiterkrankungen}
\addcontentsline{toc}{subsection}{\textemdash{} Intersexualität und Begleiterkrankungen}

\subsection{Was ist denn mit dem Ehlers-Danlos-Syndrom los?}

Diese Bindegewebsstörung hat eigentlich nichts mit der Hormonersatztherapie zu tun, aber viele Transmenschen haben sie. Also, herzlichen Glückwunsch, falls du dadurch erfahren hast, dass du auch davon betroffen bist. Abgesehen von den allgemeinen langfristigen Herz-Kreislauf-Problemen, die du vielleicht im Auge behalten solltest, mach weiter mit Krafttraining, damit deine Gelenke funktionieren. Informiere dich dazu aber lieber woanders. Siehe Frage \ref{11-16}.

\subsection{Was sollte ich bei Intersexualität beachten?}

In diesem Leitfaden habe ich Intersexualität nur am Rande erwähnt. Hier ist eine kurze Liste mit Sachen, die du auf deinen Reisen für dich selbst oder für einen Freund wissen solltest.

\subsection{Was ist eigentlich das Klinefelter-Syndrom?}

Das ist eine relativ häufige (wenn man Chromosomenmutationen überhaupt dieses Wort nutzen will) intersexuelle Erkrankung, von der manche Transfrauen vielleicht nicht wissen, dass sie sie haben, weil sich die beiden Zustände überschneiden können. Es zeigt sich normalerweise durch einen niedrigen Testosteronspiegel zu Beginn der (ersten) Pubertät. Es ist gut, den Namen zu kennen, nur für den Fall.

\subsection{Was ist das Müller-Gang-Persistenzsyndrom (PMDS)?}

Noch eine “Ich schreib das mal mit hier rein, weil du vielleicht zum ersten Mal davon hörst” intersexuelle Erkrankung, die manche Transfrauen betreffen kann, auch wenn wir nicht wissen wie wenige es tatsächlich sind, weil wir keine Zahlen haben. Die Möglichkeit einer unterentwickelten Gebärmutter, kann das zu Komplikationen und Besonderheiten führen. Du solltest wahrscheinlich extra Progesteron nehmen, um das Risiko für Gebärmutterkrebs zu vermeiden.

\subsection{Was ist denn mit dem ovotestikulären Syndrom los?}

This intersex condition in particular can cause early level fluctuations which made lead to confusing test results due to the presence of both ovarian and testicular tissues, either separate or combined in an \textit{ovotestis}. This presents in many different ways which HRT can interact with as you begin suppressing \textit{luteinizing hormone} (LH). A uterus may or may not be present, multiple sets of gonads could be present, and/or it could look outwardly typical.

\subsection{What’s the difference between intestinal cramps and uterine cramps?}\label{11-35}

These are commonly misattributed in early transition as a symptom of intersex conditions. Intestinal cramps are widespread and diffuse across your abdomen, whereas uterine cramps are highly concentrated in a location somewhere below your belly button and tend to be sharp stabs/contractions in rapid succession. Like the inside of your body is used as a stress ball. Very different!

\subsection{What about other intersex conditions?}

I have listed a few notable ones, but there are far more expressions and ways of testing them that go far beyond the scope of this guide. Anecdotally, prevalence is higher than average among trans people so basic familiarity with this is useful.

\subsection*{Oddball Questions}
\addcontentsline{toc}{subsection}{\textemdash{} Oddball Questions}

\subsection{Many DIY sources only take crypto. Is that required? How does that work?}

There are other guides that cover this in better depth than I can on how to use crypto safely, including some vendors who have their own guides. But yes, crypto is often required for a lot of reasons. “Crypto” means a lot of things, but using it as a currency was the original point after all. It’s mostly just a pain in the ass. Monero (XMR) is good.

\subsection{What about Selective Estrogen Receptor Modulator (SERM) drugs for nonbinary regimens?}

Some people use SERMs as a part of a transition that is not looking to feminize as much for a more androgynous look, but it’s pretty much entirely uncharted waters thus why their mention is otherwise absent from this guide. You’re on your own if that’s something you want to explore, so please be safe. I don’t personally rate them very highly as I have not seen much to suggest that they work well for how people usually think or want them to work, at least not without a lot more caveats, but obviously there are people who like them. It's just not something I feel comfortable giving recommendations for.

The various proposed nonbinary regimens are often highly individualized because they are specific to what a persons' particular goals are. All HRT should be individualized to a degree, but there is often more variation in desired outcomes when people ask about androgyny. Hormonally, it is nontrivial. Everything stated in this guide should be treated solely as a starting place if you are wanting to experiment with something more complicated, but do remember that there is much more to achieving transition goals than just hormones alone.

\subsection{Are things like “herbal HRT” or “phytoestrogens” legitimate?}

\textbf{No.} If someone is telling you they have “herbal HRT”, they are telling you they have snake oil. The only thing that is going to feminize you is estrogen, not plant estrogens. No amount of “natural” products are a replacement for estrogen itself. This isn’t a common scam and you probably already know, but just in case you run into it, now you know for sure. If it smells like bullshit, it’s probably bullshit. Unless we’re talking about bug steroids in which case yeah those are actually cool. Won’t feminize you though.

\subsection{Is the Reddit Doctor that people constantly talk about Good?}

No.

\subsection{I hear DIY estrogen is made in a bathtub. Is that true?}

No. I honestly have no idea where or why this joke started that people now take seriously, but there’s no step in any process where a bathtub would even be considered. Don’t believe everything you read online. I don’t even know what you could even theoretically do with a bathtub, unless you think estrogen vials are full of the bathwater of trans women. I don’t know why you would think that though. It’s obviously cum.

\subsection{How does HRT affect fertility?}\label{11-42}

It is important to understand that this is extremely understudied so exact figures cannot be stated, and given the seriousness of pregnancy, I urge you to practice safe sex and lean on the side of caution where possible. HRT itself can, and likely will, make you infertile eventually, but only through full suppression of the HPG axis (See Question \ref{2-3}) over a long time span. In other words, if you haven't had bottom surgery of any kind and you are on an HRT regimen that is less capable of HPG axis suppression (such as pills), then this is more of a consideration.

\textbf{If the HPG axis is not suppressed then it is fully possible to impregnate someone}, and the timeline for sperm maturation is long enough that this is true even after the HPG axis has been initially suppressed for \textbf{multiple months}. Please take this very seriously. Full HPG axis suppression for at minimum six months, perhaps closer to a year out of an abundance of caution, is recommended.

\subsection{Is infertility from HRT reversible?}\label{11-43}

It is theoretically possible to reverse HRT-induced infertility, assuming you weren't already infertile prior to HRT (a large assumption!), but there are not many documented cases of this so the full efficacy of fertility restoration after long-term HRT is unknown. The process would involve restarting the HPG axis with a variety of medications along with entirely stopping HRT, which would in essence require a hormonal detransition for likely six months at minimum, and even then sperm quality is not certain or guaranteed. It is not something that should be planned for, to say the least, so planning around it would be wise. A sperm bank would be recommended before or early in HRT, financially permitting, if potential biological children are a priority and if a future relationship where that is possible/desired is likely.



\section{CREATINE}

\subsection{What is creatine?}

Creatine is an organic compound in your muscles and in your brain. It recycles ADP into ATP which is important for energy production in your body, especially initial high burst applications before other energy systems take over.

\subsection{Isn’t it like a steroid or something that bodybuilders use?}

No. Bodybuilders and athletes like it because having more energy means more activity before getting tired. They aren’t the only ones who use it since it is basically the \#1 supplement in terms of things that are actually useful and are actually researched. 

\subsection{How is creatine related to HRT?}

It isn’t! But it’s something I yell about because I think it’s good and I am tired of repeating myself because people keep asking and you’re reading this anyway, aren’t you? I love a captive audience. My standup routine is at the bottom.

\subsection{Okay well why should I take creatine then?}

What a great question! It’s good for your brain and your muscles. Creatine is often found in relatively low concentrations for many people depending on their diet, especially people who don’t eat meat. There is compelling research about various chronic fatigue and post-viral conditions (long COVID in particular) being related to depleted creatine reserves in the brain, so some people find cognitive benefits from supplementing it. It isn’t magic but it is dirt cheap so it is worth trying in my opinion.

\subsection{What are the forms?}

Just \textit{creatine monohydrate} powder is what you want. The pills tend to be low dosage and are up charging you anyway, while gummies often destroy the creatine in the creation of the gummy. A lot of brands include creatine in some sort of mix but the pure stuff is usually cheaper.

\subsection{How do I take it then?}

The general recommendation is 5-10g daily dissolved in some sort of liquid. It dissolves best in things that aren’t just water. It’s mostly flavorless, so just throw a scoop or two in your coffee or a smoothie and call it a day. It can be a little chalky or gritty depending on the quantity and the fluid.

\subsection{Does it matter when I take it?}

Not really. It doesn’t have an immediate effect like that which is why it’s silly that it’s microdosed in pre-workout mixes. Take it whenever it’s convenient for you.

\subsection{How does it work then?}

It builds up in your body to a maximum level of saturation over a week or two. Then you just maintain that and reap the rewards (of maybe feeling better).

\subsection{Do I have to do a “loading” phase of taking more at first?}

Probably not. Unless you’re on some sort of intense training time crunch or something, this probably doesn’t matter at all. Just take whatever is convenient with some regularity.

\subsection{What are the side effects?}

Slight weight gain may be possible because of increased water weight in your muscles (which to be clear is Good, so don't be alarmed). If you don’t take it with water, or if you take too much at once, you might get a tummy ache. Ouchie.

\subsection{Who shouldn’t take it?}

People with kidney issues. Not because it causes them, but because creatinine (Different spelling! Creatine becomes creatinine) is used as a marker in lab tests for a number of kidney issues and supplementing might give a false positive. Just keep it in mind.

\subsection{Do you have any brand recommendations?}

No. It shouldn’t really matter. Just get whatever seems reputable and is a reasonable price. I’d give a recommendation for the one I like but when I asked the brand for affiliate link they turned me down, so their loss! No free clout.

\subsection{You seriously put creatine into this document, huh?}

Yeah it’s pretty funny. It’s not my fault that I joked about it and people told me it legitimately helped them because now I feel obligated to keep talking about it!!!

 

\section{CLOSING REMARKS}

If any of the following are true:

\begin{itemize}
\item you are still mad at me despite the disclaimer;

\item you spotted an issue or typo;

\item you have a clarifying question that should be put into the text;

\item you have an objection that hopefully isn’t an Uhm Ackshually;

\item you wish to sing my praises;

\item you wish to pledge fealty; 

\item you wish to send tithes my way;
\end{itemize}

Then please feel free to contact me and I’ll see what we can do. Bluesky is the easiest contact point, and you can DM me for my Signal. Otherwise, thank you for reading and I hope it helps.

\textbf{If you would like to donate to support this project,} \href{https://cash.app/Katitties}{CashApp}, \href{https://ko-fi.com/katitties}{Ko-Fi}, and \href{https://account.venmo.com/u/katitties}{Venmo} all work. I appreciate it!

And lastly: \textbf{The most important thing that you can do as a trans person is to live.} For as much as this document is a manual, it is in equal measure a message to you as a trans person that your existence is a gift upon the world, your presence is a blessing on those around you, and that you deserve to be treated with respect. Even if you do nothing else, your life is a feat worth praising. Thank you.



\section*{FRIENDS OF PGHRT}\label{FOPGHRT}
\addcontentsline{toc}{section}{FRIENDS OF PGHRT}

Across this document is a scattering of links to other guides and resources. Below is a consolidation of them which will also include more links to external resources as time goes on, ideally by other trans people. For the privacy minded or noided, note that some of these are Google Docs links.

\begin{enumerate}
  \item \href{https://startwith4mgestradiolenanthateweeklyandtestatonetothreemonths.com/}{SW4EEWATAOTTM} - TL;DR for PGHRT
  \item \href{https://hrtcafe.net/}{HRT Cafe} - HRT Resource Aggregator
  \item \href{https://transfemscience.org/}{Transfeminine Science} - Informational resource for trans medical literature
  \item \href{http://estrannai.se}{Estrannai.se} - Estradiol Pharmacokinetics Playground
  \item \href{https://globoho.moe/}{Globoho.moe} - Thailand Orchiectomy Medical Tourism Travel Guide 
  \item Julia's FUE Guide - COMING SOON, I'M BULLYING HER TO WRITE FASTER
  \item \href{https://docs.google.com/document/d/1-NyE5EY5TTaRRMhk7HlTbKJ7HifjEsA4jlDO1qKQVl0/edit?tab=t.0}{Sky's Feminine Figure Beginner Program} - An exercise regimen geared towards trans fems
  \item \href{https://docs.google.com/document/d/114sztSw1aVWM2pXLDl9NrHklyvewz3EmFiHiisjM71k/edit?tab=t.0}{Sky's Diet 101} - A guide towards adjusting weight in a healthy way
  \item \href{https://stainedglasswoman.substack.com/p/how-to-maintain-your-penis-function}{How to Maintain Erectile Function on HRT} - A longer form explanation on the "use it or lose it" phenomenon
  \item \href{hhttps://docs.google.com/document/d/1DXFxzN0XTudPZez_SO61fpqncRLPH_Be_QG_8Pcz9LU/edit?pli=1&tab=t.0}{Biohax Guide Googleslop Edition} - Trans Masc DIY Guide
\end{enumerate}

\section*{ABOUT THE AUTHOR}
\addcontentsline{toc}{section}{ABOUT THE AUTHOR}

Katie Tightpussy is an award-winning author and professional trans woman with nearly a decade of experience in the field of transgender. Her accomplishments include transiferating her sex through the novel technique of cross-sex hormone injections, being physically unable to shut up, and utilizing a very fortunate set of hyperfixations as they relate to transbobulation of the humors. She spends her days in the idyllic rural countryside of Los Angeles scheming of new ways to achieve world domination and enjoys riding her bicycle. Media inquiries can reach her agent at \href{http://katietightpussy.com}{katietightpussy.com}.

 

\section*{DISCLOSURES}
\addcontentsline{toc}{section}{DISCLOSURES}

No robot girls were harmed in the making of this document, including any usage of generative large language models. The author does not endorse any reproduction without attribution nor scraping of this work. Leave those poor robot girls alone.

The author declares an attraction towards women and acknowledges a potential conflict of interest for the existence of more beautiful trans women in the world.

 

\section*{ACKNOWLEDGEMENTS}
\addcontentsline{toc}{section}{ACKNOWLEDGEMENTS}

Though the text is primarily my voice, this document would not be even half as good without the contributions, feedback, and suggestions from others involved at every step along the way. A good reminder as ever that transition is not something best done alone.

Many thanks to Q, R, RM, and S in alphabetical order for close review and generally being fun nerds to talk to; love y’all. Special thanks to CB and J for close review that also inspired some very good bits. Thanks to KG for additional intersex information. Thanks to w [sic] for additional injection resources. Thanks to BIR collectively for a plethora of crucial nerd nitpicks. Appreciation for general review from C, JTP, K, S, and V. Thanks to everyone on Bluesky who encouraged me to write this up in the first place, and everyone over the years sharing knowledge. And of course: much appreciation to all HRT nerds, even when we disagree, since we’re all trying to do the best for our community where we’ve otherwise been let down. Keep up the good work everyone. 

Shout out to my IB Chemistry HL teacher many years ago who quite reasonably doubted my studiousness even though I’m now putting much of that knowledge to use for the art of transsexuality; go figure. 

 

\section*{CHANGELOG}
\addcontentsline{toc}{section}{CHANGELOG}

\noindent \href{https://github.com/Juicysteak117/pghrt/}{Source code available here on GitHub.}

\noindent Full Compilation Datetime: \DTMnow

\noindent(There aren't LaTeXML bindings for \texttt{datetime2}, \texttt{hanging}, or \texttt{hyphenat}, so the formatting is slightly ugly. If you'd really like to help me out, please write those bindings!!!)

\noindent 2025-08-20: Initial release. 15.9k words.

\noindent 2025-08-20: A lot of typos and minor verbiage tweaks. Added Question \ref{8-18}.

\noindent 2025-08-21: Typos grow on trees. Added Question \ref{5-27}.

\noindent 2025-08-21: More tweaks. Opted to remove “WHY PROG” from Question \ref{8-17}. 17.0k words.

\noindent 2025-08-22: Nitpicks, clarifications, and typos. 17.2k words.

\noindent 2025-08-24: A few more twinks sorry tweaks. 17.3k words.

\noindent 2025-08-27: How long until remaining typos are embarrasing? 17.3k words.

\noindent 2025-08-28: Reduced ambiguity in a few areas. 17.4k words.

\noindent 2025-08-29: Additional clarity for frequencies in Section \ref{td}. 17.5k words.

\noindent 2025-09-01: Sisyphus boulder meme captioned fixing typos dot png. 17.5k words.

\noindent 2025-09-07: Added donation links per request. That's very kind. 17.5k words.

\noindent 2025-09-07: Few more tweaks. Clarified an additional common progestin. 17.6k words.

\noindent 2025-09-19: Added Question \ref{4-16} plus tweaks. 17.7k words.

\noindent 2025-09-23: A wide variety of clarifications up and down the line. 18.1k words.

\noindent 2025-09-24: Added an important note about surgery to Question \ref{11-1}. 18.3k words.

\noindent 2025-09-24: “Katie my doctor told me-” It never ends. 18.5k words.

\noindent 2025-09-26: Another pass of clarification edits. Yes I should have a git diff. Sorry that I don't. I thought I'd be done by now anyway! 18.7k words.

\noindent 2025-09-30: Added some cross references for clarity. 18.7k words.

\noindent 2025-10-02: More cross references. Likely will do another pass. 18.8k words.

\noindent 2025-10-02: Added a big bold warning about recapping to Question \ref{5-13} because SOMEONE didn't watch the video smh. 18.9k words.

\noindent 2025-10-10: Added Question \ref{11-42} and Question \ref{11-43} per request. Honestly I just forgot about fertility being a thing lol. Also added the Friends Of PGHRT postword section. 19.5k words.

\noindent 2025-10-10: Added Gretchen's Version (.txt) and fixed formatting. 19.5k words.

\noindent 2025-10-11: Added permalinks to everything, yay! And finally made a git repo. Look at me being a big girl, wow. 19.5k words.

\noindent 2025-10-11: Added an external link to Question \ref{11-20}. 19.6k words.

\end{document}