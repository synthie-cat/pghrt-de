\documentclass{article}
\usepackage{hyperref}
\usepackage{float}
\usepackage{csquotes}
\usepackage[style=iso]{datetime2}
\usepackage[usenames,dvipsnames]{color}
\usepackage{booktabs}
  \setlength\heavyrulewidth{0.20ex}
  \setlength\cmidrulewidth{0.10ex}
  \setlength\lightrulewidth{0.10ex}

\usepackage[font=normalsize,labelfont={bf}]{caption}
  \captionsetup[table]{aboveskip=3pt}

\hypersetup{
    colorlinks=true,
    linkcolor=blue,
    filecolor=magenta,      
    urlcolor=magenta,
 }
 
\usepackage{graphicx}
\graphicspath{ {./img/} }
\renewcommand{\abstractname}{DISCLAIMER}
\title{EINE PRAKTISCHE ANLEITUNG ZUR FEMINISIERENDEN HORMONTHERAPIE}
\author{\href{https://katea.gay/}{Katie Tightpussy}}
\date{\today}
\setcounter{section}{-1}
\urlstyle{same}

\begin{document}


\maketitle
\tableofcontents
\section{VORWORT ZUR ÜBERSETZUNG}

Im deutschsprachigen Raum war bisher keine umfassende, aktuelle, verständlich geschriebene Informationssammlung zur feminisierenden Hormontherapie verfügbar. Diese Übersetzung des hervorragenden Werks von Katie Tightpussy ist ein Versuch, diesen Mangel entgegenzuwirken. Nicht alle sind in der Lage, englischsprachige Anleitungen und Übersichten zu verstehen. Die KI-gestützte Übersetzung, trotz ihrer Vorzüge hinsichtlich der leichten Handhabung und Schnelligkeit, ist für diese Aufgabe ebenfalls ungeeignet, da viele der Nuancen und Feinheiten des Textinhalts dabei verloren gehen. Diese in liebevoller Handarbeit entstandene Übersetzung soll es allen deutschsprachigen Transpersonen (und weiteren interessierten Menschen) ermöglichen, sich selbst zu bilden und informierte Entscheidungen treffen zu können. 


Im Text wird, wenn notwendig, durch die Abkürzung "A.d.Ü." (Anmerkung der Übersetzerin) auf besondere Entscheidungen bei der Übersetzung oder auf Unklarheiten verwiesen.


Für etwaige Fehler entschuldigt sich die Übersetzerin im Voraus. Kritik und Lobgesänge aller Art werden auf Bluesky bei @camille.gay dankend angenommen.

Virgo Sine Ruga, a.k.a. Camille, Oktober 2025

\begin{abstract}
    Ich bin keine Ärztin. Ich arbeite nicht im medizinischen Bereich. Ich bin in keiner Weise medizinisch ausgebildet. Ich bin eine Nichtfachfrau, die nichtfachmännische Meinungen anbietet, auf der Basis und im Umfang meiner eigenen Erfahrungen. Alle darauffolgenden Informationen und Behauptungen sollten dementsprechend als bloße Meinungsäußerungen verstanden werden, nicht als Fakten oder medizinische Ratschläge. Diese Anleitung zieht die in der Community erarbeitete moralische Wahrheit vor, wenn die Wissenschaft noch nicht soweit ist. Sei bitte nicht böse auf mich. 
\end{abstract}


\section{VORWORT}

Der Zweck dieses Dokuments ist die Katalogisierung meiner Gedanken und Meinungen zur feminisierenden Hormontherapie (A.d.Ü.: zur besseren Übersichtlichkeit ab jetzt mit "HRT" abgekürzt), da die verschiedenen Wikis der Community meiner Meinung nach ungeeignet sind. Diese sind zwar wertvolle Ressourcen, aus meiner Sicht jedoch für Personen die eine klare, handlungsorientierte Anleitung suchen, statt umfassende Information zu jedem biologischen Vorgang und Diagramm, wenig geeignet. Mein Ziel ist es, eine umfassende und schnell erfassbare Anleitung mit einfachen Antworten zu den meisten Fragen zur Durchführung von HRT verfügbar zu machen. Diese lebensrettende Medizin soll sowohl für Menschen, die HRT in Betracht ziehen, als auch für erfahrene Transpersonen entmystifiziert werden. Somit setze ich eine gewisse Vertrautheit mit den Effekten von HRT voraus. Falls Du nicht damit vertraut bist: HRT ist sehr effektiv und macht wahrscheinlich mehr als Du denkst. Es ist toll \textbf{Dein Geschlecht zu verändern ist richtig cool und macht Spaß. Ich empfehle es.} Du verdienst eine gute medizinische Versorgung deiner Transition und kannst die besten Entscheidungen für Dich treffen. Ich hoffe, dass dieses Dokument eine nützliche Hilfe in deinem Entscheidungsprozess sein kann, und, falls es Dich interessiert, als Startpunkt für weitere Recherche dient.

Und halte Dich fern von den transbezogenen Subreddits. Vertrau mir einfach, okay? Vermeide zumindest /r/mtf, da dieses besonders schlimm ist. Sie sind keine gesunde Orte und dort ist nicht viel Weisheit zu erwarten. Du läufst die Gefahr, jahrelang mit dem psychischen Schaden kämpfen zu müssen. Bester Rat, den ich geben kann! (A.d.Ü. /r/germantrans ist bisher ganz okay, was toxische Einstellungen etc. angeht und für unsere kleinere, zum Teil zersplitterte Community schon wichtig und hilfreich. Trotzdem nur in kleineren Mengen zu genießen!)

Für die Jungs: teile dieses Dokuments sind schon sehr relevant, aber es gibt offensichtlich entscheidende Unterschiede in den Zielen und Effekten.. \href{https://docs.google.com/document/d/1DXFxzN0XTudPZez\_SO61fpqncRLPH\_Be\_QG\_8Pcz9LU/edit?tab=t.0}{Diese Anleitung für männliche HRT} \textcolor{red}{(Warnung: Google Docs link (A.d.Ü. und auf Englisch))} sieht ziemlich gut aus, ich habe sie aber nicht vollumfänglich untersucht, setze also auf Deinen Verstand und lass Vorsicht walten. Die sollten sich sowieso ein tboy Katie Tightpussy einfallen lassen. Oliver Longdick oder so. Vielleicht Xavier. 

\textbf{Wenn Du dieses Projekt durch eine Spende unterstützen möchtest,} \href{https://cash.app/Katitties}{CashApp}, \href{https://ko-fi.com/katitties}{Ko-Fi}, und \href{https://account.venmo.com/u/katitties}{Venmo} gehen alle. Vielen Dank!

\subsection*{Wie dieses Dokument benutzt wird}

Das Dokument ist linear strukturiert als eine Reihe von Fragen und Antworten, sodass in der Regel jede Frage und Sektion fließend in die nächste übergeht. Ich empfehle es, von oben bis unten alles durchzulesen, um hoffentlich wie in einer Unterhaltung alle mögliche Fragen zu beantworten (sogar diejenige von denen Du nicht wusstest, dass Du sie hast), aber das dauert offensichtlich eine Weile. Nimm Dir Zeit und lese es Stück für Stück, wenn Du willst.

Du kannst das Inhaltsverzeichnis benutzen, um zu einer bestimmten Frage oder Sektion zu navigieren, vor allem beim Wiederlesen oder Nachschauen. Ich empfehle, diese Seite/dieses Dokument zu speichern, um es immer parat zu haben wenn Du Fragen zu deiner HRT hast. Es ist sehr viel Information auf einmal, es ist okay wenn Du es  Dir nicht alles sofort merkst! Es gibt keine große Eile. 

\noindent\textbf{\href{pghrt.pdf}{Hier kann das Dokument als PDF heruntergeladen werden. Mach das bitte.}}

\noindent\href{pghrtgretchensversion.txt}{Hier kannst Du es alternativ als 90er-Style Textdatei lesen, wenn Dir das Spaß macht.} Diese Version wird nicht aktualisiert. (A.d.Ü. und ist nur auf Englisch verfügbar!)



\section*{WIDMUNG}
\addcontentsline{toc}{section}{WIDMUNG}

Dieses Dokument ist all unseren Schwestern gewidmet, die es nicht geschafft haben. Mögen wir das Licht ihrer Fackel in einen neuen Tag tragen.
 

\section{INTRODUCTION}

\subsection{Ist Östrogen sicher?}

Mit modernen, bioidentischen Hormonen ist HRT so sicher wie noch nie. Du ersetzt einfach den Saft, den dein Körper in erster Linie benutzt, durch den anderen, und veränderst das Gleichgewicht der Hormone, die sich schon in deinem Körper befinden. Obwohl die Details der Optimierung komplex sein können, ist der Prozess ziemlich fehlertolerant. Der Körper ist biegsam und Du wirst es schaffen, es so zu justieren dass es sich gut für Dich anfühlt.

\subsection{Welche Verabreichungsform sollte ich für Östrogen wählen?}

Injektionen. Sie sind insgesamt die effektivste, einfachste, einheitlichste, sicherste und preiswerteste Form von HRT. Für manche werden Injektionen zu einem wilkommenen Ritual, oder machen sogar Spaß.

\noindent\underline{\textbf{Aber merke: irgendein Östrogen ist besser als kein Östrogen.}}

\subsection{Warum werden Pillen, Gels oder Patches nicht empfohlen?}

Weil sie alle gegenüber Injektionen große Nachteile haben. Es ist nicht so, dass sie nichts bringen, aber Du hast es nicht verdient, diese Nachteile aushalten zu müssen. Ich wiederhole: \textbf{alle Formen von HRT können gute Ergebnisse erreichen}, aber das heißt nicht, dass sie alle gleichwertig sind.

\subsection{Ist die Östrogen-Dosis die gleiche zwischen den verschiedenen Verabreichungsformen?}

Nein. Dies ist wichtig genug, dass ich es nicht in der Sektion \ref{MM} “MYTHEN AND VESCHIEDENES” gepackt habe. \textbf{Die Dosierung von Östrogen kann zwischen verschiedenen Verabreichungsformen nicht direkt verglichen werden.} 1mg von der einen Sorte ist nicht 1mg von der anderen. Unterschiedliche Formen des gleichen Wirkstoffes haben unterschiedliche Eigenschaften, die die Aufnahme des Östrogens durch den Körper beeinflussen (\textit{“Bioverfügbarkeit”}),sowie die Geschwindigkeit der Aufnahme und die daraus resultierende Halbwertszeit des Wirkstoffes.

\subsection{Was bedeutet "Halbwertszeit?}

Die \textit{Halbwertszeit} ist einfach die Zeit, die es braucht, bis die Hälfte einer Substanz im Körper aufgenommen oder eliminiert wurde. Im Kontext von HRT ist dies dafür entscheidend, wie lang eine Dosis in deinem System aktiv ist, und somit dafür, wie oft Du eine Dosis brauchst. Das ist dann dein Hormonzyklus, und es bildet eine Kurve. Der Östrogenspiegel steigt, erreicht einen Höhepunkt, und sinkt wieder ab. Die Eigenschaften dieser Kurve (wie sich der Östrogenspiegel über die Zeit verändert) sind wichtig.

\subsection{Was spricht gegen Pillen?}

Das größte Problem mit Pillen sind das erhöhte Risiko für Blutgerinnsel oder Leberstörungen bei langfristiger Anwendung. Dieses Risiko kann abgeschwächt werden, indem die Pillen sublingual oder bukkal (also indem die Pille unter der Zunge oder in der Backe aufgelöst wird) statt oral (die Pille einfach schlucken)eingenommen werden, um den sogenannten First-Pass-Effekt in der Leber zu vermeiden. Es ist jedoch anzunehmen, dass sogar bei sublingualer oder bukkaler Anwendung etwas von der Pille verschluckt wird, sodass die Risiken nicht ganz weg sind. Bitte verstehe, dass das absolute Risiko immer noch sehr niedrig ist (z.B. ist \textit{Paracetamol} um eine Größenordnung gefährlicher für die Leber als Östrogen), jedoch \textbf{erhöht sich dieses Risiko weiter in Kombination mit nikotinbedingtem Östrogen-Risiko.} Siehe Frage \ref{11-2} dazu.

Darüber hinaus sprechen zwei weitere Eigenschaften von Pillen gegen ihre Anwendung: 1) ihre kurze Halbwertszeit und schlechte Bioverfügbarkeit, 2) der in Kombination mit Pillen oft notwendige Einsatz von Antiandrogenen. Die erste Eigenschaft bedeutet, dass Pillen für Monotherapie (wird weiter unten besprochen)im Vergleich mit Injektionen meistens ungeeignet sind. Die zweite bedeutet oft eine Menge an unerwünschten Nebeneffekten, je nachdem welches Antiandrogen verwendet wird (siehe Sektion \ref{AA} “ANTIANDROGENE”). Zusammen führen diese Eigenschaften zu mehr Variabilität, die schwierige Abläufe und unerwünschte Nebeneffekte  (wie z. B. niedrige Energie/Libido und langsamere Ergebnisse) wahrscheinlicher machen als andere Verabreichungsformen. Es ist auch schwieriger, mit Pillen einen Vorrat anzulegen, und mancherorts sind sie teurer als Injektionen. Merke auch, dass die Einfuhr von Pillen aus dem Ausland in großen Mengen den Zoll alarmieren kann, was zu finanziellen Verlusten, zum Verlust der Pillen und/oder möglicherweise zu rechtlichen Problemen je nach deinem lokalen Gesetz führen kann. \textbf{Falls irgendjemand fragt, weißt du nicht, wer diese Pillen bestellt hat.}

\textbf{Falls Du aus welchem Grund auch immer Pillen nimmst, nimm bitte 4-8mg sublingual über den Tag verteilt. Unter 4mg ist fast nie eine ausreichende Dosis.

\subsection{Was spricht gegen Pflaster?}

\begin{itemize}
  \item Relativ teuer (meistens sogar teurer als Pillen);
  \item Schwieriger über DIY zu bekommen (nur über den sogenannten "grauen Markt");
  \item Benötigen meistens ein Antiandrogen (siehe Sektion \ref{AA} “ANTIANDROGENE”);
  \item Kann zu Hautirritationen führen;
  \item Müssen 24/7 dran bleiben;
  \item Neigen dazu sich abzulösen;
  \item Die Aufnahme ist nicht immer gleichmäßig (verändert sich z.B. durch Hitze);
  \item Schwieriger, einen Vorrat anzulegen (schwer, sie in größeren Mengen zu bekommen);
  \item Schaffen es meistens nicht, Östrogen-Werte über Wechseljahren-Niveau anzuheben, sogar mit mehreren Pflaster auf einmal.
\end{itemize}

\subsection{Was spricht gegen Gel?}

\begin{itemize}
  \item Schwer, es genau zu dosieren, was zu ungleichmäßigen Werten führt ;
  \item Es muss regelmäßig eingeschmiert werden, da die Halbwertszeit relativ niedrig ist;
  \item Anwendung kann anstrengend sein (schleimig);
  \item Risiko der passiven Exposition von anderen durch Hautkontakt
  \item Benötigt meistens ein Antiandrogen (siehe Sektion \ref{AA} “ANTIANDROGENE”).
\end{itemize}

Es muss jedoch angemerkt werden, dass Gel mit relativ wenig Aufwand selbst produziert werden kann, was in machen Fällen ein Segen sein kann.

\subsection{Was ist mit Pellets (Implantate?}

\begin{itemize}
  \item Meistens viel teurer als andere Optionen;
  \item Werden sehr selten angeboten;
  \item Die Dosis kann nur über einen langen Zeitraum justiert werden;
  \item Fehlerhafte Pellets können zu schlechten Werten führen;
  \item Gebrochene/zerquetschte Pellets können zu unerwarten erhöhten Werte führen ;
  \item Generell als DIY unmöglich.
\end{itemize}

Der letzte Punkt bedeutet, dass es nur über die wenige, wahrscheinlich teure Anbieter möglich ist, überhaupt an Pellets ranzukommen. Vielleicht hast du überhaupt hier zum ersten Mal davon gehört. Siehst Du das Problem? 

\subsection{Was ist mit Sprays?}

Diese sind noch recht experimentell, daher kann nicht so viel darüber gesagt werden, aber sie teilen sich Vor- und Nachteile mit Gel. Sie werden hier erwähnt damit Du weißt, dass es sie gibt.

\subsection{Ist der Unterschied wirklich so relevant?}

\textbf{Ja.}So relevant, dass ich das alles ausgeschrieben habe, damit ich mich weniger wiederholen muss indem ich diesen Punkt einfach verlinke. Ein gut dosiertes Injektionsrythmus ist zum Erreichen von Monotherapie-Werten die beste Form von Östrogen.

 

\section{WARUM INJEKTIONEN}

\subsection{Was macht Injektionen so gut?}

Einheitlichkeit. Bei HRT kommt es auch Einheitlichkeit an. Einheitliche Hormonenwerte bedeuten Stabilität, und Stabilität ist gut. Sogar die "schlimmste" Art von Injektionen (lese weiter) kann einen einheitlicheren Hormonzyklus als andere Verabreichungsformen ermöglichen.

\subsection{Sind Antiandrogene bei Injektionen notwendig?}

Meistens nicht. Ein richtig dosierter und zeitlich abgestimmter Injektionszyklus, der einheitliche, genug hohe Östrogen-Werte erreicht, kann die Testosteron-Produktion auf natürlicher Weise unterdrücken. Somit wird kein Antiandrogen benötigt, was in den meisten Fällen bevorzugt wird. Dieses Vorgehen wird \textit{“Monotherapie”} genannt.

\subsection{Wie funktioniert Monotherapie?}\label{2-3}

Ganz vereinfacht gesagt interessiert es das Gehirn nicht, welches Hormon es hat, solange es genug davon hat. Wenn ständig genug Hormone in deinem Körper vorhanden sind, werden keine weitere produziert. Diese Einheitlichkeit ist das, was Injektionen ermöglichen und andere Verabreichungsformen oft nicht schaffen. Zum Beispiel ist Monotherapie mit Pillen in den meisten Fällen so gut wie unmöglich. Genauer gesagt und im Bezug zur Hypothalamus-Hypophysen-Nebennierenrinden-Achse (HHN-Achse) werden durch ein hoher Estradiol-Serumspiegel das \textit{luteinisierende Hormon} (LH) und \textit{follikelstimulierende Hormon} (FSH) unterdrückt, was wiederum die Produktion vom Gonadotropin-Releasing-Hormon (GnRH) unterdrückt und damit die Produktion von Testosteron in den Testikeln.

\subsection{Inwiefern sind Injektionen sicherer?}

Indem Antiandrogene meistens nicht gebraucht werden, werden die damit verbundenen Langzeitrisiken vermieden(siehe Sektion \ref{AA} “ANTIANDROGENE”). Bioidentisches Östrogen, das gar nicht durch die Leber verarbeitet wird (siehe Frage \ref{11-1}), kommt einer natürlichen Östrogen-Produktion am nächsten und verringert somit weiter die Risiken.

\subsection{Aber ist das Injizieren nicht an sich schon gefährlich?}

Ja, aber durch minimale Schulung (siehe Sektion \ref{ts} “METHODE UND ZUBEHÖR”) lässt sich die Gefahr bis auf den einen oder anderen blauen Fleck minimieren. Es ist ein Bisschen wie das Fahrradfahren: sobald Du weißt, wie es geht, müsstest Du dich schon SEHR anstrengen, es richtig falsch zu machen.

\subsection{Inwieweit sind Injektionen leichter?}

Wenn Du erstmal eingepegelt bist, ist alles gut. Injektionen müssen nicht oft verabreicht werden (so z.B. mit einer wöchtentlichen Injektionen gegen mehrere Pillen am Tag), haben eine geringe Gefahr, falsch dosiert zu sein (anders als bei Gel), können nicht mitten im Zyklus abfallen (wie bei Pflaster), und man muss dafür nirgendwo hinfahren (wie es bei Pellets der Fall ist).

\subsection{Inwieweit sind Injektionen preiswert?}

In einfachen Worten, es wird weniger Östrogen gebraucht. Ein 5ml Fläschchen, das dich nahezu ein Jahr lang mit Östrogen versorgen kann, beinhaltet nur 200mg Östrogen, während das äquivalente Minimum an Pillen (4mg * 365 Tage = 1460 mg) wesentlich mehr beinhaltet. Dies ist kein rigoröser Vergleich, macht den Unterschied im Maßstab jedoch deutlich. Als weiterer lustiger vergleich kann man 1 bis 2 Jahre Östrogen-Fläschchen in einer typischen Packung Pillen für drei Monate verstauen.

\subsection{Aber ich habe keine Krankenversichrung / meine Versicherung zahlt es nicht / durch meine Versicherung sind Pillen billiger als Injektionen / Injektionen gibt es in meinem Land nicht / mein Arzt verschreibt mir keine Injektionen ?}

Siehe bitte Sektion \ref{sv} “ÖSTROGEN-QUELLEN”. Du wirst erstaunt sein und, sehr wahrscheinlich, radikalisiert werden.

\subsection{Ist es gut, auch nach Jahren der HRT zu Injektionen zu wechseln?}

\textbf{Ja.} Nichts ist sicher, aber viele Leute erleben wesentliche, bemerkbare Unterschiede nachdem sie zu Injektionen wechseln, selbst nach Jahren der Hormontherapie. Diese reichen von verbessertem Brustwachstum, verbessertem psychischen Wohlbefinden, weniger Nebeneffekte von Antiandrogene oder anderen Östrogen-Formen, bis hin zu allgemein besserer Stimmung etc. Der Wechsel lohnt sich.

\subsection{Aber Injektionen machen mir Angst?}

Ja, am Anfang kann es so sein. Niemand mag Nadeln, weil der Körper sich natürlicherweise nicht selbst stechen will, aber mit der richtigen Methode und dem richtigen zubehör tut es gar nicht weh. Es gibt zahlreiche Menschen, die eine intensive Nadelphobie hatten und das Ganze jetzt langweilig finden. Die Angst ist normal und weit verbreitet, aber sie kann überwunden werden, und es lohnt sich. “Oh, das war ja gar nicht so schlimm” ist ein Satz, den man oft hört. Wie Seneca einst schrieb: „Trau dich, sei mutig! Kein Übel ist so schlimm wie die Angst davor“. Du wirst es schaffen.

\subsection{Sind Injektionen wie eine Blutabnahme oder eine Impfung?}

Nein. Zur Blutabnahme werden meist viel größere Nadeln (auch oft Kanülen genannt) verwendet, die an einem empfindlicheren Ort reingesteckt werden, und dazu wird auch noch Blut abgenommen, was unangenehm ist. Impfungen beinhalten Impfstoffe, die schmerzhafte Immunreaktionen auslösen, weil sie Impfstoffe sind. HRT-Injektionen bringen ein Bisschen Hormone in dir rein, wodurch Du dich wohlfühlst, weil Du Hormone in dir hast. Ich hoffe Du siehst den Unterschied. Es kann auch einfacher sein, sich selbst zu injizieren, als wenn eine fremde Person es macht, je nachdem wie du drauf bist.

\subsection{Gibt es barrierefreies Zubehör für Injektionen?}

Ja. Auto-Injektoren existieren und können zum Beispiel bei Problemen mit Feinmotorik sinnvoll sein. Siehe bitte Frage \ref{5-21}, oder lese einfach weiter.

\subsection{Aber ich \textit{bin} anders und kann nicht injizieren weil ich Glasknochen habe und meine Haut ist aus Papier \textemdash{}?}

Ich verstehe die Angst, aber wenn Du wirklich unter keiner Bedingung Injektionen machen willst und keine legitime Kontraindikation wie Hämophilie hast, dann mach es halt nicht. Das kannst Du einfach sagen. Es ist okay. Wenn Du deine Meinung veränderst wird diese Anleitung immer noch da sein. Und wenn nicht, dann nicht. 

 

\section{ARTEN UND DOSIERUNGEN}\label{td}

\subsection*{Wichtige Begriffe}
\addcontentsline{toc}{subsection}{\textemdash{} Wichtige Begriffe}

\subsection{Was sind die verschiedenen Arten von injizierbarem Östrogen?}

Die vier wichtigsten Arten, die für HRT benutzt werden, sind \textit{Estradiolvalerat} (EV), \textit{Estradiolcipionat} (EC), \textit{Estradiolenantat} (EEn), und \textit{Estradiolundecylat} (EUn). Diese sind sogenannte "Ester" von \textit{Estradiol} und werden im Körper zu \textit{Estradiol} verarbeitet. 

Bemerkenswerterweise werden in manchen Regionen Pillen mit dem Namen \textit{Estradiolvalerat} verkauft, was verwirrend ist, aber hier geht es nur um die Form für Injektionen.

\subsection{Was sind die Unterschiede zwischen den verschiedenen injizierbaren Östrogen-Formen?}

Der einzige relevante Unterschied zwischen den Estern ist die jeweils unterschiedliche Halbwertszeit und die Werte-Kurve, die sich daraus ergibt. Dies beeinflusst wiederum die Dosis und Häufigkeit der Anwendung.

\subsection{Ist eine Art von injizierbarem Östrogen besser als die andere?}

\textbf{Nein.} Die Unterschiede beziehen sich auf die Dosis und die Häufigkeit der Anwendung, dies ist ein qualitativer Unterschied, der die eine oder andere Art vorteilhafter machen kann. Alle vier zeigen jedoch eine gute Wirkung und behalten die Vorteile von Injektionen gegenüber andere Methoden  . 

\subsection{Welche Art von injizierbarem Östrogen sollte ich wählen, wenn ich es mir aussuchen kann?}

Wenn Du die Wahl hast, wird \textit{Estradiolenantat} von den meisten Menschen bevorzugt, da es sehr stabile Werte ermöglicht. Es muss jedoch angemerkt werden, dass es in den meisten Ländern nur bei DIY zur Verfügung steht (siehe Sektion \ref{sv} “ÖSTROGEN-QUELLEN”). Über die ärztliche Versorgung kann es sein, dass dir\textit{Estradiolcipionat} angeboten wird (A.d.U. in Deutschland soweit mir bekannt nicht), aber meistens in niedrigkonzentrierter Form, was zu hohen Injektionsmengen führen und somit die Vorteile hinfällig machen kann. Die am häufigsten verschriebene Art (insbesondere in den USA (A.d.Ü. und in Deutscland)), \textit{Estradiolvalerat}, kann richtig gute Ergebnisse ermöglichen, hat aber ihre Macken (insbesondere bei DIY), weshalb sie nicht die erste Wahl sein sollte. Lies weiter.

\subsection{Was ist “Konzentration”?}

Östrogen-Fläschchen beinhalten Östrogen, das in Öl aufgelöst ist. Die \textit{Konzentration} eines Fläschchen ist wie viel Östrogen in dieser Öllösung ist. Dies wird als ein Verhältnis von Gewicht und Volumen für das Fläschchen ausgewiesen. Anders gesagt: Für jedes Milliliter Öl (Volumen), gibt es soviele Milligramme Östrogen (Gewicht). Oft wird die Konzentration für das Gesamtvolumen des Fläschchens angegeben (z.B. 200mg / 5ml), es ist aber fast immer besser, dieses Verhältnis vereinfacht auszudrücken (also in diesem Fall 40 mg/ml). \textbf{Typische Konzentrationen sind 5 mg/ml, 10 mg/ml, 20 mg/ml, 40 mg/ml, und manchmal 50 mg/ml.}

\subsection{Was ist mit "Dosis und Häufigkeit der Anwendung" gemeint?}

\textit{Dosierung} und \textit{Häufigkeit} sind die zwei Faktoren, die deinen Hormonzyklus bestimmen. Die \textit{Dosierung} ist die Menge an Östrogen, die Du zu dir nimmst (gemessen in mg), und die \textit{Häufigkeit} ist wie oft du es zu dir nimmst (gemessen in Tagen oder Wochen). Manchmal wird für die Kombination aus beiden Faktoren das Wort "Regimen" benutzt, das dann beschreibt, wie oft und wieviel HRT-bezogenen Sachen Du zu dir nimmst.

\subsection{Wie finde ich meine Dosierung raus?}

Deine Dosierung ist die Konzentration deines Fläschchen mal das Volumen, das du injizierst. \[Konzentration (mg/ml) * Volumen (ml) = Dosierung (mg)\] \textbf{Bitte verstehe, dass das Volumen allein keine Aussage zur Dosierung macht.} Es ist wie beim Backen: Du kannst nicht einfach sagen "45 Minuten im Ofen backen", ohne auch die Temperatur zu nennen.

\subsection{Gibt es Beispiele für Dosierungsberechnungen?}

Die Berechnung ist ganz einfach, versprochen! Weiter unten ist eine kleine Referenztabelle die Konzentratin und Volumen für einige gängige Dosierungen vergleicht. Es wird nur auf zwei Kommastellen aufgerundet. Du wirst keine Spritzen verwenden, die sowas wie 0.153ml messen können. Das liegt innerhalb des Rundungsfehlers und macht in unserem Fall keinen relevanten Unterschied.

\begin{table}[]
\centering
\caption{Beispiele für Dosierungen von üblichen Konzentrationen, bei Volumen}
\label{tab:Konzentrationen}
\begin{tabular}{@{}lllll@{}}
    \toprule
    \multicolumn{1}{c}{} & \multicolumn{4}{c}{Konzentrationen (mg/ml)} \\
    \cmidrule(rl){2-5}
            & 5    & 10  & 20 & 40    \\
            \cmidrule(rl){2-5}
Dosierung (mg) & \multicolumn{4}{c}{Volumen (mL)}  \\
    \cmidrule(r){1-1} \cmidrule(lr){2-5} 
4        & 0.8  & 0.4 & 0.2  & 0.1      \\
5        & 1    & 0.5 & 0.25 & 0.13   \\
6        & 1.2  & 0.6   & 0.3  & 0.15     \\
7        & 1.4  & 0.7 & 0.35  & 0.18  \\
8        & 1.6  & 0.8   & 0.4  & 0.2    \\
9        & 1.8  & 0.9 & 0.45  & 0.23 \\
10       & 2    & 1   & 0.5  & 0.25   \\
    \bottomrule
\end{tabular}
\end{table}

\textbf{Wie diese Tabelle gelesen wird:} Links findest du deine gewünschte Dosierung, rechts das entsprechende Volumen und in der oberen Zeile die jeweilige Konzentration. Du wirst merken, dass das benötigte Volumen für eine vernünftige Dosierung bei einer Konzentration von 5 mg/ml  nicht gut ist (zu viel). Das liegt daran, dass Fläschchen mit einer Konzentration von 5 mg/ml nicht gut sind.

\subsection{Wie kann ich Dosierungen zwischen den verschiedenen Estern konvertieren?}

\textbf{Machst Du nicht. }Da die Ester sich jeweils anders verhalten, gibt es keine "Konvertierung" in diesem Sinne. Falls Du zu einem anderen Ester wechselst, solltest Du einfach eine typische Dosierung für dieses Ester benutzen und sehen, wie es sich bei dir verhält. Es ist möglich, zwischen ihnen zu vergleichen, aber es gibt keine Methode für eine direkte Konvertierung.

\subsection{Wie kann ich Werte-Kurven und Dosierungen ziwschen Estern vergleichen?}

Falls Du etwas nerdy drauf bist, kann ich \href{http://estrannai.se}{estrannai.se} sehr empfehlen. Sowas zu benutzen ist nicht verpflichtend, aber es ist ein gutes Werkzeug für grobe Vergleiche. \href{https://estrannai.se/\#i0__cu,7,7,1-cu,5,7,3-cu,5,7,2}{Hier ist ein beispielhefter Vergleich zwischen typischen wöchentlichen Dosierungen} die wir uns jetzt einzeln anschauen werden.

\textbf{Beachte, dass die unten aufgeführten Dosierungen im unteren Bereich meist ausreichen sollten.} Fang mit der kleineren Menge an und gib mehr, wenn du es brauchst. Mehr ist nicht zwangsläufig besser, aber darüber reden wir später. Die Herkunft deines Fläschchens hat wahrscheinlich keinen Einfluss auf diese Dosierungsempfehlungen.

\subsection*{Lerne die Ester kennen!}
\addcontentsline{toc}{subsection}{\textemdash{} Lerne die Ester kennen!}

\subsection{Wie dosiere ich \textit{Estradiolvalerat}?}

Für gute Werte mit \textit{Estradiolvalerat} empfehle ich entweder eine niedrige Dosis zwei mal die Woche, oder eine höhere Dosis einmal die Woche. Es ist eine Frage der Bequemlichkeit und Verträglichkeit. Die typische Faustregel ist etwa 1 mg pro Tag in einem Zyklus von 3 bis 7 Tagen. \textbf{Ich empfehle eine wöchentliche Dosis von 6-8mg}, aber 4-5mg in  5 Tagen ist auch üblich. \textbf{Es sollte immer mindestens wöchentlich injiziert werden (also nie länger als 7 Tagen zwischen Injektionen).} Ein wöchentlicher Rythmus ist schon an der Grenze dessen, was das Ester leisten kann. Mehr Zeit zwischen Injektionen wird ausdrücklich nicht empfohlen, aufgrund der Varianz-bedingten Nebenwirkungen (Siehe Frage \ref{7-3}).

Bemerkenswerterweise werden in manchen Regionen Pillen mit dem Namen \textit{Estradiolvalerat} verkauft, was verwirrend ist, aber hier geht es nur um die Form für Injektionen.

\subsection{Wie sieht die Werte-Kurve für \textit{Estradiolvalerat} aus?}

\textit{Estradiolvalerat} ist ziemlich pingelig. Es geht schnell hoch, mit einem Gipfel wenige Tage nach der Injektion, geht aber genauso schnell wieder runter. Diese relative Instabilität kann unangenehm sein, je nach deiner Empfindung, lässt sich aber durch die richtige Einstellung von Dosierung und Frequenz teilweise mildern.

 \begin{figure}[H]
     \centering
     \includegraphics[width=1\linewidth]{ev.png}
     \caption{Estradiol-Serumspiegel (pg / ml) von Estradiolvalerat vs Zeit (Tagen) }
     \label{fig:ev}
 \end{figure}

\subsection{Wie dosiere ich \textit{Estradiolcypionat}?}

\textit{Estradiolcypionat} kann problemlos wöchentlich angewendet werden. \textbf{Eine wöchentliche Dosierung von 5-7mg ist typisch.} Eine niedrigere Frequenz (z.B. alle 10 Tage) wird nicht empfohlen, da sie weniger Effizient ist undfür gute Werte höhere Dosierungen verlangt. Jede Verlängerung über 7 Tage hinaus erhäht die Gefahr von Varianz-bedingten Nebenwirkungen (Siehe Frage \ref{7-3}).

\subsection{Wie sieht die Werte-Kurve für \textit{Estradiolcypionat} aus?}

\textit{Estradiolcypionat} ist weniger pingelig als \textit{Estradiolvalerat}. Die Kurve geht nicht so schnell hoch und runter, es kommt aber über eine Woche hinweg immer noch einer gewissen Schwankung.

 \begin{figure}[H]
     \centering
     \includegraphics[width=1\linewidth]{ec.png}
     \caption{Estradiol-Serumspiegel (pg / ml) von Estradiolcipionat vs Zeit (Tage) }
     \label{fig:ec}
 \end{figure}

\subsection{Wie dosiere ich \textit{Estradiolenanthat}?}

\textit{Estradiolenanthat} kann leicht wöchentlich eingesetzt werden und kann zur Not auch eine Frequenz von 10 Tagen ermöglichen, wenn Bedarf besteht. Ein noch längerer Abstand zwischen den Injektionen ist theoretisch möglich, jedoch aufgrund der enstehenden Varianz der Werte nicht zu empfehlen. \textbf{Eine wöchentliche Dosiserung von 4-6mg ist typisch}, bei 10 Tagen wird 5-7mg empfohlen. Eine Verlängerung auf 10 Tage scheint jedoch keine wesentliche Vorteile zu bieten, sodass die wochentliche Anwendung empfohlen wird.

\subsection{Wie sieht die Werte-Kurve für \textit{Estradiolenanthat} aus?}

\textit{Estradiolenanthat} ist bei injizierbarem Östrogen der Goldstandard. Die Kurve ist über die Dauer der typischen wöchentlichen Anwendung sehr flach (wenig Varianz). adurch werden sehr stabile Werte ermöglicht, was die Gefahr von Varianz-bedingten Nebeneffekten verringert (Siehe Frage \ref{7-3}).

 \begin{figure}[H]
     \centering
     \includegraphics[width=1\linewidth]{een.png}
     \caption{Estradiol-Serumspiegel (pg / ml) von Estradiolenanthat vs Zeit (Tage) }
     \label{fig:een}
 \end{figure}

\subsection{Wie dosiere ich \textit{Estradiolundecylat}?}

\textit{Estradiolundecylat} kann die Frequenz weit über das wöchentliche bis ins monatliche strecken. Die dafür empfohlene Dosierung ist jedoch nicht standardisiert oder bekannt. Anders als bei den anderen Estern sind die Faktoren, die die Aufnahme des Östrogens bedingen (\textit{“Pharmakokinetik”}) bei \textit{Estradiolundecylat} sehr wichtig. Dementsprechend ist seine Anwendung nooh sehr experimentell und führt über diese Anleitung hinaus. Du könntest dich zum Beispiel an deiner lokalen Hexe wenden, um den Rythmus der Injektionen an das Mond-Kalendar anzupassen.

\subsection{Wie sieht die Werte-Kurve für \textit{Estradiolundecylat} aus?}

Wissen wir nicht wirklich. Es gibt zu wenige Daten und zu viele Varianten, um ein genaues Bild davon zu haben. Wenn es dich interessiert kannst du gern selbst (und an dich selbst) experimentieren, sei dir jedoch der Risiken bewusst. Ich empfehle es nicht, wenn Du nicht ohnehin weißt, was du tust.

 \begin{figure}[H]
     \centering
     \includegraphics[width=1\linewidth]{moon.png}
     \caption{Der Mond}
     \label{fig:moon}
 \end{figure}

 

\section{BLUTTESTS UND WERTE}

\subsection*{Werte bekommen}
\addcontentsline{toc}{subsection}{\textemdash{} Werte bekommen}

\subsection{Wie oft sollte ich meine Werte testen?}

In der Einstellphase willst Du relativ oft testen. Nach jeder Veränderung in der Anwendung solltest Du eins bis zwei Monate warten, damit deine Werte sich stabilisieren, und dann testen.

\subsection{Muss ich meine Werte vor Beginn der HRT testen?}

Eigentlich nicht, da das Testosteron zu hoch und das Östrogen zu niedrig sein werden, was erwartbar ist, aber ein allgemeiner Check-Up deiner Blutwerte (Leber, Lipide etc.) kann deiner Gesundheit nicht schaden. Bei Verdacht einer Intergeschlechtigkeit, die die HRT beeinflussen könnte, kann ein vorbereitender Bluttest empfehlenswert sein, um dies auszuschließen.

\subsection{Muss ich meine Werte testen, wenn ich über längerer Zeit nicht verändert habe?}

Eingentlich nicht, da sich von selbst eben nichts verändert haben sollte. Es kann aber Sicherheit geben, wenn andere Aspekte deiner Routine sich verändert haben. Falls Du mit \textit{Estradiolundecylat} experimentierst solltest Du mindestens vierteljährlich testen.

\subsection{Ich bin nicht versichert oder habe keinen Arzt, wie kriege ich ein Bluttest?}

Suche nach privaten Labors in deiner Region, je nachdem, ob es legal ist. In vielen Regionen können private Bluttests gekauft werden, diese sind aber oft nicht billig. Online-Angebote sind manchmal billiger, das ist aber regional unterschiedlich (A.d.Ü.: Im deutschsprachigen Raum gibt es viele verschiedene Anbieter für private Bluttests. Google einfach, was es in deiner Region/dein Preissegment gibt.)

\subsection{Ich bekomme keinen Bluttest/kann mir keinen leisten. Kann ich trotzdem HRT machen?}

Es ist natürlich besser, Information über die Werte bekommen zu können, aber an sich ist HRT sehr sicher und bei typischen Dosierungen unproblematisch. Du musst dich halt mehr auf das verlassen, was du fühlst und an dich beobachtest.

\subsection{Welche Werte soll ich testen lassen?}

Mindestens \textit{Estradiol} (E2) und \textit{Testosteron gesamt} (T)da diese Werte uns am meisten interessieren. \textit{Sexualhormon-bindendes-Globulin} (SHBG), \textit{Dihydrotestosteron} (DHT), \textit{Estrone} (E1), und \textit{Prolaktin }(PRL) zu testen kann auch sinnvoll sein, wenn Du Schwierigkeiten hast, um bei der Aufklärung zu helfen. \textit{Follikel stimulierendes Hormon} (FSH) und \textit{luteinisierendes Hormon} (LH) können dir sagen, ob deine Hypothalamus-Hypophysen-Nebennierenrinden-Achse deaktviert ist, was die basis für Monotherapie bildet (Siehe Frage \ref{2-3}). Aber ich wiederhole: \textbf{\textit{Estradiol} und \textit{Testosteron gesamt} sind das Wichtigste. }

\subsection{Zu welchem Zeitpunkt in meinem Hormonzyklus sollte ich testen?}

Am Ende des Zyklus (\textit{“Talwert”}). Du willst so nah wie möglich am tiefsten Wert messen, da dieser die beste Information bietet. Man könnte sogar meinen, dass sie die einzig relevante Information ist, da gleichmäßige Talwerte hier das Wichtigste sind. Zum Beipiel: Wenn Du immer Donnerstag Nachmittag injizierst, solltest dein Bluttest am Vormittag oder frühen Nachmittag des nächsten Donnerstags machen.

\subsection{Mein Artzt sagt, ich soll den Höchstwert / den Mittelwert testen, soll ich?}

\textbf{Nein.} Der Höchstwert sagt gar nichts aus und zeigt nur, welchen Ester Du benutzst. Wohlwollend interpretiert deutet solch eine Vorgehensweise auf Inkompetenz, die auf veraltete, konservative Versorgungsstandards beruht. Etwas weniger wohlwollend interpretiert ist es böswillig, für Östrogenwerte zu sorgen, die zu schlechten Ergebnissen oder gar gesundheitlichen Schäden führen können. \textbf{Ich empfehle, trotzdem den Talwert zu messen.}

\subsection*{Bluttest auswerten}
\addcontentsline{toc}{subsection}{\textemdash{} Bluttest auswerten}

\subsection{Welche Östrogenwerte will ich erreichen?}

In der Transition ist dies vielleicht die kontroverseste Frage. Die einfache Antwort ist: Hoch genug, dass Du dich wohl fühlst und dein Testosteron unterdückt wird, wenn Du es willst. Höhere Werte sind darüber hinaus im besten Fall verschwenderisch, im schlimmsten kontraproduktiv. Das schließt jedoch ein breites Spektrum ein und durch die vielen Variablen gibt es viel individuelle Schwankungen. Anders gesagt: Du willst genug Östrogen, so dass Du dich gut fühlst, und das war's.

\subsection{Führen höhere Östrogen-Werte zu schnelleren oder besseren Ergebnissen?}

\textbf{Nein.} Östrogen-Werte, die über das Notwendige hinausgehen, werden von manchen aus subjektiven Gründen bevorzugt, aber sie verbessern die Feminisierung nicht. tatsächlich können zu hohe Werte zu unerwünschten Nebeneffekten führen, wie zum Beispiel Stimmungsschwankungen. \textbf{Für die Feminisierung ist es viel wichtiger, das Testosteron zu einem ausreichend niedrigen Wert zu unterdrücken.}

\subsection{Okay, aber welche Zahl will ich bei Östrogen auf meinem Befund sehen?}

Mit dem Verständnis, dass die genaue Zahl keine Rolle spielt, dass die Zahl aufgrund der Latenz immer etwas höher sein wird als das, was beim Talwert gemessen wird, und dass die Zahl in einer Wolke von Möglichkeiten liegt, die auf einer Reihe von Faktoren basieren, \textbf{empfehle ich einen Talwert von mindestens ca. 200 pg/ml (730 pmol/L).} Diese Empfehlung ist etwas konservativ, da die Unterdrückung der HHN-Achse schon früher erfolgt, aber sie beitet etwas Spielraum. Für die meisten funktioniert es in diesem Bereich ganz gut, aber manche mögen es etwas tiefer oder höher. Ich denke nicht, dass diese Zahl voll im Fokus stehen sollte, da sie immer vairabel ist und das wichtigste dein Gefühl ist, \textbf{aber} \textbf{über 300pg/ml (1100 pmol/L) beim Talwert ist sehr wahrscheinlich höher als es sein muss oder sollte.} Es gibt Ausnahmen, aber Du bist wahrscheinlich nicht die Ausnahme. Mach aber einfach, was sich für Dich gut anfühlt. Darüber hinaus siehe Frage \ref{11-1}.

\subsection{Was für einen Testosteron-Wert will ich?}

Die Unterdrückung des Testosterons ist die Bedingung für die Feminisierung, also reicht meistens weniger als 50 ng/dL (1.7 nmol/L). \textbf{Wohlbemerkt ist ein Testosteronwert nahe Null nicht erwünscht.} Siehe Sektion \ref{T} “TESTOSTERON”.

\subsection{Ich habe von Natur aus hohes/niedriges T. Muss ich meine Dosierung anpassen?}

Wahrscheinlich nicht. Die Testosteronwerte, die typischerweise vor der HRT bestehen, sind meistens für die Feminisierung höher als erwünscht (Siehe Frage \ref{2-3}). Die Ausnahme wäre irgend eine Form von  Intergeschlechtigkeit, die Grund sein könnte für eine genauere Justierung der Dosierung. Dies führt jedoch über die Grenzen dieser Anleitung hinaus. Du musst wahrscheinlich die Empfehlungen nicht anpassen, aber vielleicht fühlst Du dich besser, wenn Du es tust. Letztendlich sollst Du das machen, was sich gut anfühlt. Siehe Frage \ref{9-2}.

\subsection{Ich hatte eine Geschlechtsangleichende Operation (GA-OP). Sollen meine Östrogen-Werte anders sein?}

Da die Unterdrückung von Testosteron für dich kein Problem mehr ist, könntest Du dich wahrscheinlich mit niedrigeren Östrogen-Werten gut fühlen, \textbf{aber Du brauchst immer noch Östrogen.} Da Du jetzt keine eigene Hormone merh produzierst ist es extrem wichtig, dass Du deinen Körper mit ausreichend Hormonen versorgst. Keine oder zu wenige Hormonen führen zu Wechseljahresbeschwerden, wie sie ältere Cis-Frauen erleben können. Passe es so an, wie es dir passt.

Genauer gesagt: \textbf{Ein Minimum von ca 100 pg/ml (350 pmol/L) ist notwenig, um Probleme mit der Knochenmineraldichte zu vermeiden.} Wenn deine Feminisierung großtenteils schon erfolgt ist, dass ist dein Hormonprofil in vieler Hinsicht mit dem einer menopausalen Cis-Frau vergleichbar und es kann sinnvoll sein, von ihnen zu lernen (Siehe Frage \ref{11-29}). Ein Supplementierung mit Testosteron kann in manchen Fällen von Antriebslosigkeit oder Müdigkeit sinnvoll sein (Siehe Frage \ref{9-2}).

\subsection{Kann ein Bluttest irgendwie ungenau sein?}

je nachdem, welche Methode beim Bluttest angewandt wird, können Nahrungsergänzungsmittel mit Biotin den \textit{Estradiol} (E2) Wert (unter anderen, aber uns geht es \textit{Estradiol }) unerwartet hoch erscheinen lassen. Es ist nicht immer möglich zu wissen, welche Methode angewandt wird, so ist es einfacher, für einige Tage vor dem test kein Biotin zu nehmen. Es ist auch immer möglich, dass bei der Blutprobe oder bei der Ausrüstung für den Test ein Fehler vorkommt, das ist aber sehr unwahrscheinlich.

\subsection{Sind verschiedene Ester oder Verabreichungsformen in Bluttests erkennbar?}\label{4-16}

Nein. Nur durch ein Bluttest kann nicht erkannt werden, welche Östrogen-Form eine Person einnimmt. Die verschiedenen injizierbaren Ester werden alle im Körper zu \textit{Estradiol } konvertiert, genau wie wir es wollen, und so ist es auch mit Pillen, Patches, Gels, Sprays oder was auch immer. Am Ende des Tages ist es alles Östrogen.

 

\section{METHODE UND ZUBEHÖR} \label{ts}

\subsection*{Injektionsstellen \& Sicherheit}
\addcontentsline{toc}{subsection}{\textemdash{} Injektionsstellen \& Sicherheit}

\subsection{Wie führe ich eine Injektion sicher durch?}

Ich empfehle diese beiden Videos:

\begin{enumerate}
  \item \href{https://www.youtube.com/watch?v=cBabaGC2Dok}{\textit{“How to perform an intramuscular (IM) self-injection”}}
  \item \href{https://www.youtube.com/watch?v=YfNlAZLxLyw}{\textit{“Painless (for me so far) IM Injection Technique”}}
\end{enumerate}
(A.d.Ü.: Die deutschprachigen Videos, die ich zum Thema finden konnte, sind alle recht lang und für Pflegepersonal gedacht und somit für unsere Zwecke mit vielen unnützlichen Infos versehen. Im wesentlichen können die hier verlinkten Videos auch ohne Englischkenntnisse verfolgt werden und die wichtigen Punkte werden weiter unten aufgeführt. Du kannst dir die deutschen Videos auf youtube natürlich gern anschauen) 

Mithilfe dieser zwei Videos solltest Du ausreichend vorbereitet sein, um eine schmerzlose Injektion bei dir richtig vorzunehmen. Ich empfehle, sie sich aufmerksam anzuschauen und bei Bedarf zu wiederholen. \textbf{Um Schmerzen zu verhindern ist ein Punkt besonders wichtig: Den abgeschrägten Teil der Nadelspitze (die Fase) beim reinstechen nach oben zu halten).} Anders gesagt: Die Nadel hat eine klar definierte Spitze, und diese soll deine Haut als erste berühren. Du willst schön gerade reinstechen. Du kannst dir gern vorstellen, wie deine Hand bzw. dein Handgelenk sich dabei bewegen sollen, wenn es dir hilft dir die Bewegung vorzustellen, aber am Ende ist es Übungssache, bis es intuitiv wird.

\textbf{Merke: Injizieren ist eine besondere Fertigkeit, die gelernt werden will!} Du wirst mit der Zeit besser, und es braucht wirklich nicht lange. Kriegste schon hin..

\subsection{Muss ich genau so injizieren?}

Nin, Variationen sind erlaubt. Da es am Ende bloß darum geht, dich zu pieksen, gibt es viele Wege, wie man es machen kann. Finde den Weg, der für dich am besten ist. Ein schnelles, scharfes Reinstechen funktioniert meist am besten, aber wenn Du langsamer gehen willst ist das auch okay, solange Du es immer so machst und dich so mit der Zeit verbesserst.

\subsection{Wie kann ich die Angst vor dem Injizieren überwinden?}

Ich empfehle daraus ein Ritual zu machen. Wenn Du dir eine Routine aufbaust, wird es irgendwann ganz natürlich. Wenn Du dich mit Musik, ein Gespräch, Fernsehen oder was immer für dich gut ist ablenken kannst, damit dein Muskelgedächtnis übernimmt, toll! Finde heraus, was zu dir passt. Es kann helfen, wenn eine befreundete Person oder ein partner die ersten Injektionen übernimmt. Für die meisten ist die erste Injektion die aufregendste. Danach sagen die meisten "Oh, das war's?", weil es nie so schlimm ist, wie sie es erwartet hatten.

\subsection{Ist es wichtig, wo ich am Körper injiziere?}

Ja und nein. Es ist wichtig, es bei sicheren Orten zu belassen, aber sonst hängt es davon ab, wie gelenkig Du bist, was für ein Volumen Du injizierst, was für eine Spritze Du benutzst, und dein eigenes Gefühl. Es ist aber wichtig, \textbf{Injektionsstellen abzuwechseln.} Zum Beispiel kannst Du jede Woche auf eine andere Körperseite injizieren, mal ins linke Bein, mal ins rechte Bein. So wird das Risiko einer langfristigen Narbenbildung verringert.

\subsection{Welche Injektionsstellen sind sicher?}

Darüber streiten sich die Gelehrten, aber es kommt vor allem auf deinen Körperbau an. Ich empfehle die Beine, wie in den Video(s) zu sehen, da es für die meisten gut erreichbar ist und gut einheitlich sein kann, wenn Du eingeübt bist, aber machen bevorzugen die Pobacke oder den Bauch. \href{https://vertisis.com/articles/how-to-self-administer-a-subcutaneous-injection}{Dieses Video auf dieser Webseite} zeigt andere Injektionsstellen, die je nach deinem zubehör in Frage kommen können. (A.d.Ü.: Eine entsprechende Deutsche Version mit einer detaillierten Anleitung als pdf findest Du \href{https://www.google.com/url?sa=t&source=web&rct=j&opi=89978449&url=https://www.bk-trier.de/media-bkt/docs/PIZ_HZ_Subkutane_Injektion_2021.pdf&ved=2ahUKEwjE4eeYysyQAxWI2wIHHRSaBeEQFnoECEkQAQ&usg=AOvVaw0b0EnCvlrFZ1CVUbOtc9lA}{hier}.) Finde raus, was für dich am besten ist.

\subsection{Was bedeuten "intramuskulär" (IM) und “subkutan” (SubQ/SC)?}

Im Kontext von Injektionen werden diese Begriffe oft verwendet. \textit{Intramuskulär} bedeutet, dass in das Muskelgewebe injiziert wird, \textit{subkutan} dass in das Unterhautfettgewebe injiziert wird.

\subsectionWas ist der Unterschied zwischen intramuskulären (IM) und subkutanen (SubQ/SC) Injektionen?}

\textbf{Im Kontext von HRt gibt es keine wesentliche Unterschiede zwischen subkutanen und intramuskulären Injektionen.} Subkutane Injektionen werden zwar etwas langsamer aufgenommen als intramuskuläre, der Unterschied ist aber generell nicht groß genug um für die Dosierung relevant zu sein. Darüber hinaus wird bei jeder Injektion nicht immer eindeutig nur das Unterhautfettgewebe oder nur das Muskelgewebe getroffen, sodass der Unterschied in der Praxis weiter verschwimmt.

Kleine Randbemerkung: Pharmazeutische Hersteller von Estradiol-Fläschchen geben in der Regel an, dass diese nur für intramuskuläre Injektionen bestimmt sind. Das liegt daran, dass diese nur für diesen Zweck offiziell zugelassen sind. Das ist aber egal.

\subsection{Sollte ich intramuskuläre (IM) oder subkutane Injektionen (SubQ/SC) durchführen?}

\textbf{Das ist die falsche Frage.} \textbf{Eine Injektion ist eine Injektion.} Subkutane Injektionen werden oft empfohlen, da viele davon aus ausgehen, dass sie weniger schmerzhaft sind, es gibt aber keinen wesentlichen Unterschied in der Durchführung. \textbf{Die Vorteile, von denen ausgegangen wird, haben weniger mit der Art der Injektion zu tun als mit Faktoren, die den möglichen Injektionsschmerz beeinflussen.} Die bessere Frage wäre "wie kann ich Injektionsschmerzen minimieren?", aber zuerst zwei weitere Fragen.

\subsection{Spielt mein Injektionswinkel und/oder meine bevorzugte Injektionsmethode eine Rolle??}

Nein. Ich wiederhole: Der wichtigste Aspekt einer Injektion ist, dass Du eine Nadel in deinem Körper reinsteckst und eine Flüssigkeit reinspritzst. Solange die Flüssigkeit nicht wieder rauskommt (oder zumindest nicht viel davon) und es nicht weh tut (oder zumindest nicht viel)hast Du einen fantastischen Job gemacht. \textbf{Ich kann nicht genug betonen, dass die "Wahl" zwischen subkutaner und intramuskulärer nicht relevant ist und für die Wirksamkeit von injizierbaren Östrogen keine Rolle spielt.} \textit{Estradiolundecylat} ist der einzige Fall, in dem die Art der Injektion möglicherweise eienen Unterschied macht, aber die Details sind noch unklar. Der Punkt ist: Bitte mach dir Gedanken über die Sachen, die wichtig sind, und nicht über die Sachen, die unwichtig sind.

\subsection{Muss ich beim Injizieren auf die Aspiration achten?}

Nein. “Aspiration” meint das kurzeitige Zurückziehen des Spritzenstempels beim Injizieren um festzustellen, ob versehentlich ein Gefäß getroffen wurde. Über die notwendigkeit wird gestritten, aber im Kontext von Hormoninjektionen, die sich sonst an den Regeln halten, gibt es keine wesentliche Vorteile. Die Gefahr, ein Blutgefäß zu treffen, ist bei den empfohlenen Injektionsstellen sehr gering und kurze Nadelspitzen verringern diese Gefahr noch weiter. Bei der Injektion in ein Gewebe wird die Aspiration von den meisten medizinischen Fachstellen auch nicht mehr empfohlen.

\subsection{Wie kann ich den Schmerz bim Injizieren minimieren?}

Du kannst üben und deine technik verbessen, aber darüber hinaus ist deine Spritze- und Nadelkombination der wichtigste Faktor. \textbf{Um Beschwerden zu minimieren, sollte die höchstmögliche Nadelstärke ("Gauge", siehe nächste Frage) verwendet werden, die zum Trägeröl in deinem Fläschchen passt, zusammen mit einer passenden Spritze und Nadellänge. }Die richtige Frage ist "welche Nadelstärke und -länge brauche ich?". Um das rauszufinden, lass uns darüber reden wie Nadeln bzw. Kanülen funktionieren.

\subsection*{Nadelkunde}
\addcontentsline{toc}{subsection}{\textemdash{} Nadelkunde}

\subsection{Was bedeutet "Gauge" bei Nadelgrößen?}

\textit{Gauge }ist die Einheit für die Dicke der Nadel/Kanüle (A.d.Ü.: in diesem Abschnitt wird öfters der Begriff "Kanüle" verwendet, da dieser genauer ist. Nadel interessieren uns hier eigentlich nicht, da sie streng gesehen zum Nähen, Stricken oder zur Akupunktur benutzt werden. "Nadel" wird jedoch im Sinne einer einfachen Sprache auch weiter verwendet). Je höher die Nummer, desto dünner die Nadel. Eine 25G-Kanüle ist zum Beispiel dünner als eine 20G-Kanüle. Dünnere Kanülen sind meist kürzer, da sie sich leichter verbiegen können. Es ist nicht überraschend, dass dünnere Nadel weniger weh tun. Die Dicke der Kanüle hat wohlbemerkt keinen Einfluss auf die HRT selbst, es geht nur darum, wie angenehm (oder nicht) die Injektion wird.

\subsection{Was sind “Luer lock” und “Insulin” Spritzen/Nadel?}\label{5-13}

\textit{Luer lock-Spritzen} ermöglichen es, die Kanüle/Nadel zu wechseln, damit jeweils eine passende fürs Aufziehen der Lösung und für die Injektion verwendet werden kann. \textit{Insulinspritzen} habe eine feste Nadel, sodass sie sowohl beim Aufziehen als auch beim Injizieren verwendet wird. Wenn möglich werden Insulinspritzen bevorzugt, da sie bequemer sind und einen sehr geringen Hohlraum/Deadspace haben (Siehe Frage \ref{5-26}) "Luer slip" Nadel werden nicht empfohlen, da sie fehleranfällig sind.

\textbf{Sicherheitshinweis: Das Wiederaufsetzen der Schutzkappe auf Nadeln wird generell nicht empfohlen, da die Gefahr besteht, sich selbst zu stechen. Solltest Du es dennoch tun (z. B. beim Auswechseln einer Ziehnadel), drücke NIE mit deiner Hand in Richtung der Nadel.}

Die Kappe kann brechen und Du könntest dich verletzen, wenn sie nicht richtig aufgesetzt wird. Es wird bevorzugt, die Kappe sanft auf einer ebenen Fläche mit der Nadel "aufzufangen", um sie dann gegen eine Wand oder mit den Fingern an den Seiten zuzudrücken. Bei Injektionen am eigenen Körper gibt es keine Gefahr einer Krankheitsübertragung, sodass diese Warnung in diesem Fall nicht so streng genommen werden muss, aber bei Injektionen an anderen Personen ist dies SEHR wichtig. Zur Entsorgung der Spritzen siehe Frage \ref{5-27}.

\subsection{Welche Nadel-/Kanülenstärke sollte ich beim Aufziehen verwenden?}

Bei Luer-lock-Spritzen empfiehlt sich, eine niedrigere Gauge (= eine dickere Nadel) als beim Injizieren zu verwenden. Eine zu niedrige Gauge kann zum Ausstanzen eines Stopfenfragments führen (Siehe Frage \ref{5-23}), sodass mindestens 21-23G empfohlen wird. Wenn Du geduldig bist und keine großes Volumen injizieren muss wind hohe Gaugen empfohlen, um die Ausstanzgefahr zu reduzieren. Die Nadel wird durch das Einstechen in den Gummistopper nicht stumpf. Diese Frage ist bei Insulinspritzen nicht relevant, da die Nadel in diesem Fall nicht austauschbar ist.

\subsection{Welche Nadel-/Kanülenlänge sollte ich beim Aufziehen verwenden?}

Bei Luer-lock-Spritzen ist die Nadellänge beim Aufziehen nicht so wichtig, zu lange Nadeln können jedoch unpraktisch sein. Anders gesagt gibt es keinen Grund wählerisch zu sein. Diese Frage ist bei Insulinspritzen nicht relevant, da die Nadel in diesem Fall nicht austauschbar ist.

\subsection{Mit welcher Nadel-/Kanülenstärke sollte ich injizeren?}\label{5-16}

Diese Frage ist nicht ganz einfach und eher subjektiv, die Antwort hängt im Wesentlichen von 4 Faktoren: 1) das Trägeröl, was Du injizierst; 2) ob das Fläschchen ein zusätzliches Lösemittel beinhaltet; 3) ob Du die Geduld hast, die Nadel länger in dir zu haben; and 4) deine Bereitschaft/Fähigkeit, den Spritzenstempel stärker runterzudrücken. Es ist eine Frage der Gemütlichkeit. Dickflüssigere Öle können mit hohen Gauge länger brauchen und mehr Druck benötigen, aber dafür tun sie weniger weh beim Reinstechen. \textbf{In der Regel ist 25G das Minimum, um keine Schmerzen zu verursachen. }Die meisten Öle gehen bis 27G gemütlich rein, während MCT-Öl bemerkenswerterweise bis 30G gut geht (Siehe Frage \ref{6-16}).

\subsection{Mit welcher Nadel-/Kanülenlänge sollte ich injizieren?}

\textbf{Ich empfehle zwischen 12.5mm und 25mm, je nach Gauge.} Unter 12.5mm erhöht wird Ausfluss wahrscheinlicher. 6.5mm kann funktionieren, je nach deiner Technik und dem Öl, den Du injizierst, but aber 12.5mm ist die sichere Wahl. Alles über 25mm ist unnötig beängstigend und schmerzhaft, ohne irgendeinen Mehrwert zu bieten.

\subsection{Ist die Spritzengröße wichtig?}

\textbf{Ja, die Größe ist wichtig.} Dafür gibt es zwei Gründen. 1) Größere Spritzen mit mehr Volumen sind meist weniger genau und können zu ungenauen Dosierungen führen, und 2) größere Spritzen mit mehr Volumen sind physikalisch schwieriger zu benutzen. Um genau zu dosieren willst Du eine Spritze benutzen, die nicht viel größer ist als das Volumen was Du injizierst (z.B. sollten für Injektionen von weniger als 0.1ml Spritzen benutzt werden, die kleiner sind als 1ml). \textbf{Vermeide 3ml-Spritzen gänzlich, soweit Du kannst.} Du kannst sie natülich benutzen, wenn es nichts anderes gibt, aber warum diese am häufigsten in Apotheken ausgegeben werden, erschließt sich mir nicht. Vielleicht ein schlechter Scherz. (A.d.Ü.: Im deutschsprachigen Raum sollte dies kein Problem sein. Nadeln und Spritzen sind frei verkäuflich und in einer breiten Auswahl verfügbar.)

\subsection{Wo kaufe ich Spritzen und Nadeln/Kanülen?}

Es hängt von deinem Standort ab, da der Verkauf von Nadeln und Spritzen mancherorts als Strafe gegen drogenabhängige Personen eingeschränkt ist. Sonst sind medizinische und veterinarische Versorgungsgeschäfte gute Quellen, oder direkt bei den Herstellern. \textbf{Amazon wird nicht empfohlen} da die Qualität dort oft unsicher ist. (A.d.Ü.: Siehe oben. Im deutschsprachigen Raum Apotheke deiner Wahl, auch online.)

\subsection{Ist es okay, Nadeln oder Spritzen wiederzuverwenden?}

\textbf{Nein. Benutze Nadeln und Spritzen nur einmal. }Und teile sie auch nicht mit anderen. Das weißt Du wahrscheinlich schon, aber ich wiederhole es, weil es wirklich nicht gut oder sicher ist, es zu tun!

\subsection{Was wenn ich Injektionen machen will, es aber schwierig finde, mich selbst zu injizieren?}\label{5-21}

Du könntest ein Autoinjektor probieren. Wie es im Name steht führt der Autoinjektor die Injizierung für dich aus. Ein Autoinjektor wie der \href{https://unionmedico.com/90-super-grip/}{\textit{UnionMedico 45/90 Super Grip}} (A.d.Ü.: auf Deutsch hier \href{https://www.b12-injektion.de/}) kann 1ml-Spritzen aufnehmen und das Injizieren vereinfachen (aber Du musst immer noch selbst raufdrücken), während Autonijektoren wie der \href{https://www.owenmumford.com/us/medical-devices/autoject-2}{\textit{Owen Mumford Autoject 2}} (A.d.Ü.: auf Deutsch hier \href{https://shop.owen-mumford.de/Sonstiges/Autoinjektor-Autoject-2-mit-austauschbarer-Nadel.html}) die Nadel einer Insulinspritze ganz verstecken und von selbst runterdrücken. Es gibt auch verschiedene 3D-gedruckte Modelle,d ie online verfügbar sind. Ich habe keine dieser Produkte getestet und dies ist keine Empfehlung.

\subsection*{Fläschchenkunde}
\addcontentsline{toc}{subsection}{\textemdash{} Fläschchenkunde}

\subsection{Worauf sollte ich bei einem Fläschchen achten?}

Abgesehen von den Anzeichen eines Ausstanzens des Stopfens (siehe unten)solltest Du auf Anzeichen von Verfärbung, Separation der Lösung, Kontaminierung, Kristallisierung, Bruch im Glas, Staub, Haare, etc. Ein gut hergestelltes Fläschchen sollte sich von den anderen nicht unterscheiden. \textbf{Untersuche dein Fläschchen immer, bevor Du es benutzst. Benutze kein Fläschchen, das nicht gut aussieht.}

\subsection{Was ist mit dem Ausstanzen des Stopfens gemeint?}\label{5-23}

Jedes Fläschchen hat ein Gummistopfen, der die Lösung schützt. Das \textit{Ausstanzen} passiert, wenn ein Stück vom Gummi rausgeschnitten wird/rausbricht und in die Lösung gelangt. Das kann passieren, wenn eine zu große Nadel beim Ausiziehen verwendet wird, wenn immer wieder an der gleichen Stelle reingestochen wird, oder wenn zu viel reingestochen wird (z.B. wenn immer sehr geringen Mengen aus einem recht großen Fläschchen genommen werden). \textbf{Ein Fläschchen mit ausgestanzten Gummistopfen sollte weggeworfen werden. }Mit der \href{https://www.youtube.com/watch?v=w5F0SLoMjC8}{\textit{45-90° technik}} kann diese Gefahr verringert werden.

Du willst einfach keine Gummistückchen in deinen Körper injizieren. Wenn Du größere Gummistücke siehst, könnte es auch kleinere, unsichtbare Stücke geben. Darüber hinaus soll der Gummistopfen die Lösung vor der Luft und vor Bakterien schützen, wenn es einen Loch gibt erhöht sich die Gefahr einer Kontaminierung oder Oxidierung. \textbf{Nebenbei: Bitte entferne die Metallabdeckung oben am Fläschchen, bevor Du es benutzst. }Das mag selbstverständlich erscheinen, aber bei manchen Fläschchen kann man verwirrt sein. 

\subsection{Wie lang ist ein Fläschchen haltbar?}

Ein geschlossenes Fläschchen könnte sich jahrelang halten, wenn es bei stabilen Temperaturen und in der Dunkelheit gelagert wird. Bei der Haltbarkeit geht es vor allem um die Gefahr der Oxidierung oder der Sterilität. Ein angefangenes Fläschchen, das ein Koservierungsstoff beinhaltet (siehe Frage \ref{6-17}), sollte mindestens ein Jahr halten, oder wie immer lang es braucht bis es aufgebraucht ist. Oft steht auf Fläschchen "28 Tage haltbar", das ist aber nur das Minimum, was von Herstellern verlangt wird, nicht die tatsächliche maximale Haltbarkeit. 

\subsection{Wie lagere ich das Fläschchen?}

Stabile Raumtemperatur und dunkel. Hitze und Sonnenlicht können dem Trägeröl schaden, und Kälte kann zur Kristallisierung führen. Kristalle lösen sich wieder auf, wenn die Lösung erwärmt wird, aber wenn dies nicht vollständig passiert kann es zu Irritation beim Injizieren führen. Das trifft auf angefangene und nicht angefangene Fläschchen.

\subsection{Was ist Hohlraum/Deadspace?}\label{5-26}

Mit \textit{Hohlraum/Deadspace} ist die Menge an Flüssigkeit gemeint, die bei einer Injektion verschwendet wird. Diese Flüssigkeit bleibt in der Kanüle/in der Spritze gefangen. Bei einer normalen Luer-lock-Nadel/Spritze kann dies bis zu 1mL betragen, bei einer Insulinspritze kann es swenig sein wie 0.003mL. Es lohnt sich, Hohlraum zu vermeiden, da am Ende ganz schön viel Östrogen dadurch verloren geht. \href{https://hrtcafe.net/Calc/}{Dieser Rechner} kann helfen, die verschwendete Menge je nach Spritzenart einzuschätzen.

Wenn Du die Nadel/Kanüle zwischen dem Aufziehen und dem Injizieren wechselst, dann solltest Du den Spritzenstempel leicht zurück ziehen bevor Du die Aufziehnadel abnimmst, damit die darn enthaltene Flüssigkeit nicht verlorengeht. Es ist nicht viel, aber es kann einen Unterschied machen. Siehe Frage \ref{7-7} für eine andere mögliche Herangehensweise wenn Deadspace ein Problem ist.

\subsection{Was mache ich mit meinen benutzten Nadeln und Spritzen?}\label{5-27}

Werfe sie alle (vorsichtig, mit der Sptize nach unten) in eine dedizierte Entsorgungsbox weg (entweder eine richtige mediznische Entsorgungsbox für medizinischen Sondermüll oder einen stich- und bruchfesten Abfallbehälter, z.B. eine Blechdose vom Kaffee). Wenn es dreiviertel-voll wird, mache es fest zu, sodass es sich nicht von selbst öffnen kann. Schreibe "BENUTZTE SPRITZEN" drauf und entsorge es nach den bei dir geltenden Richtlinien. \textbf{Du darfst es NICHT in den Restmüll tun.} A.d.Ü: In Deutschland ist es zumindest so. Am besten ist es, den Behälter in der Apotheke oder bei irgendeinem Arztbesuch abzugeben, damit es fachgerecht und sicher entsorgt werden kann.



\section{Fläschchen besorgen}\label{sv}

\subsection{Wo bekomme ich Östrogen-Fläschchen zum Injizieren?}

Generell gibt es zwei Möglichkeiten: \textit{pharmazeutische Quellen} und \textit{DIY-Quellen}. Für \textit{pharmazeutische Quellen} brauchst Du meistens ein Rezept vom Arzt, da HRT in den meisten Ländern nicht rezeptfrei ist (oder zumindest nicht in injizierbarer Form). \textit{DIY-Quellen} umfassen alles andere.

\subsection{Sollte ich pharmazeutische Quellen benutzen, oder DIY-Quellen?}

Das ist deine Entscheidung, aber manchmal gibt es gar keine Entscheidung. Es gibt bei jeder Variante Vor- und Nachteile. Es hält dich natürlch nichts davon ab, Östrogen aus verschiedenen Quellen zu beziehen, um die Vorteile beider Varianten zu genießen. In vielen Situationen kann es empfehlenswert sein.

\subsection*{Pharmazeutische Quellen}
\addcontentsline{toc}{subsection}{\textemdash{} Pharmazeutische Quellen}

\subsection{Was sind die Vorteile von pharmazeutischen Quellen?}

\begin{itemize}
  \item Vertrauen in der Qualitätskontrolle und Zertifizierung;
  \item Die Krankenversicherung kann es gänzlich oder teilweise übernehmen;
  \item Kann praktischer Sein, je nachdem wieviel Glück Du mit deinen Ärzten hast;
  \item Das Produkt wird sehr wahrscheinlich konsistent sein;
  \item \textbf{Für die Übernahme von Operationen durch die Krankenkasse kann zumindest der Anschein von einer Benutzung von pharmazeutischen Quellen notwendig sein.}
\end{itemize}

\subsection{Was sind die Nachteile von pharmazeutischen Quellen?}

\begin{itemize}
  \item Kleiner (oder gar keine) Auswahl zwischen Estern;
  \item Möglicherweise lange Wartezeit (Monate oder Jahren);
  \item Möglicherweise nur auf Rezept (ja nach Land) (A.d.Ü.: So ist es in Deutschland);
  \item Die Krankenkasse übernimmt es vielleicht nicht ganz oder gar nicht;
  \item Wird möglicherweise in deinem Land gar nicht verschrieben;
  \item Dein Arzt kann sich einfach weigern, es dir zu verschreiben;
  \item Dein Arzt kann sich einfach weigern, dir eine neues Rezept zu geben;
  \item Lieferengpässe können dein Rezept nutzlos machen;
  \item Du wirst wahrscheinlich unter den stringenten WPATH-Richtlinien behandelt, oder was schlimmeres;
  \item Schweiriger, ein Vorrat anzulegen;
  \item Dein Zugang zu HRT hängt stark von der politischen Stimmung deines Landes ab und dein Transsein wird sehr wahrscheinlich in der medizinischen Akte vermerkt. 
\end{itemize}

\subsection*{DIY-Quellen}
\addcontentsline{toc}{subsection}{\textemdash{} DIY-Quellen}

\subsection{Was sind die Vorteile von DIY-Quellen?}

\begin{itemize}
\item In den meisten Regionen viel billiger;
\item Überall auf der Welt verfügbar;
\item Der Zugang kann viel schneller sein (Wartezeit nur für Produktion und Versand);
\item Leicht, ein Vorrat anzulegen;
\item Volle Auswahl an Estern;
\item Keine Notwendigkeit, mit einem Gesundheitssystem umzugehen;
\item Es ist wahrscheinlich mit Liebe gemacht.
\end{itemize}

\subsection{Was sind die Nachteile von DIY-Quellen?}

\begin{itemize}
  \item Sehr wahrscheinlich nicht in einem zertifizierten Reinraum hergestellt;
  \item Qualität kann je nach Quelle variieren;
  \item Kann je nach Quelle umständlich sein;
  \item Setzt Vertrauen in die Quelle voraus;
  \item Es muss eine Quelle gefunden werden;
  \item Quellen verschwinden eher als deine lokale Apotheke;
  \item Versandzeiten können variieren;
  \item Arbeiten sehr wahrscheinlich mit Krypto, was nervig ist;
  \item Wir nicht von der Krankenkasse übernommen.
\end{itemize}
Darüber hinaus wird, wie bereits erwähnt, für die meisten OPs eine gewisse Zeit mit einer "offiziellen" Hormonbehandlung vorausgesetzt. Je nach deinen Plänen kann das relevant sein

\subsection{Welche Formen von injizierbaren Östrogen gibt es nur bei DIY-Quellen?}

Vor allem \textit{Estradiolenanthat}. Bei pharmazeutischen Quellen gibt es fast immer \textit{Estradiolvalerat}, aber nicht immer mit einer Konzentration von 40 mg/ml. \textit{Estradiolcypionat} wird manchmal verschrieben, aber selten mit Konzentrationen über 5 mg/ml oder 10 mg/ml, was die Dosierung schwierig macht. Allein die Vorteile von \textit{Estradiolenanthat} sind ein guter GRund, DIY in Betracht zu ziehen, aber es ist möglich jeden Ester mit einer Konzentration von 40mg/ml von DIY-Quellen zu beziehen. \textit{Estradiolundecylat} ist auch bei DIY möglich, aber wie bereits erwähnt würde ich es nur bei experimentierfreudigen Menschen empfehlen.

\subsection{Was sind DIY-Quellen \textit{wirklich}?}

Kommerzielle Hersteller, solidarische Projekte, deine Freunde, und sogar Du, wenn Du ein unternehmerischer Geist hast!

\subsection{Wo kann ich DIY-Fläschchen bekommen?}

Bist Du bei der Polizei oder was? Das sage ich dir nicht. Darum geht es in dieser Anleitung nicht. Dafür gibt es andere Quellen (A.d.Ü.: Zum Beispiel in den relevanten Reddits!). Bleib fokusiert.

\subsection{Wie können DIY-Quellen billiger sein als pharmazeutische Quellen?}

Der Herstellungspreis für ein Fläschchen ist ca. \$10, inklusive der Personalkosten und der amortisierten Investitionskosten. Das ist wahrscheinlich eine Überschätzung. Für die kommerziellen DIY-Quellen entstehen die meisten Kosten durch die benötigte Anonymität und der Versand. Nicht-kommerzielle DIY-Quellen habe keine solchen Kosten. Pharmazeutische Quellen haben meist keine Motivation, ihre Preise zu senken. 

\subsection{Ist DIY legal?}\label{6-11}

In den meisten Ländern, inklusive der USA, wird Östrogen nicht streng reguliert, während Testosteron mehr oder weniger streng reguliert wird. Die USA sind in der Hinsicht eine Ausnahme, da die meisten Länder den Besitz von Testosteron nicht kriminalisieren, aber eine strafrechtliche Verfolgung ist selten. \textbf{Dieser Leitfaden stellt keine Rechtsberatung dar.} 
(A.d.Ü.: Nach dem Verständnis der Übersetzerin bringt man sich in Deutschland rechtlich nicht in Gefahr, wenn man Östrogen zu DIY-Zwecken bestellt. Im schlimmsten Fall kann z.B. Östrogen aus dem Nicht-EU Ausland vom Zoll eingezogen werden. Endverbraucherinnen werden (bisher) nicht verantwortlich gemacht. Der kommerzielle Vertrieb ist wahrscheinlich eine andere Frage, aber darum geht es hier nicht).

\subsection{Ist DIY sicher?}

“DIY” als Ganze ist weder sicher noch unsicher, aber nicht alle DIY-Quellen sind gleich. Wenn es darum geht, eine Substanz in deinen Körper zu injizieren, ist die wichtige Frage: Hast Du genug Vertrauen in der Person, die das Fläschchen produziert hat, und darin dass sie richtig gearbeitet hat, damit das Fläschchen steril ist und nur das beinhaltest, was Du brauchst? Bei den pharmazeutischen Quellen wird dieses Vertrauen aufgrund der geltenden Gesetzen und Regeln angenommen. Bei DIY-Quellen muss dieses Vertrauen erst aufgebaut werden, durch Information zum Herstellungsprozess, unabhängige Prüfungen und Feedback innerhalb der Community.

\subsection{Worauf sollte ich aufpassen um rauszufinden, ob eine DIY-Quelle vertrauenswürdig ist?}

Verlasse dich auf dein Bauchgefühl und deinen Kopf. 

\begin{itemize}
  \item Reden sie offen über ihren Produktionsprozess oder machen Angaben dazu? (Wird zum Beispiel Staub gefiltert? Die Antwort sollte ja sein!!!)
  \item Machen sie einen kompetenten Eindruck?
  \item Haben sie ihr Produkt testen lassen? 
  \item Werden sie in der Community als vertrauenswürdig eingeschätzt?
  \item Werden sie von anderen, dir vertrauten Menschen aus der Community empfohlen? Gibt es Rezensionen oder Erfahrungsberichte? 
  \item Fehler können passieren, aber wie wird damit umgegangen? Wird offen damit umgegangen oder werden negative Berichte unterdrückt?
  \item Werden bei kommerziellen Quellen Probleme mit Bestellungen aufgeklärt?
  \item Werden bei kommerziellen Quellen Bestellungen angenommen, obwohl das Produkt noch gar nicht vorliegt? (Du solltest niemals etwas bestellen, was noch nicht produziert wurde!)
  \item Beinhalten die Fläschchen Konservierungsmittel? (Sollten sie!)
  \item Wie lange sind produzieren sie schon? (Diese Frage wird aus guten Gründen nicht immer beantwortet!)
  \item Wie viel produzieren sie? (Diese Frage wird aus guten Gründen nicht immer beantwortet!)
  \item Fühlt sich irgendwas einfach \textit{falsch} an?
\end{itemize}

Diese Beispielfragen können dir helfen rauszufinden, ob die Quelle vertrauenswürdig ist, und ob ihr die Qualität des Produkts so wichtig ist wie dir.

\subsection{Soll ich bei verschiedenen DIY-Quellen verschiedene Standards anwenden?}

Wahrscheinlich ja. Kommerzielle Anbieter, die dein Geld nehmen, sollten hohe Standards erfüllen können, da sie es sich leisten können. Bei Produkten aus der gegenseitigen Hilfe, die umsonst verteilt werden, kann man nicht ganz so pingelig sein, was aber nicht heißt, dass sie unbedingt besser oder schlechter sind. Bei Freunden oder bei deiner eigenen Produktion kannst nur Du es einschätzen!

\subsection*{Aufbau eines Fläschchens}
\addcontentsline{toc}{subsection}{\textemdash{} Aufbau eines Fläschchens

\subsection{Was sollte in einem Fläschchen drin sein?}

Der Inhalt setzt sich aus einem \textit{“aktiven”} Stoff und einem \textit{“Träger”} oder Hilfsstoff zusammen. Der \textit{ aktive} Stoff ist in unserem Fall das Östrogen-Ester, während der \textit{Träger} und weitere Hilfsstoffe den Rest ausmachen. Es gibt generell insgesamt drei oder vier Ingredienten im Fläschchen: 1) Das Östrogen-Ester; 2) das Trägeröl; 3) ein Konservierungsmittel; und manchmal 4) weitere Lösemittel. Die Ester haben wir schon in der Sektion \ref{td} “ARTEN UND DOSIERUNGEN” besprochen. Fläschchen aus pharmazeutischen Quellen beinhalten fast immer alle vier Ingredienten.

\subsection{Welches Trägeröl will ich im Fläschchen haben?}\label{6-16}

Die Antwort zu dieser Frage ist subjektiv und hängt von der persönlichen Vertrläglichkeit und von möglichen Allergien ab. \textbf{Der wichtigste relevante Aspekt für die Injektion ist die Viskosität des Öls, da die Injektion dadurch bequemer oder praktischer sein kann.} Dünnere, flüssigere Öle sind, wie bereits besprochen (Siehe Frage \ref{5-16}), für das Aufziehen und Injizieren mit dünneren Kanülen von Vorteil. \textbf{MCT/MKT-Öl und Rizinusöl werden bei HRT am häufigsten benutzt. }Rizinusöl ist dickflüssiger, führt aber dafür am wenigstenzu Irritationen und wird typischerweise von pharmazeutischen Herstellern benutzt. MCT-Öl ist dünnflüssiger, führt aber bei manchen Menschen zu Irritationen und wird nur von DIY-Quellen benutzt. Baumwollsamenöl und Traubenkernöl werden manchmal eingesetzt, aber meistens nicht für HRT. Andere Öle wie z.B. Sonnenblumenkernöl oder Sesamöl werden manchmal eingesetzt, sind aber nicht empfohlen. Je nach deinen Umständen kann diese Frage irrelevant oder zwingend sein, oder Du kannst gar keine Auswhal haben. 

\subsection{Welches Konservierungsmittel will ich im Fläschchen haben?}\label{6-17}

Für Injektionen wird niedrig konzentriertes \textit{Benzylalkohol} (BA) am häufigsten verwendet. Das muss sein und darf nicht fehlen. \textbf{Benutze nie ein Fläschchen ohne Konservierungsmittel. }Für Menschen mit der seltenen Allergie dagegen wird meist \textit{Chlorobutanol }als Alternative verwendet, aner meistens nie bei DIY-Quellen, da sie den Stoff dafür extra auftreiben müssten.

\subsection{Welche zusätzliche Lösemittel will ich im Fläschchen haben?}

Am häufigsten wird \textit{Benzylbenzoat} (BB) verwendet,  das die Lösung flüssiger macht. Das ist an sich optional, wird aber oft empfohlen und ist bei manchen Trägerölen und Konzentrationen notwendig. Bei manchen Menschen führt es zu Irritationen, bei anderen nicht. 

 

\section{FEHLERBEHEBUNG}

\subsection*{Unsicherheit bei der Dosierung}
\addcontentsline{toc}{subsection}{\textemdash{} Unsicherheit bei der Dosierung}

\subsection{Meine Werte sind nicht so geworden, wie ich es erwartet habe. Warum?}

Dafür kann es verschiedene Ursachen geben. Erinnere dich daran, dass Modelle und Simulationen nicht alle mögliche Faktoren einbeziehen und dadurch abweichende Werte liefern können. Erinnere dich auch, dass es mehrere Injektionen braucht, bis deine Werte stabil sind. Wenn Du vor kurzem die Dosierung verändert hast, kann es also daran liegen. Lass dich von einer befreundeten Person vergewissern, dass Du tatsächlich so viel injizierst, wie Du denkst. Dieses Problem kommt öfters vor als man denken würde, aber bei DIY-Quellen kann auch aufgrund von Unerfahrung oder ungenaues Equipment die Konzentration niedriger sein als erwartet. In diesem Fall solltest Du bei diesem Fläschchen etwas mehr injizieren als sonst. \textbf{Aber am wichtigsten bleibt immer wie Du dich fühlst, und nicht die Werte an sich!. }Beachte auch, dass sogar in einer Apotheke professionell angemischten Fläschchen manchmal schlecht oder abweichend sein können, auch wenn das (hoffentlich) sehr selten vorkommt!

\subsection{Kann ich die Werte von verschiedenen Tests vergleichen wenn ich nicht den Talwert gemessen habe?}

\textbf{Nein.} Zumindest nicht genau. Gena deshalb sollte immer der Talwert gemessen werden. Mehrere Stunden vor dem normalen Zeitpunkt deiner Nächsten Injektion; dann willst Du messen. Die Daten werden viel nützlicher, wenn Du so viele Variablen wie möglich ausschließt. Wenn Du dir sonst nichts aus dieser Anleitung merkst, dann bitte das: Messe den Talwert.

\subsection{An den Tagen um den Talwert herum fühle ich mich richtig schlecht. Was soll ich tun?}\label{7-3}

In den meisten Fällen ist entweder die Dosierung zu niedrig oder der Abstand zwischend en Injektionen zu groß. Das ist bei \textit{Estradiolvalerat} und \textit{Estradiolcypionat} besonders relevant. Passe deine Dosierung (innerhalb der empfohlenen Werten) oder dein Rythmus an. Finde raus, was sich für dich gut anfühlt. Insbesondere bei \textit{Estradiolvalerat} kann es sein, dass deine Dosierung tatsächlich zu *hoch* ist und nicht zu tief, da die großen Schwankungen innerhalb des Zyklus die Verstimmung auslösen können. Kurz gesagt: wechsle zu \textit{Estradiolenanthat}, wenn Du kannst.

\subsection*{Injektionsprobleme}
\addcontentsline{toc}{subsection}{\textemdash{} Injektionsprobleme}

\subsection{Die Injektion ist schwieriger, wenn es kalt is. Was soll ich tun?}

Erwärme das Fläschchen vor dem Aufziehen, und erwärme die Spritze vor dem Injizieren. Beies kannst Du machen, indem Du das Fläschchen oder die Spritze zwischen den Handflächen rollst oder hälst. Mach dir das zur Gewohnheit, um deine Injektionspraxis zu verbessern.

\subsection{Die Injektion tut mehr weh, wenn es kalt ist. Was soll ich tun?}

Wärme  deine Injektionsstelle vorher auf. Du kannst die Muskeln mit einer Massage oder einer warmen Dusche vor dem Injizieren entspannen (zum Beispiel indem Du, wenn Du am Bein injizierst, mit dem heißen Wasserstrahl gezielt die Stelle aufwärmst).

\subsection{Ich habe nach der Injektion geblutet. Werde ich sterben?}

Nein. Das bedeutet nur, dass Du ein kleines Blutgefäß getroffen hast, das kann passieren. Vielleicht kriegst Du einen blauen Fleck oder es tut später etwas weh. Wenn Du einen süßen Dino-Pflaster drauf klebst, wird es schneller heilen. 

\subsection{Es gab etwas Luft in meiner Spritze. Werde ich sterben?}\label{7-7}

Nein. Du willst natürlich nicht nur Luft injizieren, und zu viel Luft kann die Dosierung beeinflussen, aber alles unter 0.1ml ist sehr wahrscheinlich egal. Bei manchen Substanzen kann es sogar empfohlen sein. So wird die \textit{"air lock technique"} standardweise bei irritierenden oder färbenden Flüssigkeiten eingesetzt: dabei werden 0.1 bis 0.3ml Luft mitinjiziert (diese Info ist für HRt nicht wirklich relevant). Du musst dir also keine Sorgen machen, du injizierst ja nicht intravenös.

\subsection{Ein Teil der Flüssigkeit ist aus der Injektionsstelle ausgelaufen. Habe ich meine Injektion verschwendet und/oder werde ich sterben?}

Nein. Es kann aus vielen Gründen dazu kommen, dass einBisschen was ausläuft, und es ist nur selten genug, um einen Unterschied zu machen. Die Injektion musst Du nciht wiederholen. Versuche in der Zukunft, die Nadel 5-10 Sekunden drin zu lassen und danach auf die Stelle zu drücken. Wenn Du dir besonders viele Sorgen um das Auslaufen machst, kannst Du versuchen, die vorher erwähnte Air-lock-technik anzuwenden.

\subsection{Manchmal tut es nach einer Injektion richtig weh. Werde ich sterben?}

Nein. Auch wenn Du alle Anweisungen hier sonst gefolgt hast, kann es trotzdem vorkommen, dass die Flüssigkeit sich an einer besonders ungemütlichen Stelle sammelt. Nächstes Mal hast Du besseres Glück. \textbf{Pass auf, deine Injektionsstellen zu rotieren!} Du willst nicht, dass sich durch das wiederholte injizieren an der gleichen Stelle eine Narbe bildet. Wenn eine Stelle bereits weh tut, willst Du sie nicht noch schmerzhafter machen

\subsection{Nach dem Injizieren juckt es und die Stelle ist irritiert. Werde ich sterben?}

Wahrscheinlich nicht. Dafür kann es verschiedene Gründe geben. Am schlimmsten wäre eine Infektion, aber das ist in den meisten Fällen sehr unwahrscheinlich. \textbf{Geh sofort zu einem Arzt wenn Du Fieber oder starke Schmerzen hast oder Muskelschmerzen, Eiter, sich ausbreitende Rötungen oder andere Zeichen von Infektion bemerkst. }Jucken, leichte Rötungen, leichtes Anschwellen oder Wärme an der Injektionsstelle gehen jedoch meistens darauf zurück, dass das Östrogen und das Trägeröl sich in der Lösung getrennt haben. Siehe unten. Es ist auch möglich, dass Du eine allergische Reaktion auf das Trägeröl (oder eine weitere Zutat) erlebst, aber wenn dies plötzlich auftaucht und die vorigen Injektionen ohne probleme waren, wird es eher an einer Trennung der Lösung liegen.

\subsection{In meinme Fläschchen sind Kristalle. Kann ich es trotzdem benutzen?}

Sehr wahrscheinlich liegt es daran, dass es zu kalt geworden ist. Wärme es leicht auf und schüttel es, um die Lösung wieder zu vermischen . Wenn die Kristalle nicht verschwinden kann es sein, dass die Lösung sich gänzlich getrennt hat. Mit viel mehr Wärme und Durchmischung könnten die Kristalle vielleicht wieder verschwinden, aber es ist besser das Fläschchen einfach auszutauschen, wenn es geht.

 

\section{PROGESTERON}

\subsection{Will ich Progesteron nehmen?}

\textbf{Wahrscheinlich.} Aus was für einen Grund auch immer ist diese Frage kontrovers. Gegner (vor allem Ärzte) argumentieren, dass eine feminisierende Wirkung durch keine Studien belegt wird und es dementsprechend nicht genommen werden sollte. Abgesehen davon, dass transfeminine Themen chronisch unterstudiert sind, ist Progesteron heuristisch gesehen ein zentrales weibliches sexuelles Hormon, das viele wichtige Funktionen im Körper und im Gehirn erfüllt. Ungeachtet der äußerlichen Feminisierung ist ein wichtiges Hormon für eine gute Gesundheit und sollte nicht leichtfertig übersehen werden.

\subsection{Was ist der Unterschied zwischen “Progesteron” und “Progestine” / ”Progestagene”?}

Die Hormone, die auf die Progesteron-Rezeptoren wirken, heißen “Gestagene” und können sowohl bioidentisch als auch natürlich bzw. bioidentisch sein. Das wichtigste, natürliche und biodentische Hormon darunter ist “Proges\textbf{teron}”. Synthetische Gestagene werden manchmal als “Proges\textbf{tine}” bezeichnet. Die Bezeichnung sind sich alle recht ähnlich und werden oft untereinander verwechselt, obwohl sie \textbf{nicht }gleichwertig sind. (A.d.Ü.: Die deutsche Nomenklatur weicht in diesem Kontext relativ stark von der englischen ab, das Problem bleibt aber ein Stück weit bestehen.)

\subsection{Will ich Progesteron oder ein Progestin/Gestagen?}

Progesteron. Du willst bioidentisches Progesteron.

\subsection{Was ist das Problem mit Progestine/Gestagene?}

Progestine, darunter typischerweise \textit{Medroxyprogesteron}, \textit{Medroxyprogesteronacetat }, oder \textit{Levonorgestrel}, sind oft mit den schädlichen Nebeneffekten (Brustkrebs, Blutgerinnsel, Depression, etc) verbunden, die fälschlicherweise Progesteron zugeschrieben werden. Sie sind nicht bioidentisch und verhalten sich deshalb nicht wie Progesteron, und können nicht direkt damit verglichen werden.

\subsection{Wie trägt Progesteron zur Femninisierung bei?}

Es wird angenommen, dass Progesteron vor allem eine wichtige Rolle beim Brustwachstum und bei der Libido spielt, aber es ist wie gesagt auch davon abgesehen ein wichtiges Hormon. Es ist auch ein Antigonadotrop (das heißt, es trägt zur Unterdrückung von Testosteron bei) was manchmal relevant sein kann.

\subsection{Ist es relevant, wann ich mit Progesteron anfange?}

Das wissen wir nicht. Es wird manchmal behauptet, dass eine frühe Einnahme das langfristige Brustwachstum stören kann, aber das ist rein theoretisch und es gibt ankedotische, gegenteilige Berichte. Somit ist die Antwort unklar. Eine konservative Empfehlung wäre, ca. bis ein Jahr nach Anfang der HRT (also bis Tanner Stadium 3 oder 4) zu warten, für den Fall, dass es einen Unterschied macht.

\subsection{Wie wird Progesteron normalerweise eingenommen?}

Abgesehen von örtlichen Anwendungen wird es meist als Pille verabreicht. Es wird als Pille verschrieben, ist aber als Zäpfchen effektiver. Sprays und Salben zur örtlichen Anwendung funktionieren auch gut.

\subsection{Meinst Du es ersnt, dass das Progesteron als Zäpfchen, also rektal, eingenommen werden soll?}

Progesteron wird rektal ganz anders metabolisiert als oral, da es oral erstmal durch die Leber geht. Oral eingenommenes Progesteron wird primär zu \textit{Allopregnanolon} verarbeitet, das zu schwerer Müdigkeit führen kann, während rektales Progesteron primär zu Progesteron selbst verarbeitet wird, was wir wollen (ein wenig davon wird trotzdem zu anderen Stoffen verarbeitet). Manche Menschen nehmen extra orales Progesteron ein, um besser schlafen zu können, aber zu viel \textit{Allopregnanolon }kann manchmal auch zu negativen psychischen Nebenwirkungen führen.

\subsection{Wie nehme ich Progesteron als Zäpchen ein?}

Etwas Wasser auf der Pille sollte reichen, dann trocknen und Hände Waschen. Offensichtlich solltest Du innerhalb der nächsten Stunde oder so nicht auf Klo gehen, daher empfiehlt es sich vor dem Schlafen. Falls Du Probleme damit hast, dass es sich nicht auflöst, kannst du es probieren, die Kapsel vorher anzupieksen, aber das sollte meistens keine Problem sein. Bei großen, hausgemachten Zäpchen mit Kokosöl solltest Du dir bewusst sein, dass das Öl nicht in dir bleiben will.

\subsection{Wieviel Progesteron sollte ich nehmen?}

Bei Pillen sind 100-200mg täglich nachts standard. Es ist eine etwas arbiträre Dosis; 200mg ist das Maximum, das von den meisten Ärzten verschrieben wird. Manche Menschen nehmen manchmal mehr als 200mg, aber eine kurzfristige Erhöhung der Werte kann zu einem unangenehmen Crash führen, siehe die nächste Frage.

Bei örtlicher Anwendung auf der Haut weiß es niemand, da diese Verabreicherungsform sehr variabel ist und es keine Leitlinien zu den gewünschten Werten oder zur Frequenz (wahrscheinlich täglich) gibt, da Progesteron einfach unterstudiert ist. Deshalb würde ich empfehlen, deine Dosis langsam anzupassen um zu verstehen, wie Progesteron bei dir wirkt.

\subsection{Bringt es Vorteile, Progesteron "zyklisch" einzunehmen?}\label{8-11}

Nein. Manche Menschen tun es, um den Zyklus einer Cisfrau zu imitieren, aber es gibt keine Gründe, positive Effekte zu erwarten. Es kann sogar eher zu PMS-ähnlichen negativen Symptomen führen. Die einzige Ausnahme wäre beim Verdacht einer Intergeschlechtigkeit. Sonst empfehle ich es nicht. Siehe Frage \ref{11-10}.

\subsection{Wie lange soll ich Progesterone einnehmen?}

So lange wie Du Östrogen einnimmst und so lange wie Du willst. Also wahrscheilich für immer.

Manchmal sagen Leute (oder Ärzte), man sollte Progesteron nur für X Jahren einnehmen. Es gibt null theoretische oder empirische Gründe für diese arbiträre Empfehlung. Es macht genauso viel Sinn wie wenn eine Person (oder einen Arzt) einer Transperson fragen würde, wie lange sie vorhat, HRT zu nehmen\textemdash{}oh nee warte das fragen die schon!

\subsection{Kann Progesteron in \textit{Dihydrotestosteron} (DHT) umgewandelt werden?}

Nein. Naja, ganz genau gesehen schon, aber auch wieder nicht. Es handelt sich dabei größtenteils um einen Mythos \href{https://whsah.co/posts/rethinking-progesterone-and-androgens/}{wie von alix in diesem Artikel ausführlich dargelegt}, jedoch kann es bei Menschen mit \textit{ nicht-klassischem adrenogenitalen Syndrom zu negativen Nebeneffekten durch erhöhte Androgenaktivität aufgrund der Einnahme von Progesteron kommen. In solchen Fällen sollte Progesteron nicht eingenommen und eine passende Diagnose und Therapie für Nebennierenerkrankungen gesucht werden.

\subsection{Gibt es neben der Einnahme von Tabletten noch weitere Vorteile bei der topischen Anwendung von Progesteron??}

Vielleicht. Es ist eine Alternative zu den Pillen, insbesondere im Falle eine Erdnuss-Allergie (die meisten Pillen beinhalten Erdnussöl), aber auch hier ist die Dosierung unklar. Manche Menschen finden mehr Progesteron besser. Pass auf dich auf und hab Spaß.

Zur Klarheit: Salben können auf den Schenkelinnenseiten (oder anderswo falls so angewiesen) aufgetragen werden, oder optional skrotal (dort ist die Haut dünn und besonders gut durchblutet), insbesondere bei Sprays. Und nein, das Progesteron direkt auf den Brüsten aufzutragen wird sie nicht größer oder schneller wachsen lassen. 

\subsection{Kann ich Progesteronpulver schnupfen??}

Bitte nicht. Es ist ziemlich schlecht für die Nebenhöhlen. Es ist nicht schwer, ein Spray selbst herzustellen, dazu gibt es Anleitungen. Mach das lieber. Es ist viel effektiver, konsistenter und sicherer.

\subsection{Wo bekomme ich Progesteron?}

Progesteron ist bei DIY-Quellen oft teurer, da größere Mengen an Hormonen benötigt werden, daher solltest Du es idealerweise von pharmazeutischen Quellen über die Krankenversicherung beziehen. Es gibt auch sogenannte "graue" Apotheken aus dem Ausland, aber dort zu bestellen ist oft schwieriger. Salben zur örtlichen Anwendung sind in manchen Ländern auch rezeptfrei erhältlich, aber je nach Konzentration machen sie vielleicht wirtschaftlich keinen Sinn.

\subsection{Ich würde gern mehr zu Progesteron im Kontext von HRT lesen. Welche Quellen bieten sich an?}\label{8-17}

Ursprünglich war hier ein Dokument verlinkt, das ich aber entfernt habe, da es teilweise irreführend war. Das Problem mit Progesteron ist, dass sich niemand über einen einzigen Aspekt davon einig ist. Ich kenne keine einzige Quelle, die von allen als gut akzeptiert wird. Ich sag's dir, es können sich nicht mal alle darüber einigen, dass es mit "P" beginnt. \textbf{Das wichtigste ist, dass Progesteron für die volle Feminisierung oder für Brustwachstum nicht zwingend notwendig ist. Wenn es keine Kontraindikationen gibt, lohnt es sich wahrscheinlich jedoch, es einzunehmen.}

Es ist anzumerken, dass es im Zusammenhang mit Gestagenen unzählige Mythen und Lügen gibt, die von Befürwortern wie Kritikern gleichermaßen erfunden wurden. Dies erschwert es zusätzlich, die Wahrheit aus der ohnehin lückenhaften Forschung herauszufiltern. Fantastische Behauptungen über magische Vorteile und die Panikmache über angebliche, haltlose Risiken sind gleichermaßen kontraproduktiv, wobei Letzteres meiner Meinung nach noch schlimmer ist, wenn es von einer medizinischen Autorität stammt, sei es aus Nachlässigkeit oder böswilliger Absicht.

\subsection{Interagiert Progesteron mit anderen HRT-bezogenen Medikamenten?}\label{8-18}

Wenn Du 5-Alpha-Reduktasehemmer wie \textit{Finasterid} und \textit{Dutasterid} einnimmst (Siehe Sektion \ref{AA} “ANTIANDROGENE”, oder les weiter) können diese die Verarbeitung von Progesteron in \textit{Allopregnanolon} beeinflussen, was wiederum in manchen Menschen zu Verstimmungen führen kann, egal wie sie das Progesteron einnehmen. Es ist nicht ganz klar, inwieweit die Art der Einnahme der 5-Alpha-Reduktasehemmer (örtlich oder oral) eine Rolle spielt, aber eine niedriegere systemische Absorption durch örtliche AAnwendung könnte diese negative Nebeneffekte reduzieren. Es wird empfohlen, die Hemmer nicht einzunehmen, wenn diese Nebeneffekte bei dir auftreten, aber sie treten sowieso nicht in allen Fällen auf. merke, dass diese depressive Verstimmungen bis zu einem Monat nach der Unterbrechung der Einnahme anhalten können. 
 

\section{TESTOSTERON}\label{T}

\subsection{Warum wollen nicht \textbf{kein} Testosteron?}

Testosteron ist ein lebenswichtiges Hormon, das eine Schlüsselrolle in deiner Gesundheit und deinem Wohlbefinden spielt. Wir wollen es für die Feminisierung unterdrücken, aber estrem niedriges Testosteron (weniger als 10 ng/dl, oder 0.35 nmol/L) kann sich negativ auswirken: schlechte Libido, Antriebslosigkeit, Schwäche (über den Muskelverlust aufgrund der HRT hinaus), Konzentrationsschwierigkeiten, Schlaflosigkeit, etc. Diese Symptome ähneln wohlbemerkt denen von Cisfrauen in den Wechseljahren. Cisfrauen haben auch Testosteron in ihrem Körper, das braucht also nicht deine Sorge sein. \textbf{Gute Hormonwerte sind wichtig!}

\subsection{Gibt es Situationen, in denen ich Testosteron zusätzlich einnehmen möchte?}\label{9-2}

Ja. Wenn Du die oben beschriebenen Symptome spürst und deine Östrogenwerte sonst gut sind, könntest Du überlegen, eine Mikrodosis Testosteron zusätzlich einzunehmen. Vielleicht willst Du deine Fähigkeit, Erektionen zu bekommen, verbessern, oder die Atrophie deiner Geschlechtsteile im Vorfeld einer OP entgegenwirken, oder einfach experimentieren, welche Hormon-Kombination sich für dich am besten anfühlt. Alles gute Gründe, Testosteron in einem anderen Kontext als vor der HRT zu erkunden.

\subsection{Falls ich Testosteron zusätzlich einnehmen will, wie könnte ich es tun?}

Es gibt einige Möglichkeiten. Testosteron gibt es entweder als Injektionslösungen oder als Sprays/Gels, wie beim Östrogen. Ein örtliche Verabreicherungsform wie Gel wird eher verschrieben. Örtliche Verabreichungsformen haben die gleichen Nachteile, die wir schon bei Östrogen besprochen haben, aber in diesem Fall ist eine genaue Dosierung weniger wichtig.

\subsection{Was sind die örtlichen Verabreichungsformen für Testosteron?}

Es gibt Gel und Salben. Meistens wird Gel verschrieben, aber manche Apotheken können eine Salbe mit geringer Penetrationsrate herstellen, falls es nur um eine örtliche Anwendung auf den Genitalien geht. Das ist aber schwieriger zu bekommen und oft teurer.

\subsection{Ist bei Testosteron der Ort der Anwendung wichtig?}

Es hängt davon ab, ob es ein Gel oder eien Salbe ist. Bei einer örtlichen Salbe wie oben beschrieben sollte sie direkt auf den genitalien aufgetragen werden. Gel wird auf den Oberarmen oder Schultern aufgetragen. Pass auf, nichts anzufassen, bis es wirlich ganz trocken ist!

\subsection{Wie viel und wie oft sollte ich Testosteron anwenden?}

Je nach Geschmack. Es hängt vor allem davon ab, wie Du dich fühlst. Falls Du zu viel nimmst, können Testosteron-Nebeneffekte auftreten (z.B. ölige Haut oder Körperbehaarung), aber nur Du weißt, was sich gut für dich anfühlt. Eine wöchentliche Injektion von 5 bis 10mg \textit{Testosteroncypionat} könnte für dich funktionieren, aber die 1-prozentige Gels, die oft in 25/50mg Packungen kommen, können mehr Variarion herbeiführen. Ein halbe Packung ist fast immer zu viel, vor allem nciht täglich. IIch würde dir raten, mit viel weniger anzufangen, als Du denkst zu brauchen, und zu sehen wie es sich anfühlt.

\subsection{Wo kann ich Testosteron bekommen?}

Du könntest in deinem lokalen Fitnessstudio nach den muskulösesten, geschwollensten Bodybuilders Ausschau halten und dann höflich fragen. Achtung: Das war ein Witz. Siehe Frage \ref{6-11} “Ist DIY legal?”

\subsection{Können andere Steroide Testosteron ersetzen, im HRT-Kontext?}

Anabol-androgene Steroide, also Stoffe, die in ihrer Struktur testosteron ähneln, sind nicht alle gleichwertig. Oft benutzte Schwarzmarkt-Steroide wie \textit{Trenboloneazetat} haben viele negative Nebeneffekte, aber Steroide wie \textit{Nandrolondecanoat }werden manchmal bei postmenopausalen Cisfrauen eingesetzt, da sie relativ niedrige androgene Eigenschaften haben. Das macht sie für transfeminine Menschen auch interessant. Nichtsdestotrotz ist es unwahrscheinlich, dass dir etwas anderes als Testosteron verschrieben wird. (A.d.Ü.: im Original wird hier explizit auf die US-Lage hingewiesen. Wie es im deutschsprachigen Raum läuft und inwiefern hierzulande Alternativen verschrieben werden, weiß ich nicht. Schreib mir, wenn Du es weißt!)

\subsection{Was ist die Beziehung zwischen Testosteron und \textit{Dihydrotestosteron} (DHT)?}

\textit{Dihydrotestosteron} wird im Körper auf der Basis von Testosteron durch das Enzym 5α-Reduktase gebildet, dabei wird ca. 5\% des Testosterons im Körper umgewandelt. Grob gesagt, wenn der Testosteronwert richtig unterdrückt ist (oder wenn Du eine geschlechtsangleichende OP hattest), dann sollte es nicht viel Testosteron zum Umwandeln geben, der Wert wird aber nicht null sein, da einiges immer noch lokal produziert wird. Je nachdem, wie es bei deinem Körper läuft, könnte dies ein Grund sein, ein 5$\alpha$-Reduktase-Hemmer zu supplementieren, wie in der nächsten Sektion besprochen wird. Zur Erinnerung, \textit{Dihydrotestosteron }sorgt für Körperbehaarung und androgenen Haarverlust.

\textbf{Für die Transmascs, die hier mitlesen} Ich will hier kurz besprechen, dass es bisher nicht bekannt ist, inwieweit dieses Hormon beim bottom growth eine Rolle spielt, sei es bei der Geschwindigkeit oder Größe, im Kontext der Unterdrückung von 5$\alpha$-Reduktase. Das heißt: bekannt ist, dass \textit{Dihydrotestosteron }eine primäre Rolle bei der Penis-Entwicklung spielt, aber es ist unklar, inwieweit die Abwesenheit davon eine transmaskuline Person treffen könnte. Wenn wir das Wissen zur Behandlung von Mikropenissen anwenden, wissen wir, dass eine lokal angewandte Salbe effektiver ist als Injektionen, insbdesondere wie \textit{Dihydrotestosteron }-Salbe bei Patienten sinnvoll ist, die auf Testosteron nicht reagieren (wie bei 5$\alpha$-Reduktase-Defizienz). Etwas zum Nachdenken. Oliver Longdick soll sich darum kümmern!

 

\section{ANTIANDROGENE}\label{AA}

\subsection{Was sind "Antiandrogene"?}

\textit{Antiandrogene, }oft "Testoblocker" oder nur "Blocker" genannt, verhindern die Wirkung von Androgenen (also Testosteron) im Körper, und deshalb heißen sie auch so. Es gibt viele verschiedene Antiandrogene und sie werden oft als Teil der HRT verschrieben. Sie werden gebraucht, wenn die Person noch Testosteron produziert und eine Form der HRT macht, die sich für Monotherapie nicht eignet, aber an sich sind nicht erwünscht. Es muss auch bemerkt werden, dass (die meisten) Antiandrogene die tatsächlichen Testosteronwerte nicht bedeutend unterdrücken, sondern die Effekte vom Testosteron im Körper reduzieren/vermeiden. Das ist bei der Auswertung von Blutwerten und so weiter wichtig.

\subsection{Warum würde ich keine Antiandrogene wollen?}

Das größte Problem mit den meisten Antiandrogenen ist, dass sie oft unerwünschte Nebeneffekte haben und gar nicht notwendig wären, wenn das Testosteron durch genug Östrogen unterdrückt wird. Diese Nebeneffekte könnten also (in den meisten Fällen) durch eine wohldosierte Monotherapie prinzipiell vermieden werden. Eine geschlechtsangleichende OP macht sie ebenfalls (in den meisten Fällen) überflüssig.

\subsection{Wann könnte ich Antiandrogene wollen?}

Wenn Du nicht "die meisten Fälle" bist, wenn es dir einen inneren Frieden bringen würde, oder wenn deine Krankenversicherung das zur Voraussetzung für andere Prozeduren machst, dann könntest Du Antiandrogene gebrauchen. Die Stoffe, die als Antiandrogene verwendet werden, können auch andere Effekte haben, die für deine Gesundheit hilfreich sein können. Und falls Du Androgene supplementierst, könntest Du dir einen \textit{Dihydrotestosteron }-Blocker wünschen, um die Nebeneffekten (Körperbehaarung und Haarverlust) zu minimieren. Das hängt aber auch davon ab, ob Du bioidentisches Testosteron einnimmst(z.B. \textit{Nandrolondecanoat}), da nicht alle Androgene sich gleich verhalten.

\textbf{Es muss angemerkt werden, dass die Anwendung von Antiandrogenen am Beginn einer geplanten Monotherapie weder notwendig noch empfohlen ist.} So oder so kommt es zu einer Anpassungszeit, wenn der Körper sich auf die neuen Hormone einstellt. Es gibt also keinen Grund, die Sache komplizierter zu machen. Mach dir keine Sorgen.

\subsection{Was für Antiandrogene gibt es?}

Im Kontext von HRT werden \textit{Spironolacton}, \textit{Bicalutamid} und \textit{Cyproteron-Azetat} zur Unterdrückung von Testosteron verwendet. Die Medikamente zur Uterdrückung der Umwandlung von Testosteron in \textit{Dihydrotestosteron} (DHT), die “5$\alpha$-Reduktase-Hemmer”,  sind \textit{Finasterid} und \textit{Dutasterid}. GnRH-Analoge wie \textit{Leuprorelin} und \textit{Triptorelin} werden als Pubertätsblocker verwendet, aber in manchen europäischen Ländern werden sie auch bei Erwachsenen angewendet.

\subsection{Wann könnte ich \textit{Spironolacton} wollen?}

Da für die antiandrogene Wirkung sehr hohe Dosen mit signifikanten Nebeneffekten erforderlich sind, würde ich \textit{Spironolacton} nur dann empfehlen, wenn Du von den anderen Effekten etwas hättest, z.B. der Wirkung als  Aldosteron-Antagonist im Kontext von Blutdruckproblemen oder Schwellungen. \textbf{Wenn Du darauf bestehst, \textit{Spironolacton} einzunehmen, dann bitte nicht mehr als 100mg täglich.} Der schlechte Ruf ist begründet. Sozusagen der Teufel.

Falls Du es nicht weißt, einige der Nebeneffekte sind: brain fog, lethargy, poor memory, increased urination frequency, low blood pressure, low sodium / electrolyte imbalance, etc. In other words, \textit{spironolactone} is a blood pressure lowering dieurtic that is a mediocre antiandrogen which is typically prescribed at high dosages in an otherwise-healthy population for questionably-effective off-label use. In any other healthcare context this would (or SHOULD!) be highly unadvisable given the undesirable side effect profile and the widely-available preferable alternatives that already exist, but that's the state of trans healthcare for you.

\subsection{When might I want to take \textit{bicalutamide}?}

If you are going to take an antiandrogen, \textit{bicalutamide} is likely the one to take. It is generally well tolerated, barring 1\% cases of abnormal liver function test results and symptoms of liver dysfunction, but otherwise performs the job with relatively minimal side effects. \textbf{If you take \textit{bicalutamide}, ensure regular liver function tests to make sure that your results are in range. }The liver risks are dependent on your body rather than cumulative so any problem would likely present itself within the first year. Otherwise, there should be no issues. 

\subsection{When might I want to take \textit{cyproterone acetate}?}

Likely never. Take \textit{bicalutamide} instead.

The long term risk profile is poor and there is no situation that I can think of in which I would recommend this over an alternative solution. You can do everything \textit{cyproterone acetate} can by just taking more estrogen and adding progesterone to your regimen.

\subsection{When might I want to take \textit{dutasteride}?}

If you are extremely concerned about possible hair loss and/or want to maximize your chances for hair regrowth, you may want to take \textit{dutasteride}. If your testosterone is otherwise suppressed then it theoretically shouldn’t have much benefit as your \textit{dihydrotestosterone} levels should be relatively low, but bodies can be complicated, so it may be something of interest to you. Also, see Question \ref{11-14}.

It should be noted that \textit{dutasteride} can cause adverse mood effects in some people, in which case stopping is strongly recommended. Note as well that these depressive effects may be felt for up to a month after stopping. 

\subsection{When might I want to take \textit{finasteride}?}

If \textit{dutasteride }is not something prescribed to you or if your insurance mandates \textit{finasteride} specifically to cover a hair treatment. Otherwise, \textit{dutasteride} is preferred as it is more effective and better tolerated.

It should be noted that \textit{finasteride} can cause adverse mood effects in some people, in which case stopping is strongly recommended. Note as well that these depressive effects may be felt for up to a month after stopping.

\subsection{Where can I get antiandrogens?}

Aside from being prescribed them by your doctor or perhaps available over-the-counter, there is also the option of grey market foreign pharmacies. These are simply pharmacies in another country, although these often take some hurdles to purchase from. \textit{Dutasteride} and \textit{finasteride }are generally the easiest to get over-the-counter because of their commonality as hair loss medication.

 

\section{MYTHS AND MISCS}\label{MM}

\subsection*{Common Questions}
\addcontentsline{toc}{subsection}{\textemdash{} Common Questions}

\subsection{Should I be worried about blood clots?}\label{11-1}

Yes and no. It is true that there is a correlation between estrogen dosages/levels and blood clot risk, but this is primarily related to the route of administration and the type of estrogen. Synthetic estrogens are the rightful cause of scorn and do lead to significantly increased blood clot risk, but bioidentical estrogens are not as concerning. In particular, the route of administration makes a major difference. Oral bioidentical estrogen passes through the liver which is what causes the increased blood clot risk. Injections bypass the liver, and there's no evidence to suggest nor reason to believe that injections of bioidentical estrogen provide any significant risk increase beyond the innate differences between testosterone and estrogen. The pervasive fearmongering towards all estrogen has persisted for decades despite these differences.

\textbf{If you are undergoing surgery, please know that pausing hormones out of concern for blood clots is no longer recommended by WPATH.} Many surgeons still include it in their pre-surgery guidelines out of concern for blood clots, but this is torture that has been disproven and even WPATH doesn't recommend it anymore. Remarkable, I know. Per \href{https://www.tandfonline.com/doi/pdf/10.1080/26895269.2022.2100644}{WPATH SOC 8 Statement 12.19}: \blockquote{After careful examination, investigators have found no perioperative increase in the rate of VTE [KT: \textit{venous thromboembolism}, i.e. a blood clot] among transgender individuals undergoing surgery, while being maintained on sex steroid treatment throughout when compared with that among patients whose sex steroid treatment was discontinued preoperatively (Gaither et al., 2018; Hembree et al., 2009; Kozato et al., 2021; Prince \& Safer, 2020).} I should put this in another question entirely, but to not break links, it would have to be at the bottom of a section and I think this is too important for that, so I note it here. A very important clarification that I should have had sooner.

\subsection{Is it okay to use nicotine while on HRT?}\label{11-2}

This is related to the above question. \textbf{Nicotine usage on HRT, especially if you’re on pills, compounds your risk of a blood clot on top of all the other reasons that nicotine is not good.} This extends to all forms of nicotine usage, but obviously smoking is by far the worst. You really do not want a blood clot. Even if you are not on pills, nicotine disrupts the way estrogen is metabolized and can lead to significantly reduced feminization effects. This aspect is understudied but community anecdotal reports are common. It’s not easy to quit, but I believe in you. There are good resources out there and strategies like tapering down by using patches really does work. You got this.

However, to be abundantly clear, \textbf{this does not mean that you cannot or should not take estrogen. The downsides of not taking estrogen at all far exceed the downsides of using nicotine.} This section is simply seeking to make you aware of any increased risks and potentially slower transition as a very strong recommendation and encouragement to quit. One step at a time.

\subsection{Is there benefit to starting at a low dosage vs a high dosage?}

To the best of knowledge, no. Sex hormones are not like other drugs that need to be titrated to manage side effects as we know the dosages that work for the majority of people, so personally I view “starter dosages” and “antiandrogen first” regimens as medical abuse. Some people believe that mimicking the slow timeline of puberty might be best (even though there are far more things happening than just estrogen levels), but there’s no evidence to support this. An orchiectomy day one might be best for all we know, but who is going to do that the moment they decide they are trans and/or want to start HRT?

Reframing this in another way: \textbf{there is no reason to believe that “starting slowly” on a dosage below the typical range is advantageous or preferable for feminization outcomes.} There isn't a concern of going “too fast” or anything like that. Both doctors and other trans women seemingly invent new myths by the day. 

\subsection{Does body weight affect dosage?}

No. Because there is no “optimal” blood level for estrogen and because the therapeutic range of acceptable levels is so wide, body weight does not meaningfully affect dosage for HRT. It is for the same reason that slight deviations in dosage are unlikely to affect how you feel. There is no such thing as being “too light” or “too heavy” for HRT in any capacity.

Adjusting your dosage in increments of 0.1mg is a difference that should not be expected to be perceived simply because our bodies are not sensitive enough to such exact measurements, let alone the high possibility of imprecision when performing an injection that makes that certainty of this measurement unlikely. In other words, the accuracy of your dosage is more important than the precision.

\subsection{Is there such a thing as starting estrogen too late?}

\textbf{No.} No matter when you start, estrogen is able to do a LOT and a proper regimen will be able to have powerful results. Sex hormones are some of the strongest hormones in our body in terms of our appearance. Everybody always wishes that they could’ve started sooner, but that’s no reason not to start now. Even if you’ve been on estrogen for years, there is still a benefit to be had in improving the quality of your regimen.

\subsection{Does feminization / breast development stop after X years?}

\textbf{No.} There is not an arbitrary time where estrogen suddenly stops working. Various numbers are given and usually it’s either 1) entirely made up or 2) pointing to a study that only went for X years. Doctors in particular love to tell trans women not to expect more than B cup breasts (which isn’t even how breast sizing \textit{works}, but I digress) or for any growth after 2 years, but this is simply not true. There are cases of people who restarted estrogen after stopping for many years and still experiencing new growth.

\subsection{I haven’t seen any changes in years on injections. Would swapping back to pills make a difference?}

Maybe, but maybe not. There are some anecdotes of people swapping back from injections to pills (or adding pills on top of injections) and experiencing more breast growth after “stalling out”, but the mechanism is not clear. There is speculation that the E1:E2 ratio (\textit{estrone} : \textit{estradiol}) heavily weighted towards E1 with oral pills compared to E2 for injections might make a difference for some people, although \textit{estrone} is not typically associated with feminization. There likely are other factors at play, but you are free to experiment if you wish. Data is limited.

\subsection{Is low energy and low libido normal on HRT?}

Generally, no. How libido is expressed changes in the beginning, but the vast majority of the time that someone experiences abnormally low libido it’s because they haven’t gotten their hormones sorted. The same goes for low energy. Get your hormones squared away, and barring that, check your diet/vitamins next. Make sure you don’t randomly have critically low vitamin D levels or something like that. It happens more often than you think.

\subsection{I hear about [random drug / strategy] that my friend said helps feminization. Does it actually?}

Maybe, but probably not. There is a lot of wild speculation about ways to achieve feminization goals, but many of them are akin to snake oil or have potentially serious risks far beyond HRT itself. You have the right to bodily autonomy and I cannot stop you, but I can encourage you to be smart about what you are doing. The more you get into the weeds of biology as it relates to transition, the shakier the ground becomes as quality data becomes less and less available. Desperation can lead to a lot of unwise and dangerous decisions. So be smart, and be safe. 

\subsection{Do we want to mimic the estrogen cycle of cis women?}\label{11-10}

Arguably no. This is controversial, but I am of the belief that because we (well, most of us) do not have a uterus and corresponding menstrual cycle synced to our hormone levels, then there is no reason we should strive to copy that behavior. This is an \textit{is-ought} problem, in my view. The primary hormonal concern for most trans women is testosterone suppression which necessitates consistently high enough levels (barring post bottom surgery, where there is no testosterone to suppress), so high fluctuation and/or relatively low levels are likely to cause undue distress. You’re welcome to experiment, of course. Especially if testosterone suppression is no longer a concern for you. See Question \ref{8-11}, and see below.

\subsection{Do trans women experience periods?}

Similar to the last question, it’s important to understand what is happening. The unique hormone curve produced by your particular ester, your dosage, and your frequency can cause changes in your mood as your estrogen levels oscillate between injections. Some trans women liken this phenomenon to a period, but the underlying cause for these physiological changes is different and is usually a sign that your regimen needs tweaking so that you feel the best that you can as suffering is not virtuously feminine. Pain and discomfort are not requirements for womanhood nor should we assert ourselves based on bioessentialist arguments. The exception here are the intersex trans women who have a uterus and literally are having a period, in which case: yeah duh. See Question \ref{11-35}.

\subsection{Can too much estrogen convert to testosterone?}

\textbf{No.} Aromatase is the enzyme responsible for converting testosterone into estrogen, but there is no mechanism to convert estrogen into testosterone. This cannot happen. This is a completely false myth and you should be immediately wary of the knowledge level of anyone who says it to you. Unfortunately, it is doctors who repeat this myth the most.

\subsection{Does bottom surgery cause an increase in testosterone?}

No. This is not a thing. There is not a magic mechanism that suddenly causes testosterone to increase the moment that testicles are removed. Even if magic was stored in the balls, this simply isn’t how hormone production works. “Well, your adrenals…” They don’t work like that either. The only possible rare exception would be undiagnosed adrenal hyperandrogenism conditions that were suppressed by an antiandrogen like \textit{spironolactone }prior to surgery which might show itself after antiandrogens are ceased. Please stop repeating this myth.

\subsection{How do I prevent/revert hair loss?}\label{11-14}

Mechanically, it is pretty simple. A standard HRT regimen alone is borderline magic (don’t ask where the magic is stored) in this regard already, but the inclusion of 5$\alpha$-Reductase Inhibitors (5-ARI) as discussed in Section \ref{AA} “ANTIANDROGENS” is recommended in more extreme cases to completely halt any loss. Topical minoxodil 5\% is the only thing that works to firm up your hairline beyond hormones alone, but keep in mind that aside from miracle cases, you’re only saving dying/dormant follicles. Dead follicles don’t come back.

If this alone is insufficient for you, hair transplant technology has improved significantly. The Follicular Unit Extraction (FUE) procedure is what you want to look into. Here is where in the future I will link a guide written by an expert on getting insurance to cover that, once she writes it. This is peer pressure. Watch this space.

\subsection{Does exercise affect feminization?}

Probably. HRT causes gradual body recomposition, so you can help encourage your body to shift through exercise. Keep in mind that this process is VERY SLOW, so it is crucial that you eat enough to fuel how patient you have to be. The growth hormones from muscle stimulation via strength training also play a role in breast development, so it’s probably a good thing even aside from the rest of the obvious health benefits of exercise.

This is NOT just the writer’s barely-disguised fetish; strength training is important for your health! I mention this because a lot of trans women believe that touching a dumbbell will make them look like the hulk. I get it, but if you have no testosterone in you and you aren’t on steroids, then you aren’t going to look like that. Let alone the constant time, effort, and diligence required to even get close.

\subsection{What should I exercise then?}\label{11-16}

Cardio is useful for living which is important. Lower body exercises will fill out your hips and glutes to accentuate your figure. Upper body exercises will improve your posture and support your breasts which will make them look bigger. In other words, everything. You’re on estrogen. Have you seen cis women athletes? Exercise will feminize you.

\href{https://docs.google.com/document/d/1-NyE5EY5TTaRRMhk7HlTbKJ7HifjEsA4jlDO1qKQVl0/edit?tab=t.0}{This guide was shared with me} \textcolor{red}{(Warning: Google Docs link)} and looks to be a good starting place. I will note that there aren't particular exercises that feminize vs masculinize as bodies don't work like that, but you may wish to focus more on lower body exerices and flexibility more than the typical lifter.

\subsection{Can estrogen really cause height shrinkage?}

Yes. It is possible that it’s related to water content changes within tendons and ligaments, but it is not something that has been studied so the cause is fully speculation. Scientists: free study idea!

\subsection{Can estrogen really cause foot shrinkage?}

Yes. See above.

\subsection{Can estrogen really cause any other kinds of shrinkage?}

Well, “use it or lose it” like they always say.

\subsection*{Sexual Health}
\addcontentsline{toc}{subsection}{\textemdash{} Sexual Health}

\subsection{How do I improve erectile function on HRT?}\label{11-20}

Aside from using it regularly, ways to improve erectile function include: 1) Improving your fitness and physical health, particularly your cardiovascular ability; 2) consider medication like \textit{tadalafil} or \textit{sildenafil}; and 3) consider testosterone supplementation (see Section \ref{T} “TESTOSTERONE”).

If you would like to read a longer explanation for how erectile function works, \href{https://stainedglasswoman.substack.com/p/how-to-maintain-your-penis-function}{this Substack article} provides a good overview of the topic.

\subsection{How do I increase cum/pre-cum volume on HRT?}

Don’t be embarrassed, it’s a common question. Sunflower lecithin and pygeum increase both of those. It seems to also make a difference for vaginal wetness and arousal for those who have had bottom surgery, but data and anecdotes are limited so it’s hard to say. Otherwise just be sure you drink enough water and have your nutrition in check.

\subsection{Can I lactate on HRT?}

Yes. Domperidone, fenugreek, sunflower lecithin, ample estrogen, and ample progesterone. Get a pump. Knock yourself out.

It should be noted that domperidone has side effects and risks associated with it, and that ability to lactate does not affect breast development. Newman-Goldfarb protocols would be what you want to look into.

\subsection{Can HRT change your senses and your perceptions, i.e. smell?}

You very likely were dissociated and depressed for years prior to starting HRT. The world is more vibrant now because you are no longer dissociating 24/7. The wonders of modern medicine!

It can, however, directly change your eye prescription. That can definitely happen.

\subsection{Can HRT change your sexuality?}

Similar to being dissociated as with above, HRT often incurs a lot more openness and acceptance with yourself which can cause a shift in how your sexuality presents itself. It is largely a semantics argument as to whether that is chemical or behavioral. A matter of perspective. 

\subsection{Should I be on PrEP?}

\textbf{Yes.}

\subsection*{Medical Malpractice}
\addcontentsline{toc}{subsection}{\textemdash{} Medical Malpractice}

\subsection{I heard that injections are actually less stable because you do them less frequently. Is that true?}

Only if you follow the dipshit WPATH SOC 8 guidelines that list a recommended regimen of \textit{estradiol valerate} or \textit{estradiol cypionate} in the range of 5-30mg every two weeks which, to be abundantly clear, you absolutely should never do in a million years. “Do no harm”, my ass. 

\subsection{But my doctor said-?}

The average doctor has essentially no training in anything related to trans healthcare, and \href{https://www.endocrine.org/news-and-advocacy/news-room/2017/endocrinologists-want-training-in-transgender-care }{4/5 endocrinologists have never had any formal training in trans healthcare}. It is most likely that you are their first trans patient and that they are inexperienced in the practical elements of managing a trans patient. Even among doctors who care a lot, they are often limited by conservative standards of care that they are forced to follow which do not always align with the care best for you. See above.

Please also be aware of “trans broken arm syndrome”, aka the tendency of doctors to blame everything on HRT. If your arm is broken, it's probably not “because of those hormones”!

And I should put this as a separate question but I don't want to break the formatting: in line with medical malpractice, there is no situation in which it is reasonable for a doctor to request to see or feel your breasts to “monitor growth” or for any other reason. It is far less common these days, thankfully, but it is sexual assault and completely unacceptable.

\subsection{My doctor won’t prescribe me injections. What do I do?}

Attempt to convince them, replace them, or seek DIY sources. Do not let a gatekeeping medical establishment prevent you from receiving the appropriate care that you deserve. \textbf{The most crucial aspect of interfacing with the medical system while trans is that you have to advocate for yourself. }This is compounded with disability, ethnicity, and other afflictions that scare doctors like womanhood.

\subsection{How does HRT for menopausal cis women relate to HRT for trans women?}\label{11-29}

While we generally have different goals and crucially have very different dosage requirements, there is an immense amount of overlap in experience for trans women and menopausal cis women. Medical misogyny in the form of incompetence, dismissiveness, antagonism, and/or misinformation is something that we unfortunately both experience. It is for this reason that it is paramount to build solidarity on this front. To give an example of what I mean, \href{https://www.youtube.com/watch?v=W0XW6av2wLQ}{the first 30-40 minutes of this interview} will likely sound extremely familiar to you if you would like to raise your blood pressure. The interviewee herself notes the connection too! The WHI ruined the lives of countless women.

\subsection*{Intersexuality and Comorbidities}
\addcontentsline{toc}{subsection}{\textemdash{} Intersexuality and Comorbidities}

\subsection{What’s up with Ehlers-Danlos Syndrome?}

This connective tissue disorder doesn’t actually relate to HRT but a lot of trans people have it so congrats in case this is how you learned that you do too. Aside from general cardiovascular long term concerns to maybe look into, keep up with strength training so that your joints work. Look into that elsewhere though. See Question \ref{11-16}.

\subsection{What kind of intersex things should I keep in mind?}

Throughout this guide, I have mentioned intersex conditions vaguely. Below is a short list of things that might be useful for you to know in your travels for yourself or for a friend. 

\subsection{What’s up with Klinefelter Syndrome?}

This is a relatively (considering chromosomal mutations) common intersex-related condition that some trans women might not realize that they have as the two can overlap. It generally presents as low testosterone at the start of puberty. Good for you to know the name, just in case.

\subsection{What’s up with Persistent Müllerian Duct Syndrome (PMDS)?}

Another “I’m putting this here because this might be the first time you’ve even heard of the term” intersex-related condition that can affect some trans women, however few that may be since we don’t have numbers. The possible presence of an underdeveloped uterus leads to some possible complications and oddities. You probably extra want to have progesterone to avoid uterine cancer risks.

\subsection{What's up with ovotesticular syndrome?}

This intersex condition in particular can cause early level fluctuations which made lead to confusing test results due to the presence of both ovarian and testicular tissues, either separate or combined in an \textit{ovotestis}. This presents in many different ways which HRT can interact with as you begin suppressing \textit{luteinizing hormone} (LH). A uterus may or may not be present, multiple sets of gonads could be present, and/or it could look outwardly typical.

\subsection{What’s the difference between intestinal cramps and uterine cramps?}\label{11-35}

These are commonly misattributed in early transition as a symptom of intersex conditions. Intestinal cramps are widespread and diffuse across your abdomen, whereas uterine cramps are highly concentrated in a location somewhere below your belly button and tend to be sharp stabs/contractions in rapid succession. Like the inside of your body is used as a stress ball. Very different!

\subsection{What about other intersex conditions?}

I have listed a few notable ones, but there are far more expressions and ways of testing them that go far beyond the scope of this guide. Anecdotally, prevalence is higher than average among trans people so basic familiarity with this is useful.

\subsection*{Oddball Questions}
\addcontentsline{toc}{subsection}{\textemdash{} Oddball Questions}

\subsection{Many DIY sources only take crypto. Is that required? How does that work?}

There are other guides that cover this in better depth than I can on how to use crypto safely, including some vendors who have their own guides. But yes, crypto is often required for a lot of reasons. “Crypto” means a lot of things, but using it as a currency was the original point after all. It’s mostly just a pain in the ass. Monero (XMR) is good.

\subsection{What about Selective Estrogen Receptor Modulator (SERM) drugs for nonbinary regimens?}

Some people use SERMs as a part of a transition that is not looking to feminize as much for a more androgynous look, but it’s pretty much entirely uncharted waters thus why their mention is otherwise absent from this guide. You’re on your own if that’s something you want to explore, so please be safe. I don’t personally rate them very highly as I have not seen much to suggest that they work well for how people usually think or want them to work, at least not without a lot more caveats, but obviously there are people who like them. It's just not something I feel comfortable giving recommendations for.

The various proposed nonbinary regimens are often highly individualized because they are specific to what a persons' particular goals are. All HRT should be individualized to a degree, but there is often more variation in desired outcomes when people ask about androgyny. Hormonally, it is nontrivial. Everything stated in this guide should be treated solely as a starting place if you are wanting to experiment with something more complicated, but do remember that there is much more to achieving transition goals than just hormones alone.

\subsection{Are things like “herbal HRT” or “phytoestrogens” legitimate?}

\textbf{No.} If someone is telling you they have “herbal HRT”, they are telling you they have snake oil. The only thing that is going to feminize you is estrogen, not plant estrogens. No amount of “natural” products are a replacement for estrogen itself. This isn’t a common scam and you probably already know, but just in case you run into it, now you know for sure. If it smells like bullshit, it’s probably bullshit. Unless we’re talking about bug steroids in which case yeah those are actually cool. Won’t feminize you though.

\subsection{Is the Reddit Doctor that people constantly talk about Good?}

No.

\subsection{I hear DIY estrogen is made in a bathtub. Is that true?}

No. I honestly have no idea where or why this joke started that people now take seriously, but there’s no step in any process where a bathtub would even be considered. Don’t believe everything you read online. I don’t even know what you could even theoretically do with a bathtub, unless you think estrogen vials are full of the bathwater of trans women. I don’t know why you would think that though. It’s obviously cum.

\subsection{How does HRT affect fertility?}\label{11-42}

It is important to understand that this is extremely understudied so exact figures cannot be stated, and given the seriousness of pregnancy, I urge you to practice safe sex and lean on the side of caution where possible. HRT itself can, and likely will, make you infertile eventually, but only through full suppression of the HPG axis (See Question \ref{2-3}) over a long time span. In other words, if you haven't had bottom surgery of any kind and you are on an HRT regimen that is less capable of HPG axis suppression (such as pills), then this is more of a consideration.

\textbf{If the HPG axis is not suppressed then it is fully possible to impregnate someone}, and the timeline for sperm maturation is long enough that this is true even after the HPG axis has been initially suppressed for \textbf{multiple months}. Please take this very seriously. Full HPG axis suppression for at minimum six months, perhaps closer to a year out of an abundance of caution, is recommended.

\subsection{Is infertility from HRT reversible?}\label{11-43}

It is theoretically possible to reverse HRT-induced infertility, assuming you weren't already infertile prior to HRT (a large assumption!), but there are not many documented cases of this so the full efficacy of fertility restoration after long-term HRT is unknown. The process would involve restarting the HPG axis with a variety of medications along with entirely stopping HRT, which would in essence require a hormonal detransition for likely six months at minimum, and even then sperm quality is not certain or guaranteed. It is not something that should be planned for, to say the least, so planning around it would be wise. A sperm bank would be recommended before or early in HRT, financially permitting, if potential biological children are a priority and if a future relationship where that is possible/desired is likely.



\section{CREATINE}

\subsection{What is creatine?}

Creatine is an organic compound in your muscles and in your brain. It recycles ADP into ATP which is important for energy production in your body, especially initial high burst applications before other energy systems take over.

\subsection{Isn’t it like a steroid or something that bodybuilders use?}

No. Bodybuilders and athletes like it because having more energy means more activity before getting tired. They aren’t the only ones who use it since it is basically the \#1 supplement in terms of things that are actually useful and are actually researched. 

\subsection{How is creatine related to HRT?}

It isn’t! But it’s something I yell about because I think it’s good and I am tired of repeating myself because people keep asking and you’re reading this anyway, aren’t you? I love a captive audience. My standup routine is at the bottom.

\subsection{Okay well why should I take creatine then?}

What a great question! It’s good for your brain and your muscles. Creatine is often found in relatively low concentrations for many people depending on their diet, especially people who don’t eat meat. There is compelling research about various chronic fatigue and post-viral conditions (long COVID in particular) being related to depleted creatine reserves in the brain, so some people find cognitive benefits from supplementing it. It isn’t magic but it is dirt cheap so it is worth trying in my opinion.

\subsection{What are the forms?}

Just \textit{creatine monohydrate} powder is what you want. The pills tend to be low dosage and are up charging you anyway, while gummies often destroy the creatine in the creation of the gummy. A lot of brands include creatine in some sort of mix but the pure stuff is usually cheaper.

\subsection{How do I take it then?}

The general recommendation is 5-10g daily dissolved in some sort of liquid. It dissolves best in things that aren’t just water. It’s mostly flavorless, so just throw a scoop or two in your coffee or a smoothie and call it a day. It can be a little chalky or gritty depending on the quantity and the fluid.

\subsection{Does it matter when I take it?}

Not really. It doesn’t have an immediate effect like that which is why it’s silly that it’s microdosed in pre-workout mixes. Take it whenever it’s convenient for you.

\subsection{How does it work then?}

It builds up in your body to a maximum level of saturation over a week or two. Then you just maintain that and reap the rewards (of maybe feeling better).

\subsection{Do I have to do a “loading” phase of taking more at first?}

Probably not. Unless you’re on some sort of intense training time crunch or something, this probably doesn’t matter at all. Just take whatever is convenient with some regularity.

\subsection{What are the side effects?}

Slight weight gain may be possible because of increased water weight in your muscles (which to be clear is Good, so don't be alarmed). If you don’t take it with water, or if you take too much at once, you might get a tummy ache. Ouchie.

\subsection{Who shouldn’t take it?}

People with kidney issues. Not because it causes them, but because creatinine (Different spelling! Creatine becomes creatinine) is used as a marker in lab tests for a number of kidney issues and supplementing might give a false positive. Just keep it in mind.

\subsection{Do you have any brand recommendations?}

No. It shouldn’t really matter. Just get whatever seems reputable and is a reasonable price. I’d give a recommendation for the one I like but when I asked the brand for affiliate link they turned me down, so their loss! No free clout.

\subsection{You seriously put creatine into this document, huh?}

Yeah it’s pretty funny. It’s not my fault that I joked about it and people told me it legitimately helped them because now I feel obligated to keep talking about it!!!

 

\section{CLOSING REMARKS}

If any of the following are true:

\begin{itemize}
\item you are still mad at me despite the disclaimer;

\item you spotted an issue or typo;

\item you have a clarifying question that should be put into the text;

\item you have an objection that hopefully isn’t an Uhm Ackshually;

\item you wish to sing my praises;

\item you wish to pledge fealty; 

\item you wish to send tithes my way;
\end{itemize}

Then please feel free to contact me and I’ll see what we can do. Bluesky is the easiest contact point, and you can DM me for my Signal. Otherwise, thank you for reading and I hope it helps.

\textbf{If you would like to donate to support this project,} \href{https://cash.app/Katitties}{CashApp}, \href{https://ko-fi.com/katitties}{Ko-Fi}, and \href{https://account.venmo.com/u/katitties}{Venmo} all work. I appreciate it!

And lastly: \textbf{The most important thing that you can do as a trans person is to live.} For as much as this document is a manual, it is in equal measure a message to you as a trans person that your existence is a gift upon the world, your presence is a blessing on those around you, and that you deserve to be treated with respect. Even if you do nothing else, your life is a feat worth praising. Thank you.



\section*{FRIENDS OF PGHRT}\label{FOPGHRT}
\addcontentsline{toc}{section}{FRIENDS OF PGHRT}

Across this document is a scattering of links to other guides and resources. Below is a consolidation of them which will also include more links to external resources as time goes on, ideally by other trans people. For the privacy minded or noided, note that some of these are Google Docs links.

\begin{enumerate}
  \item \href{https://startwith4mgestradiolenanthateweeklyandtestatonetothreemonths.com/}{SW4EEWATAOTTM} - TL;DR for PGHRT
  \item \href{https://hrtcafe.net/}{HRT Cafe} - HRT Resource Aggregator
  \item \href{https://transfemscience.org/}{Transfeminine Science} - Informational resource for trans medical literature
  \item \href{http://estrannai.se}{Estrannai.se} - Estradiol Pharmacokinetics Playground
  \item \href{https://globoho.moe/}{Globoho.moe} - Thailand Orchiectomy Medical Tourism Travel Guide 
  \item Julia's FUE Guide - COMING SOON, I'M BULLYING HER TO WRITE FASTER
  \item \href{https://docs.google.com/document/d/1-NyE5EY5TTaRRMhk7HlTbKJ7HifjEsA4jlDO1qKQVl0/edit?tab=t.0}{Sky's Feminine Figure Beginner Program} - An exercise regimen geared towards trans fems
  \item \href{https://docs.google.com/document/d/114sztSw1aVWM2pXLDl9NrHklyvewz3EmFiHiisjM71k/edit?tab=t.0}{Sky's Diet 101} - A guide towards adjusting weight in a healthy way
  \item \href{https://stainedglasswoman.substack.com/p/how-to-maintain-your-penis-function}{How to Maintain Erectile Function on HRT} - A longer form explanation on the "use it or lose it" phenomenon
  \item \href{hhttps://docs.google.com/document/d/1DXFxzN0XTudPZez_SO61fpqncRLPH_Be_QG_8Pcz9LU/edit?pli=1&tab=t.0}{Biohax Guide Googleslop Edition} - Trans Masc DIY Guide
\end{enumerate}

\section*{ABOUT THE AUTHOR}
\addcontentsline{toc}{section}{ABOUT THE AUTHOR}

Katie Tightpussy is an award-winning author and professional trans woman with nearly a decade of experience in the field of transgender. Her accomplishments include transiferating her sex through the novel technique of cross-sex hormone injections, being physically unable to shut up, and utilizing a very fortunate set of hyperfixations as they relate to transbobulation of the humors. She spends her days in the idyllic rural countryside of Los Angeles scheming of new ways to achieve world domination and enjoys riding her bicycle. Media inquiries can reach her agent at \href{http://katietightpussy.com}{katietightpussy.com}.

 

\section*{DISCLOSURES}
\addcontentsline{toc}{section}{DISCLOSURES}

No robot girls were harmed in the making of this document, including any usage of generative large language models. The author does not endorse any reproduction without attribution nor scraping of this work. Leave those poor robot girls alone.

The author declares an attraction towards women and acknowledges a potential conflict of interest for the existence of more beautiful trans women in the world.

 

\section*{ACKNOWLEDGEMENTS}
\addcontentsline{toc}{section}{ACKNOWLEDGEMENTS}

Though the text is primarily my voice, this document would not be even half as good without the contributions, feedback, and suggestions from others involved at every step along the way. A good reminder as ever that transition is not something best done alone.

Many thanks to Q, R, RM, and S in alphabetical order for close review and generally being fun nerds to talk to; love y’all. Special thanks to CB and J for close review that also inspired some very good bits. Thanks to KG for additional intersex information. Thanks to w [sic] for additional injection resources. Thanks to BIR collectively for a plethora of crucial nerd nitpicks. Appreciation for general review from C, JTP, K, S, and V. Thanks to everyone on Bluesky who encouraged me to write this up in the first place, and everyone over the years sharing knowledge. And of course: much appreciation to all HRT nerds, even when we disagree, since we’re all trying to do the best for our community where we’ve otherwise been let down. Keep up the good work everyone. 

Shout out to my IB Chemistry HL teacher many years ago who quite reasonably doubted my studiousness even though I’m now putting much of that knowledge to use for the art of transsexuality; go figure. 

 

\section*{CHANGELOG}
\addcontentsline{toc}{section}{CHANGELOG}

\noindent \href{https://github.com/Juicysteak117/pghrt/}{Source code available here on GitHub.}

\noindent Full Compilation Datetime: \DTMnow

\noindent(There aren't LaTeXML bindings for \texttt{datetime2}, \texttt{hanging}, or \texttt{hyphenat}, so the formatting is slightly ugly. If you'd really like to help me out, please write those bindings!!!)

\noindent 2025-08-20: Initial release. 15.9k words.

\noindent 2025-08-20: A lot of typos and minor verbiage tweaks. Added Question \ref{8-18}.

\noindent 2025-08-21: Typos grow on trees. Added Question \ref{5-27}.

\noindent 2025-08-21: More tweaks. Opted to remove “WHY PROG” from Question \ref{8-17}. 17.0k words.

\noindent 2025-08-22: Nitpicks, clarifications, and typos. 17.2k words.

\noindent 2025-08-24: A few more twinks sorry tweaks. 17.3k words.

\noindent 2025-08-27: How long until remaining typos are embarrasing? 17.3k words.

\noindent 2025-08-28: Reduced ambiguity in a few areas. 17.4k words.

\noindent 2025-08-29: Additional clarity for frequencies in Section \ref{td}. 17.5k words.

\noindent 2025-09-01: Sisyphus boulder meme captioned fixing typos dot png. 17.5k words.

\noindent 2025-09-07: Added donation links per request. That's very kind. 17.5k words.

\noindent 2025-09-07: Few more tweaks. Clarified an additional common progestin. 17.6k words.

\noindent 2025-09-19: Added Question \ref{4-16} plus tweaks. 17.7k words.

\noindent 2025-09-23: A wide variety of clarifications up and down the line. 18.1k words.

\noindent 2025-09-24: Added an important note about surgery to Question \ref{11-1}. 18.3k words.

\noindent 2025-09-24: “Katie my doctor told me-” It never ends. 18.5k words.

\noindent 2025-09-26: Another pass of clarification edits. Yes I should have a git diff. Sorry that I don't. I thought I'd be done by now anyway! 18.7k words.

\noindent 2025-09-30: Added some cross references for clarity. 18.7k words.

\noindent 2025-10-02: More cross references. Likely will do another pass. 18.8k words.

\noindent 2025-10-02: Added a big bold warning about recapping to Question \ref{5-13} because SOMEONE didn't watch the video smh. 18.9k words.

\noindent 2025-10-10: Added Question \ref{11-42} and Question \ref{11-43} per request. Honestly I just forgot about fertility being a thing lol. Also added the Friends Of PGHRT postword section. 19.5k words.

\noindent 2025-10-10: Added Gretchen's Version (.txt) and fixed formatting. 19.5k words.

\noindent 2025-10-11: Added permalinks to everything, yay! And finally made a git repo. Look at me being a big girl, wow. 19.5k words.

\noindent 2025-10-11: Added an external link to Question \ref{11-20}. 19.6k words.

\end{document}